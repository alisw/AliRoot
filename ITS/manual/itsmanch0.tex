\chapter*{Introduction}
\section*{Purpose of this document}

This document is intended to both explain who to use the ALICE simulation and
reconstruction code with respect to or using the ITS detector as the
examples. This document is also to explain how to add new code to the ITS
simulation and reconstruction library, and who the existing ITS simulation and
reconstruction code works. All comments from every user is greatly encouraged.

\section*{How to Run AliRoot}

At this point, we will assume that AliRoot has been compiled and all the
necessary environment variables have been defined, including your path. We will
assume you are in an appropriate directory, for example \texttt{/data} or some
such thing.

to start with type, at your shell prompt, aliroot to start the program.

\scriptsize
\begin{verbatim}
# aliroot
Constant Field Map1 created: map= 1, factor= 1.000000
  *******************************************
  *                                         *
  *        W E L C O M E  to  R O O T       *
  *                                         *
  *   Version   2.26/00  10 November 2000   *
  *                                         *
  *  You are welcome to visit our Web site  *
  *          http://root.cern.ch            *
  *                                         *
  *******************************************

FreeType Engine v1.x used to render TrueType fonts.
Compiled with thread support.

CINT/ROOT C/C++ Interpreter version 5.14.58, Oct 24 2000
Type ? for help. Commands must be C++ statements.
Enclose multiple statements between { }.

WELCOME to ALICE

root [0]
\end{verbatim}
\normalsize

This will initialize ROOT and load all of the ALICE libraries. At this point
you can do anything you can do in ROOT in addition you have access to every
class defined in the ALICE libraries, including the ALICE global variables. One
very useful ALICE global variable is \texttt{gAlice} which is of type
\texttt{AliRun}. Of the many function defined in the class \texttt{AliRun} are
\texttt{Init(const char* setup="Config.C")} and 
\texttt{Run(const char* setup="Config.C")}. \texttt{Run} both executes
\texttt{Init} and starts executing an ALICE detector simulation. Both of these
functions functions load and execute a configuration file. By default this
configuration file is called \texttt{Config.C}. If such a file exists in your
local directory, for example \texttt{/data}, or if there is no such file in
your local directory it will execute the file
\texttt{\$ALICE\_ROOT/macros/Config.C}. Some other configuration file can be
run simplely by entering that file's name as the argument, for example
\texttt{Init("MyConfig.C")} or \texttt{Run("MyConfig.C")} where
\texttt{MyConfig.C} is either in your local directory (\texttt{/data}) or in
\texttt{\$ALICE\_ROOT/macros}. Of course the full path of \texttt{MyConfig.C}
can be used.

Now lets assume you just want to simulate one event using the standard
\texttt{Config.C} file. This is done simply as
\scriptsize
\begin{verbatim}
root [0] gAlice->Run()
Warning in <AliRun::SetField>: Invalid magnetic field flag:  -999; Helix trackin
g chosen instead

Warning in <AliFRAMEv1::ReadEuclidMedia>: file: $(ALICE_ROOT)/Euclid/frame.tme i
s now read in

Warning in <AliPIPEv0::ReadEuclidMedia>: file: $(ALICE_ROOT)/Euclid/pipe.tme is 
now read in

Warning in <AliITSv5::ReadEuclidMedia>: file: /home/CERN/aliroot/dev/Euclid/ITSg
eometry_5.tme is now read in


 MZSTOR.  ZEBRA table base TAB(0) in /MZCC/ at adr   281557647    10C83A8F HEX

 MZSTOR.  Initialize Store  0  in /GCBANK/
          with Store/Table at absolute adrs    33632021   281557647
                                        HEX     2012F15    10C83A8F
                                        HEX    F138F25A           0
                              relative adrs  -247926182           0
          with     1 Str. in     2 Links in   5300 Low words in 2999970 words.
          This store has a fence of   16 words.
\end{verbatim}
\normalsize
\vdots
lots more messages

\vdots
\scriptsize
\begin{verbatim}
 TOC1      0.171%; TSSW      0.001%; TSWC      0.083%; TSCE      0.000%; TWES      0.001%;
 TSWB      0.009%; TPEL      2.240%; TPMW      0.901%; TPEW      0.128%; TESR      0.000%;
 TESB      0.116%; TPLS      0.019%; TPUS      0.023%; TPSS      0.000%; THVM      0.012%;
 TPSR      0.052%; THVL      0.025%; FLTA      0.003%; FLTB      0.000%; FLTC      0.005%;
 FMYA      0.016%; FMYB      0.014%; FMYC      0.019%; FPLA      0.023%; FPLB      0.080%;
 FPLC      0.025%; FSTR      0.138%; FNSF      0.012%; FMYX      0.002%; FGRL      0.004%;
 FPAD      0.020%; FPEA      0.039%; FPEB      0.007%; FPEC      0.062%; FECA      0.058%;
 FECB      0.058%; FECC      0.053%; FWAA      0.096%; FWAB      0.148%; FWAC      0.073%;
 FBPA      0.057%; FBPB      0.018%; FBPC      0.085%; UAFI      0.109%; UAFM      0.018%;
 UAFO      0.044%; UAII      0.001%; UAIM      0.001%; UAIO      0.001%; UCFI      0.259%;
 UCFM      0.445%; UCFO      0.312%; UCII      0.000%; UCIM      0.000%; UCIO      0.000%;
 UL01      0.022%; UL02      0.006%; UL03      0.226%; UL04      0.006%; UL05      0.012%;
 UL06      0.001%; UL07      0.002%; UL08      0.081%; UL09      0.034%; UL10      0.038%;
 UL11      0.013%; TRD1      0.003%; TRD2      0.004%; TRD3      0.000%; BR2_      0.018%;
 CB2_      0.011%; R1R2      0.018%; R2R2      0.898%; R3R2      0.003%; R3L2      0.001%;
 R1R1      0.414%; R2R1      0.010%; R3R1      0.009%; R1L1      0.036%; R3L1      0.001%;
 CA02      0.016%; CG02      0.000%; CA03      0.042%; CG03      0.000%; EMCA      0.013%;
 PTXW      0.013%; PUFP      0.005%; PTCB      0.008%; PPAP      0.005%; PXTL      0.072%;
 PASP      0.026%; MPPS      0.004%; UAPP      0.000%; LCPP      0.037%; DW11      0.000%;
 DV11      0.000%; DPPB      0.659%; DPFE      0.157%; DPMD      0.000%; DIQU      0.016%;
***************************************************************************
root [1] 
\end{verbatim}
\normalsize

At this point a file called \texttt{galice.root} is created in your present
directory. This file contains all of the ``hits'' produced by the simulation,
all of the particle information, all of the detectors that were defined in the
simulation, and a lot of other information.

One nice thing to do now is to run the hit display program so that you can see
the hits you have just produce in the file \texttt{galice.root}. Either from
your present \texttt{aliroot} session, a new \texttt{aliroot} session, or
from a new \texttt{root} session, you can run the standard macro
\texttt{display.C}. Because you should have \texttt{\$ALICE\_ROOT/macros} in
your path, this is easily done as follows.

\scriptsize
\begin{verbatim}
root [1] .x display.C
\end{verbatim}
\normalsize

You should see the following screen appear (Figure~\ref{ITSMan:ch0:display0}).

\begin{figure*}[htbp]
  \begin{center}
    \leavevmode
    \epsfig{file=display_Canvas.eps,width=12cm}
    \caption{Typical ALICE hit display output. Note that the geometry being
    displayed is a simplified geometry, not that used by the AliRoot
    simulation.}
    \label{ITSMan:ch0:display0}
  \end{center}
\end{figure*}

Just like from \texttt{root}, to exit from \texttt{aliroot} just type
\texttt{.q} at the prompt.

\scriptsize
\begin{verbatim}
root [2] .q
\end{verbatim}
\normalsize

For a more interactive way of running \texttt{AliRoot}, instead of typing
\texttt{root[0] gAlice->Run()} you can run a very nice macro called
\texttt{menu.C} also located in \texttt{\$ALICE\_ROOT/macros}. Instead of
running a simulation, it create a menu that that will let you either run an
\texttt{AliRoot} simulation or do a number of other nice things, see
figure~\ref{ITSMan:ch0:menu0}. Please note, that if you want to display any
pictures, first they can take a lot of time and memory, second you have to run
\texttt{Init} first so that the detectors you want to display are defined.

\begin{figure*}[htbp]
  \begin{center}
    \leavevmode
    \epsfig{file=menubar.eps,width=5cm}
    \caption{This shows a number of ``Automated'' commands that can be
    performed. Simply by clicking on one of these buttons the action will take
    place. The ``Trees'' refers to the tree of geometric objects that make up
    the defined AliRoot geometry as defined in \texttt{Config.C}. The
    ``Pictures'' refer to pictures of the Alice geometry as defined in
    \texttt{Config.C}. Since these are the detailed pictures they may require a
    lot of time and memory to execute. ``Hide'', ``Shading'', and ``Box Clip''
    refer to how the pictures will be displayed. ``Run Lego'' will produce a
    radiation map of different parts of the defined Alice detector by creating
    and transporting Geantinos~\cite{GeantMan:Geantino}, a non-interacting
    particle used only to measure radiation length between points along it's
    straight line path.}
    \label{ITSMan:ch0:menu0}
  \end{center}
\end{figure*}

\section*{Controling your Run: The Config.C file}

There are basicly two files that control nearly every aspect of an
\texttt{AliRoot} simulations. The first and most important is the file
\texttt{\$ALICE\_ROOT/macros/Config.C}. This file will be described in more
detail below. The second file \texttt{\$ALICE\_ROOT/data/galice.cuts} sets the
non-default energy cuts for a large numbers of materials. For the ITS, Every
material has energy-loss via delta rays turned on. For each sensitive material
the lowest energy $\gamma$, $e^{\pm}$, bremsstrahlung, and $\delta$-ray have
their lowest energy set to $7.0\times 10^{-5}$ GeV or $70$ KeV. This is done to
assure that low enough energy $\delta$-rays are properly taken into account for
the signals produce in each detector.

The \texttt{Config.C} file\footnotemark determines just about every thing there
is to about how \texttt{AliRoot} will run. Lets take a look at the first part
of this file.

\footnotetext{The name of the file isn't important. It can be specified either
as an argument to \texttt{gAlice->Run(``Another\_Config.C'')} or as an
argument to \texttt{gAlice->Init(``Another\_Config.C'')}. The default file
name is \texttt{Config.C} and is first looked for in your present directory and
if no such file is found then it uses the one in 
\texttt{\$ALICE\_ROOT/macros}. }

\scriptsize
\begin{verbatim}
0001  void Config()
0002 {
0003 
0004 new AliGeant3("C++ Interface to Geant3");
0005 
0006 //=======================================================================
0007 //  Create the output file
0008 
0009 TFile *rootfile = new TFile("galice.root","recreate");
0010 rootfile->SetCompressionLevel(2);
0011 TGeant3 *geant3 = (TGeant3*)gMC;
\end{verbatim}
\normalsize

This file is a \texttt{C++} macro/program. The first line gives the routine
name which must be \texttt{Config()}. Line \texttt{0004} Creates the
\texttt{AliRoot} interface class to GEANT 3.12. Line \texttt{0009} creates the
root output file and its pointer \texttt{rootfile}. Line \texttt{0010} sets the
level of compression to be used in writing this file. Level 2 is a lot of
compression, level 0 is no compression\footnote{see \texttt{TFile} in the
\texttt{ROOT} documentation for a full description.}. Line \texttt{0011}
retrieves the basic transport Monte Carlo that was set at line
\texttt{0004}. Since this is GEANT3.21 this pointer is cast as
\texttt{TGeant3}. This will allow us to set and modify the simulation according
to GEANT3.21.

\scriptsize
\begin{verbatim}
0012 //
0013 // Set External decayer
0014  AliDecayer* decayer = new AliDecayerPythia();
0015  decayer->SetForceDecay(all);
0016  decayer->Init();
0017  gMC->SetExternalDecayer(decayer);
\end{verbatim}
\normalsize

In this next section of the file, the default operation of GEANT 3.21 is
modified by replacing the particle decay mechanism with that of Pythia. This
has the advantage of treating all particle decays in the same way but also make
it easier to determine the parent particles from their decayed products. Line
\texttt{0014} creates the Pythia decay routines, \texttt{0015} sets a flag
in Pythia to decay all particles, and \texttt{0016} initiates the Pythia
package. Line \texttt{0017} replaces the default decay routine in the Monte
Carlo (GEANT 3.12 in this case) with that of Pythia. Next we set some flags
that are specific to GEANT 3.21.

\scriptsize
\begin{verbatim}//
0018 //
0019 //=======================================================================
0020 // ******* GEANT STEERING parameters FOR ALICE SIMULATION *******
0021 geant3->SetTRIG(1); //Number of events to be processed 
0022 geant3->SetSWIT(4,10);
0023 geant3->SetDEBU(0,0,1);
0024 //geant3->SetSWIT(2,2);
0025 geant3->SetDCAY(1);
0026 geant3->SetPAIR(1);
0027 geant3->SetCOMP(1);
0028 geant3->SetPHOT(1);
0029 geant3->SetPFIS(0);
0030 geant3->SetDRAY(0);
0031 geant3->SetANNI(1);
0032 geant3->SetBREM(1);
0033 geant3->SetMUNU(1);
0034 geant3->SetCKOV(1);
0035 geant3->SetHADR(1); //Select pure GEANH (HADR 1) or GEANH/NUCRIN (HADR 3)
0036 geant3->SetLOSS(2);
0037 geant3->SetMULS(1);
0038 geant3->SetRAYL(1);
0039 geant3->SetAUTO(1); //Select automatic STMIN etc... calc. (AUTO 1) or manual (AU TO 0)
0040 geant3->SetABAN(0); //Restore 3.16 behaviour for abandoned tracks
0041 geant3->SetOPTI(2); //Select optimisation level for GEANT geometry searches (0,1,2)
0042 geant3->SetERAN(5.e-7);
0043 
0044 Float_t cut    = 1.e-3; // 1MeV cut by default
0045 Float_t tofmax = 1.e10;
0046 //             GAM ELEC NHAD CHAD MUON EBREM MUHAB EDEL MUDEL MUPA TOFMAX
0047 geant3->SetCUTS(cut,cut, cut, cut, cut, cut,  cut,  cut, cut,  cut, tofmax);
\end{verbatim}
\normalsize

In lines \texttt{0021} through \texttt{0042} set different GEANT 3.21 specific
options and flags. I will not describe all of these, but will refer you to the
GEANT 3.21 manual. There are a few of possible interest. Line 
\texttt{0021 geant3->SetTRIG(1);} sets the number of events that will be
processed. The default is one. Line \texttt{0022 geant3->SetSWIT(4,10);} sets
how often a line like 
\texttt{GTREVE\_ROOT : Transporting primary track No 74130} will be
displayed. Here the 10 means every 10 primary tracks. Most of the rest turn on
or off different physics processes. These values apply to every material that
isn't listed in \texttt{\$ALICE\_ROOT/data/galice.cuts} or if that material has
its corresponding flag or value set to -1. Similarly line \texttt{0047} sets
the default lowest energy, largest time of flight, for different physics
processes that are not overwritten by the values in the \texttt{galice.cuts}
file.

\scriptsize
\begin{verbatim}
0048 //
0049 //=======================================================================
0050 // ************* STEERING parameters FOR ALICE SIMULATION **************
0051 // --- Specify event type to be tracked through the ALICE setup
0052 // --- All positions are in cm, angles in degrees, and P and E in GeV
0053 AliGenHIJINGpara *gener = new AliGenHIJINGpara(84210);
0054 gener->SetMomentumRange(0,999);
0055 gener->SetPhiRange(0,360);
0056 gener->SetThetaRange(0.28,179.72);
0057 gener->SetOrigin(0,0,0);        //vertex position
0058 gener->SetSigma(0,0,0);         //Sigma in (X,Y,Z) (cm) on IP position
0059 gener->Init();
0060 // 
0061 // Activate this line if you want the vertex smearing to happen
0062 // track by track
0063 //
0064 //gener->SetVertexSmear(perTrack); 
0065 
0066 gAlice->SetField(-999,2);    //Specify maximum magnetic field in Tesla (neg. ==> default field)
\end{verbatim}
\normalsize

In lines \texttt{0053} through \texttt{0059} setup the particle generator to be
used. Line \texttt{0053} defines the type of particle generator to be used. In
this case it will be the ALICE HIJING parameterized particle distribution. 
This can be replace with many others~\footnote{See
\texttt{\$ALICE\_ROOT/macros/Config\_gener.C} for more examples.}. Its
parameter value $84210$ is the number of primary tracks that will be
generated. For simple testing this number it typically set to a value like
50. Lines \texttt{0054} through \texttt{0058} set different parameters for the
generator. Specifically line \texttt{0054 gener->SetMomentumRange(0,999);} set
the limits of the momentum range of the particles that will be generated. Line
\texttt{0055 gener->SetPhiRange(0,360);} sets the angular (degrees) range in
$\phi$ or around the barrel that the partials will be generated in. In this
case the full $\phi$ range will be used. The next line \texttt{0056
gener->SetThetaRanger(0.28,179.72);} sets the $\theta$ angular (degrees) range
for the primary tracks to be generated in. In this case $0.28^{\circ}$ to
$179.72^{\circ}$ covers the pseudorapidity range $-6.0\leq \eta \geq
+6.0$. With these values of $\theta$ and $\phi$, $84210$ primary tracks
represents about $8000$ primary particles in the central one unit of
pseudorapidity, the maximum track density that ALICE has been designed for.

Line \texttt{0057 gener->SetOrigin(0,0,0);} sets the central position of the
event origin. Line \texttt{0058 gener->SetSigma(0,0,0);} set the
$\sigma_{x,y,z}$ range over which multiple events will have their origins
distributed. The mean position being given by the \texttt{SetOrigin} command
above. It is also possible in one event, to distribute each primary track over
a region who's mean is set by the \texttt{SetOrigin} command and who's width is
set with the \texttt{SetSigma} command. If this is what you want then uncomment
out line \texttt{0064}. Line \texttt{0059 gener->Init();} initializes the
particle generation routine defined by \texttt{0053} with any and all
parameters defined later on. Lastly line \texttt{0066
gAlice->SetField(-999,2);} defines what kind, if any, magnetic field should be
used. These values instruct \texttt{AliRoot} to read a magnetic field map from
\texttt{\$ALICE\_ROOT/data/field02.dat}.

\scriptsize
\begin{verbatim}
0067 
0068 Int_t iMAG=1;
0069 Int_t iITS=1;
0070 Int_t iTPC=1;
0071 Int_t iTOF=1;
0072 Int_t iHMPID=1;
0073 Int_t iZDC=0;
0074 Int_t iCASTOR=1;
0075 Int_t iTRD=1;
0076 Int_t iABSO=1;
0077 Int_t iDIPO=1;
0078 Int_t iHALL=1;
0079 Int_t iFRAME=1;
0080 Int_t iSHIL=1;
0081 Int_t iPIPE=1;
0082 Int_t iFMD=1;
0083 Int_t iMUON=1;
0084 Int_t iPHOS=1;
0085 Int_t iPMD=0;
0086 Int_t iSTART=0;
0087 
\end{verbatim}
\normalsize

In these following lines \texttt{0068} through \texttt{0086} a series of flags
are defined to make it easy to turn on or off different detectors. There is one
such flag for each detector/major structural element. A value of 1 turns on
that detector and a value of 0 turns it off. For example, Line 
\texttt{0068 Int\_t iMAG=1;} turns on the magnetic field for all of 
ALICE\footnotemark. Line \texttt{0069 Int\_t iITS=1;} turns on the ITS. Line
\texttt{0073 Int\_t iZDC=0;} turns off the ZDC\footnotemark. Note, in this
case the ZDC will not even be defined in ALIROOT. You will not be able to make
any fancy pictures of it or anything. It simply will not exist.

\addtocounter{footnote}{-1}
\footnotetext{Be car full, if a particular material was defined to have a
magnetic field this may over ride this flag. This should have been fixed, I
haven't confirmed that yet.}

\addtocounter{footnote}{1}
\footnotetext{Since the ZDC is outside of the range of produce primary
particle, in this example, there is no real use to keep it in the simulation.}

Now the following lines in the \texttt{Config.C} file, depending on the value
of the above flags, will create the specific detectors and structural
elements. Since this is an ITS manual, I will show and explain the ITS part of
this remaining file.

\scriptsize
\begin{verbatim}
0138
0139 if(iITS) {
0140 //=================== ITS parameters ============================
0141 //
0142 // EUCLID is a flag to output (=1) both geometry and media to two ASCII files 
0143 // (called by default ITSgeometry.euc and ITSgeometry.tme) in a format
0144 // understandable to the CAD system EUCLID. The default (=0) means that you 
0145 // dont want to use this facility.
0146 //
0147 AliITS *ITS  = new AliITSv5("ITS","normal ITS");
0148 ITS->SetEUCLID(0);
0149 }
0150 
\end{verbatim}
\normalsize

This is the section of the \texttt{Config.C} file that defines what the ITS
is. You will note that if \texttt{iITS} = 0\footnote{A value of 0 is equivalent
to FALSE. A non-zero value is equivalent to TRUE.}, then this section of the
file will not be executed and no ITS will be defined. Line \texttt{0147}
creates the ITS. In this case it creates ITS version 5. There are two necessary
parameters. The first is the name of this detector ``ITS''. The second is a
brief description of the detector ``normal ITS''. Both of these are required so
that AliRoot knows what detector this is. AliRoot does not know about any
detectors and it is only via this detector name that we can later get access to
the information stored about the ITS. Line \texttt{0148 ITS-SetEUCLID(0);} sets
a flag for the ITS describing weather or not to create a file in the
EUCLID-GEANT3 compatible format. By default this should be 0. In the future we
may add additional flags and parameters as needed.

On line \texttt{0147 AliITS *ITS = new AliITSv5(``ITS'',''normal ITS'');} this
create an ITS of type v5. This is the detailed ITS TDR geometry. For a full
description of the possible ITS geometries see section
~\ref{ITSMan:ch2:Detaile_geometry_section}. By default, every detailed geometry
is required to have a corresponding course geometry. In this case that is
v0. All course geometries are required not to generate ``hits'' and therefore
\texttt{AliITSv0} does not and therefore can not be used in any study where
signals from the ITS may be used or required.

\section*{Other things that can be done}

There are a number of macros that have been written to do different
things. Take a look at the files in ending in \texttt{.C} in
\texttt{\$ALICE\_ROOT/macros}. There you will find a number of macros to view
and/or draw different detector geometries. These are the geometries used by in
the simulation and so you will need to execute \texttt{gAlice->Init();}
first. In here are also kept general example macros to read hits and plot some
information, \texttt{newanal.C} for example. Typically each detector group
writes their own detector specific macros and places them in their directories,
\texttt{\$ALICE\_ROOT/ITS/AliITStest.C} which will simulate one event and then
do the ITS detector simulations and create what are called ``digits'' and also
reconstruct these ``digits'' into ``reconstructed points''.

There is also another nice macro, requiring Geant 3, that is menu driven and
lets you display all or part of the Alice geometry. This is located in
\texttt{\$ALICE\_ROOT/TGeant3} and is called \texttt{TGeant3GUI.C}, see
figure~\ref{ITSMan:ch0:TGeant3GUI}. As with the other macros, you will need to
execute \texttt{gAlice->Init();} first. Root also provides some nice features
that you can use. Either in \texttt{aliroot} or in \texttt{root} you can create
a root object called a \texttt{TBrowser}, see
figure~\ref{ITSMan:ch0:TBrowser}. This is a menu driven way to look at or even
histogram the contents of root files or even memory. These macros are always
being modified, deleted, changed, and new one added. Take a look at the Alice
off-line web page \texttt{http://AliSoft.cern.ch/offline/} for a more up to
date and complete description.

\begin{figure*}[htbp]
  \begin{center}
    \leavevmode
%    \epsfig{file=TGeant3GUI_Canvas.eps,width=12cm}
    \caption{This shows a typical menu from running \texttt{TGeant3GUI.C}.}
    \label{ITSMan:ch0:TGeant3GUI}
  \end{center}
\end{figure*}

\begin{figure*}[htbp]
  \begin{center}
    \leavevmode
%    \epsfig{file=TBrower_Canvas.eps,width=12cm}
    \caption{This shows a typical TBrowser window.}
    \label{ITSMan:ch0:TBrowser}
  \end{center}
\end{figure*}
