\section{Ali\-HLTComponent Class Reference}
\label{classAliHLTComponent}\index{AliHLTComponent@{AliHLTComponent}}
{\tt \#include $<$Ali\-HLTComponent.h$>$}

Inheritance diagram for Ali\-HLTComponent::\begin{figure}[H]
\begin{center}
\leavevmode
\includegraphics[height=1.35135cm]{classAliHLTComponent}
\end{center}
\end{figure}


\subsection{Detailed Description}
Base class of HLT data processing components. The class provides a common interface for HLT data processing components. The interface can be accessed from the online HLT framework or the Ali\-Root offline analysis framework. \subsection{Component identification and properties}\label{classAliHLTComponent_alihltcomponent-properties}
Each component must provide a unique ID, input and output data type indications, and a spawn function. \subsubsection{Required property methods}\label{classAliHLTComponent_alihltcomponent-req-methods}
\begin{itemize}
\item {\bf Get\-Component\-ID}{\rm (p.\,\pageref{classAliHLTComponent_a9})}\item {\bf Get\-Input\-Data\-Types}{\rm (p.\,\pageref{classAliHLTComponent_a10})} (see {\bf Component type}{\rm (p.\,\pageref{classAliHLTComponent_alihltcomponent-type})} for default implementations.)\item {\bf Get\-Output\-Data\-Type}{\rm (p.\,\pageref{classAliHLTComponent_a11})} (see {\bf Component type}{\rm (p.\,\pageref{classAliHLTComponent_alihltcomponent-type})} for default implementations.)\item {\bf Get\-Output\-Data\-Size}{\rm (p.\,\pageref{classAliHLTComponent_a12})} (see {\bf Component type}{\rm (p.\,\pageref{classAliHLTComponent_alihltcomponent-type})} for default implementations.)\item {\bf Spawn}{\rm (p.\,\pageref{classAliHLTComponent_a13})}\end{itemize}
\subsubsection{Optional handlers}\label{classAliHLTComponent_alihltcomponent-opt-mehods}
\begin{itemize}
\item {\bf Do\-Init}{\rm (p.\,\pageref{classAliHLTComponent_b5})}\item {\bf Do\-Deinit}{\rm (p.\,\pageref{classAliHLTComponent_b6})}\end{itemize}
\subsubsection{Data processing}\label{classAliHLTComponent_alihltcomponent-processing-mehods}
\subsubsection{Component type}\label{classAliHLTComponent_alihltcomponent-type}
Components can be of type\begin{itemize}
\item {\bf k\-Source}{\rm (p.\,\pageref{classAliHLTComponent_w9w1})} components which only produce data\item {\bf k\-Processor}{\rm (p.\,\pageref{classAliHLTComponent_w9w2})} components which consume and produce data\item {\bf k\-Sink}{\rm (p.\,\pageref{classAliHLTComponent_w9w3})} components which only consume data\end{itemize}


where data production and consumption refer to the analysis data stream. In order to indicate the type, a child component can overload the {\bf Get\-Component\-Type}{\rm (p.\,\pageref{classAliHLTComponent_a8})} function. \paragraph{Standard implementations}\label{classAliHLTComponent_alihltcomponent-type-std}
Components in general do not need to implement this function, standard implementations of the 3 types are available:\begin{itemize}
\item {\bf Ali\-HLTData\-Source}{\rm (p.\,\pageref{classAliHLTDataSource})} for components of type {\bf k\-Source}{\rm (p.\,\pageref{classAliHLTComponent_w9w1})} \par
 All types of data sources can inherit from {\bf Ali\-HLTData\-Source}{\rm (p.\,\pageref{classAliHLTDataSource})} and must implement the {\bf Ali\-HLTData\-Source::Get\-Event}{\rm (p.\,\pageref{classAliHLTDataSource_d0})} method. The class also implements a standard method for {\bf Get\-Input\-Data\-Types}{\rm (p.\,\pageref{classAliHLTComponent_a10})}.\end{itemize}


\begin{itemize}
\item {\bf Ali\-HLTProcessor}{\rm (p.\,\pageref{classAliHLTProcessor})} for components of type {\bf k\-Processor}{\rm (p.\,\pageref{classAliHLTComponent_w9w2})} \par
 All types of data processors can inherit from {\bf Ali\-HLTData\-Source}{\rm (p.\,\pageref{classAliHLTDataSource})} and must implement the {\bf Ali\-HLTProcessor::Do\-Event}{\rm (p.\,\pageref{classAliHLTProcessor_d0})} method.\end{itemize}


\begin{itemize}
\item {\bf Ali\-HLTData\-Sink}{\rm (p.\,\pageref{classAliHLTDataSink})} for components of type {\bf k\-Sink}{\rm (p.\,\pageref{classAliHLTComponent_w9w3})} \par
 All types of data processors can inherit from {\bf Ali\-HLTData\-Source}{\rm (p.\,\pageref{classAliHLTDataSource})} and must implement the {\bf Ali\-HLTData\-Sink::Dump\-Event}{\rm (p.\,\pageref{classAliHLTDataSink_d0})} method. The class also implements a standard method for {\bf Get\-Output\-Data\-Type}{\rm (p.\,\pageref{classAliHLTComponent_a11})} and {\bf Get\-Output\-Data\-Size}{\rm (p.\,\pageref{classAliHLTComponent_a12})}.\end{itemize}
\subsubsection{Running environment}\label{classAliHLTComponent_alihltcomponent-environment}
In order to adapt to different environments (on-line/off-line), the component gets an environment structure with function pointers. The base class provides member functions for those environment dependend functions. The member functions are used by the component implementation and are re-mapped to the corresponding functions. \subsection{Component interfaces}\label{classAliHLTComponent_alihltcomponent-interfaces}
Each of the 3 standard component base classes {\bf Ali\-HLTProcessor}{\rm (p.\,\pageref{classAliHLTProcessor})}, {\bf Ali\-HLTData\-Source}{\rm (p.\,\pageref{classAliHLTDataSource})} and {\bf Ali\-HLTData\-Sink}{\rm (p.\,\pageref{classAliHLTDataSink})} provides it's own processing method (see {\bf Standard implementations}{\rm (p.\,\pageref{classAliHLTComponent_alihltcomponent-type-std})}), which splits into a high and a low-level method. For the {\bf Low-level interface}{\rm (p.\,\pageref{classAliHLTComponent_alihltcomponent-low-level-interface})}, all parameters are shipped as function arguments, the component is supposed to dump data to the output buffer and handle all block descriptors. The {\bf High-level interface}{\rm (p.\,\pageref{classAliHLTComponent_alihltcomponent-high-level-interface})} is the standard processing method and will be used whenever the low-level method is not overloaded.\subsubsection{High-level interface}\label{classAliHLTComponent_alihltcomponent-high-level-interface}
The high-level component interface provides functionality to exchange ROOT structures between components. In contrast to the {\bf Low-level interface}{\rm (p.\,\pageref{classAliHLTComponent_alihltcomponent-low-level-interface})}, a couple of functions can be used to access data blocks of the input stream and send data blocks or ROOT TObject's to the output stream. The functionality is hidden from the user and is implemented by using ROOT's TMessage class.\paragraph{Interface methods}\label{classAliHLTComponent_alihltcomponent-high-level-int-methods}
The interface provides a couple of methods in order to get objects from the input, data blocks (non TObject) from the input, and to push back objects and data blocks to the output. For convenience there are several functions of identical name (and similar behavior) with different parameters defined. Please refer to the function documentation.\begin{itemize}
\item {\bf Get\-Number\-Of\-Input\-Blocks}{\rm (p.\,\pageref{classAliHLTComponent_b12})} \par
 return the number of data blocks in the input stream\item {\bf Get\-First\-Input\-Object}{\rm (p.\,\pageref{classAliHLTComponent_b13})} \par
 get the first object of a specific data type\item {\bf Get\-Next\-Input\-Object}{\rm (p.\,\pageref{classAliHLTComponent_b15})} \par
 get the next object of same data type as last Get\-First\-Input\-Object/Block call\item {\bf Get\-First\-Input\-Block}{\rm (p.\,\pageref{classAliHLTComponent_b18})} \par
 get the first block of a specific data type\item {\bf Get\-Next\-Input\-Block}{\rm (p.\,\pageref{classAliHLTComponent_b21})} \par
 get the next block of same data type as last Get\-First\-Input\-Block/Block call\item {\bf Push\-Back}{\rm (p.\,\pageref{classAliHLTComponent_b23})} \par
 insert an object or data buffer into the output\item {\bf Create\-Event\-Done\-Data}{\rm (p.\,\pageref{classAliHLTComponent_b28})} \par
 add event information to the output\end{itemize}


In addition, the processing methods are simplified a bit by cutting out most of the parameters. The component implementation \begin{Desc}
\item[See also:]{\bf Ali\-HLTProcessor}{\rm (p.\,\pageref{classAliHLTProcessor})} {\bf Ali\-HLTData\-Source}{\rm (p.\,\pageref{classAliHLTDataSource})} {\bf Ali\-HLTData\-Sink}{\rm (p.\,\pageref{classAliHLTDataSink})}\end{Desc}
{\em IMPORTANT:\/} objects and block descriptors provided by the high-level interface {\bf MUST NOT BE DELETED} by the caller.\paragraph{High-level interface guidelines}\label{classAliHLTComponent_alihltcomponent-high-level-int-guidelines}
\begin{itemize}
\item Structures must inherit from the ROOT object base class TObject in order be transported by the transportation framework.\item all pointer members must be transient (marked {\tt //!} behind the member definition), i.e. will not be stored/transported, or properly marked ({\tt //-$>$}) in order to call the streamer of the object the member is pointing to. The latter is not recomended. Structures to be transported between components should be streamlined.\item no use of stl vectors/strings, use appropriate ROOT classes instead\end{itemize}
\subsubsection{Low-level interface}\label{classAliHLTComponent_alihltcomponent-low-level-interface}
The low-level component interface consists of the specific data processing methods for {\bf Ali\-HLTProcessor}{\rm (p.\,\pageref{classAliHLTProcessor})}, {\bf Ali\-HLTData\-Source}{\rm (p.\,\pageref{classAliHLTDataSource})}, and {\bf Ali\-HLTData\-Sink}{\rm (p.\,\pageref{classAliHLTDataSink})}.\begin{itemize}
\item {\bf Ali\-HLTProcessor::Do\-Event}{\rm (p.\,\pageref{classAliHLTProcessor_d0})}\item {\bf Ali\-HLTData\-Source::Get\-Event}{\rm (p.\,\pageref{classAliHLTDataSource_d0})}\item {\bf Ali\-HLTData\-Sink::Dump\-Event}{\rm (p.\,\pageref{classAliHLTDataSink_d0})}\end{itemize}
\subsection{Component handling}\label{classAliHLTComponent_alihltcomponent-handling}
The handling of HLT analysis components is carried out by the {\bf Ali\-HLTComponent\-Handler}{\rm (p.\,\pageref{classAliHLTComponentHandler})}. Component are registered automatically at load-time of the component shared library under the following suppositions:\begin{itemize}
\item the component library has to be loaded from the {\bf Ali\-HLTComponent\-Handler}{\rm (p.\,\pageref{classAliHLTComponentHandler})} using the {\bf Ali\-HLTComponent\-Handler::Load\-Library}{\rm (p.\,\pageref{classAliHLTComponentHandler_a6})} method.\item the component implementation defines one global object (which is generated when the library is loaded)\end{itemize}
\subsubsection{General design considerations}\label{classAliHLTComponent_alihltcomponent-design-rules}
The analysis code should be implemented in one or more destict class(es). A {\em component\/} should be implemented which interface the destict analysis code to the component interface. This component generates the analysis object dynamically. \par


Assume you have an implemetation {\tt  Ali\-HLTDet\-My\-Analysis }, another class {\tt  Ali\-HLTDet\-My\-Analysis\-Component } contains: \small\begin{alltt}
 private:
   AliHLTDetMyAnalysis* fMyAnalysis;  //!
 \end{alltt}\normalsize 
 The object should then be instantiated in the Do\-Init handler of {\tt Ali\-HLTDet\-My\-Analysis\-Component }, and cleaned in the Do\-Deinit handler.

Further rules:\begin{itemize}
\item avoid big static arrays in the component, allocate the memory at runtime\end{itemize}
\subsection{Class members}\label{classAliHLTComponent_alihltcomponent-members}




Definition at line 197 of file Ali\-HLTComponent.h.\subsection*{Public Types}
\begin{CompactItemize}
\item 
enum {\bf TComponent\-Type} \{ {\bf k\-Unknown} = 0, 
{\bf k\-Source} = 1, 
{\bf k\-Processor} = 2, 
{\bf k\-Sink} = 3
 \}
\item 
enum {\bf Ali\-HLTStopwatch\-Type} \{ \par
{\bf k\-SWBase}, 
{\bf k\-SWDA}, 
{\bf k\-SWInput}, 
{\bf k\-SWOutput}, 
\par
{\bf k\-SWType\-Count}
 \}
\end{CompactItemize}
\subsection*{Public Member Functions}
\begin{CompactItemize}
\item 
{\bf Ali\-HLTComponent} ()
\item 
{\bf Ali\-HLTComponent} (const {\bf Ali\-HLTComponent} \&)
\item 
{\bf Ali\-HLTComponent} \& {\bf operator=} (const {\bf Ali\-HLTComponent} \&)
\item 
virtual {\bf $\sim$Ali\-HLTComponent} ()
\item 
virtual int {\bf Init} ({\bf Ali\-HLTComponent\-Environment} $\ast$environ, void $\ast$environ\-Param, int argc, const char $\ast$$\ast$argv)
\item 
virtual int {\bf Deinit} ()
\item 
int {\bf Process\-Event} (const {\bf Ali\-HLTComponent\-Event\-Data} \&evt\-Data, const {\bf Ali\-HLTComponent\-Block\-Data} $\ast$blocks, {\bf Ali\-HLTComponent\-Trigger\-Data} \&trig\-Data, {\bf Ali\-HLTUInt8\_\-t} $\ast$output\-Ptr, {\bf Ali\-HLTUInt32\_\-t} \&size, {\bf Ali\-HLTUInt32\_\-t} \&output\-Block\-Cnt, {\bf Ali\-HLTComponent\-Block\-Data} $\ast$\&output\-Blocks, {\bf Ali\-HLTComponent\-Event\-Done\-Data} $\ast$\&edd)
\item 
virtual int {\bf Do\-Processing} (const {\bf Ali\-HLTComponent\-Event\-Data} \&evt\-Data, const {\bf Ali\-HLTComponent\-Block\-Data} $\ast$blocks, {\bf Ali\-HLTComponent\-Trigger\-Data} \&trig\-Data, {\bf Ali\-HLTUInt8\_\-t} $\ast$output\-Ptr, {\bf Ali\-HLTUInt32\_\-t} \&size, vector$<$ {\bf Ali\-HLTComponent\-Block\-Data} $>$ \&output\-Blocks, {\bf Ali\-HLTComponent\-Event\-Done\-Data} $\ast$\&edd)=0
\item 
virtual {\bf TComponent\-Type} {\bf Get\-Component\-Type} ()=0
\item 
virtual const char $\ast$ {\bf Get\-Component\-ID} ()=0
\item 
virtual void {\bf Get\-Input\-Data\-Types} (vector$<$ {\bf Ali\-HLTComponent\-Data\-Type} $>$ \&)=0
\item 
virtual {\bf Ali\-HLTComponent\-Data\-Type} {\bf Get\-Output\-Data\-Type} ()=0
\item 
virtual void {\bf Get\-Output\-Data\-Size} (unsigned long \&const\-Base, double \&input\-Multiplier)=0
\item 
virtual {\bf Ali\-HLTComponent} $\ast$ {\bf Spawn} ()=0
\item 
int {\bf Find\-Matching\-Data\-Types} ({\bf Ali\-HLTComponent} $\ast$p\-Consumer, vector$<$ {\bf Ali\-HLTComponent\-Data\-Type} $>$ $\ast$tgt\-List)
\item 
void {\bf Print\-Data\-Type\-Content} ({\bf Ali\-HLTComponent\-Data\-Type} \&dt, const char $\ast$format=NULL) const 
\item 
void {\bf Print\-Component\-Data\-Type\-Info} (const {\bf Ali\-HLTComponent\-Data\-Type} \&dt)
\item 
int {\bf Set\-Stopwatch} (TObject $\ast$p\-SW, {\bf Ali\-HLTStopwatch\-Type} type=k\-SWBase)
\item 
int {\bf Set\-Stopwatches} (TObj\-Array $\ast$p\-Stopwatches)
\end{CompactItemize}
\subsection*{Static Public Member Functions}
\begin{CompactItemize}
\item 
int {\bf Set\-Global\-Component\-Handler} ({\bf Ali\-HLTComponent\-Handler} $\ast$p\-CH, int b\-Overwrite=0)
\item 
int {\bf Unset\-Global\-Component\-Handler} ()
\item 
string {\bf Data\-Type2Text} (const {\bf Ali\-HLTComponent\-Data\-Type} \&type)
\item 
void {\bf Fill\-Event\-Data} ({\bf Ali\-HLTComponent\-Event\-Data} \&evt\-Data)
\end{CompactItemize}
\subsection*{Protected Member Functions}
\begin{CompactItemize}
\item 
void {\bf Fill\-Block\-Data} ({\bf Ali\-HLTComponent\-Block\-Data} \&block\-Data) const 
\item 
void {\bf Fill\-Shm\-Data} ({\bf Ali\-HLTComponent\-Shm\-Data} \&shm\-Data) const 
\item 
void {\bf Fill\-Data\-Type} ({\bf Ali\-HLTComponent\-Data\-Type} \&data\-Type) const 
\item 
void {\bf Copy\-Data\-Type} ({\bf Ali\-HLTComponent\-Data\-Type} \&tgtdt, const {\bf Ali\-HLTComponent\-Data\-Type} \&srcdt)
\item 
void {\bf Set\-Data\-Type} ({\bf Ali\-HLTComponent\-Data\-Type} \&tgtdt, const char $\ast$id, const char $\ast$origin)
\item 
virtual int {\bf Do\-Init} (int argc, const char $\ast$$\ast$argv)
\item 
virtual int {\bf Do\-Deinit} ()
\item 
void $\ast$ {\bf Alloc\-Memory} (unsigned long size)
\item 
int {\bf Make\-Output\-Data\-Block\-List} (const vector$<$ {\bf Ali\-HLTComponent\-Block\-Data} $>$ \&blocks, {\bf Ali\-HLTUInt32\_\-t} $\ast$block\-Count, {\bf Ali\-HLTComponent\-Block\-Data} $\ast$$\ast$output\-Blocks)
\item 
int {\bf Get\-Event\-Done\-Data} (unsigned long size, {\bf Ali\-HLTComponent\-Event\-Done\-Data} $\ast$$\ast$edd)
\item 
void {\bf Data\-Type2Text} (const {\bf Ali\-HLTComponent\-Data\-Type} \&type, char output[{\bf k\-Ali\-HLTComponent\-Data\-Typef\-IDsize}+{\bf k\-Ali\-HLTComponent\-Data\-Typef\-Origin\-Size}+2]) const 
\item 
int {\bf Get\-Event\-Count} () const 
\item 
int {\bf Get\-Number\-Of\-Input\-Blocks} () const 
\item 
const TObject $\ast$ {\bf Get\-First\-Input\-Object} (const {\bf Ali\-HLTComponent\-Data\-Type} \&dt={\bf k\-Ali\-HLTAny\-Data\-Type}, const char $\ast$classname=NULL, int b\-Force=0)
\item 
const TObject $\ast$ {\bf Get\-First\-Input\-Object} (const char $\ast$dt\-ID, const char $\ast$dt\-Origin, const char $\ast$classname=NULL, int b\-Force=0)
\item 
const TObject $\ast$ {\bf Get\-Next\-Input\-Object} (int b\-Force=0)
\item 
{\bf Ali\-HLTComponent\-Data\-Type} {\bf Get\-Data\-Type} (const TObject $\ast$p\-Object=NULL)
\item 
{\bf Ali\-HLTUInt32\_\-t} {\bf Get\-Specification} (const TObject $\ast$p\-Object=NULL)
\item 
const {\bf Ali\-HLTComponent\-Block\-Data} $\ast$ {\bf Get\-First\-Input\-Block} (const {\bf Ali\-HLTComponent\-Data\-Type} \&dt={\bf k\-Ali\-HLTAny\-Data\-Type})
\item 
const {\bf Ali\-HLTComponent\-Block\-Data} $\ast$ {\bf Get\-First\-Input\-Block} (const char $\ast$dt\-ID, const char $\ast$dt\-Origin)
\item 
const {\bf Ali\-HLTComponent\-Block\-Data} $\ast$ {\bf Get\-Input\-Block} (int index)
\item 
const {\bf Ali\-HLTComponent\-Block\-Data} $\ast$ {\bf Get\-Next\-Input\-Block} ()
\item 
{\bf Ali\-HLTUInt32\_\-t} {\bf Get\-Specification} (const {\bf Ali\-HLTComponent\-Block\-Data} $\ast$p\-Block=NULL)
\item 
int {\bf Push\-Back} (TObject $\ast$p\-Object, const {\bf Ali\-HLTComponent\-Data\-Type} \&dt, {\bf Ali\-HLTUInt32\_\-t} spec={\bf k\-Ali\-HLTVoid\-Data\-Spec})
\item 
int {\bf Push\-Back} (TObject $\ast$p\-Object, const char $\ast$dt\-ID, const char $\ast$dt\-Origin, {\bf Ali\-HLTUInt32\_\-t} spec={\bf k\-Ali\-HLTVoid\-Data\-Spec})
\item 
int {\bf Push\-Back} (void $\ast$p\-Buffer, int i\-Size, const {\bf Ali\-HLTComponent\-Data\-Type} \&dt, {\bf Ali\-HLTUInt32\_\-t} spec={\bf k\-Ali\-HLTVoid\-Data\-Spec})
\item 
int {\bf Push\-Back} (void $\ast$p\-Buffer, int i\-Size, const char $\ast$dt\-ID, const char $\ast$dt\-Origin, {\bf Ali\-HLTUInt32\_\-t} spec={\bf k\-Ali\-HLTVoid\-Data\-Spec})
\item 
int {\bf Estimate\-Object\-Size} (TObject $\ast$p\-Object) const 
\item 
int {\bf Create\-Event\-Done\-Data} ({\bf Ali\-HLTComponent\-Event\-Done\-Data} edd)
\end{CompactItemize}
\subsection*{Private Member Functions}
\begin{CompactItemize}
\item 
int {\bf Increment\-Event\-Counter} ()
\item 
int {\bf Find\-Input\-Block} (const {\bf Ali\-HLTComponent\-Data\-Type} \&dt, int start\-Idx=-1) const 
\item 
int {\bf Find\-Input\-Block} (const {\bf Ali\-HLTComponent\-Block\-Data} $\ast$p\-Block) const 
\item 
TObject $\ast$ {\bf Create\-Input\-Object} (int idx, int b\-Force=0)
\item 
TObject $\ast$ {\bf Get\-Input\-Object} (int idx, const char $\ast$classname=NULL, int b\-Force=0)
\item 
int {\bf Cleanup\-Input\-Objects} ()
\item 
int {\bf Insert\-Output\-Block} (void $\ast$p\-Buffer, int i\-Size, const {\bf Ali\-HLTComponent\-Data\-Type} \&dt, {\bf Ali\-HLTUInt32\_\-t} spec)
\end{CompactItemize}
\subsection*{Private Attributes}
\begin{CompactItemize}
\item 
{\bf Ali\-HLTComponent\-Environment} {\bf f\-Environment}
\begin{CompactList}\small\item\em transient \item\end{CompactList}\item 
{\bf Ali\-HLTEvent\-ID\_\-t} {\bf f\-Current\-Event}
\item 
int {\bf f\-Event\-Count}
\item 
int {\bf f\-Failed\-Events}
\item 
{\bf Ali\-HLTComponent\-Event\-Data} {\bf f\-Current\-Event\-Data}
\item 
const {\bf Ali\-HLTComponent\-Block\-Data} $\ast$ {\bf fp\-Input\-Blocks}
\item 
int {\bf f\-Current\-Input\-Block}
\begin{CompactList}\small\item\em transient \item\end{CompactList}\item 
{\bf Ali\-HLTComponent\-Data\-Type} {\bf f\-Search\-Data\-Type}
\item 
string {\bf f\-Class\-Name}
\item 
TObj\-Array $\ast$ {\bf fp\-Input\-Objects}
\item 
{\bf Ali\-HLTUInt8\_\-t} $\ast$ {\bf fp\-Output\-Buffer}
\begin{CompactList}\small\item\em transient \item\end{CompactList}\item 
{\bf Ali\-HLTUInt32\_\-t} {\bf f\-Output\-Buffer\-Size}
\begin{CompactList}\small\item\em transient \item\end{CompactList}\item 
{\bf Ali\-HLTUInt32\_\-t} {\bf f\-Output\-Buffer\-Filled}
\item 
vector$<$ {\bf Ali\-HLTComponent\-Block\-Data} $>$ {\bf f\-Output\-Blocks}
\item 
TObj\-Array $\ast$ {\bf fp\-Stopwatches}
\end{CompactItemize}
\subsection*{Static Private Attributes}
\begin{CompactItemize}
\item 
{\bf Ali\-HLTComponent\-Handler} $\ast$ {\bf fgp\-Component\-Handler} = NULL
\end{CompactItemize}


\subsection{Member Enumeration Documentation}
\index{AliHLTComponent@{Ali\-HLTComponent}!AliHLTStopwatchType@{AliHLTStopwatchType}}
\index{AliHLTStopwatchType@{AliHLTStopwatchType}!AliHLTComponent@{Ali\-HLTComponent}}
\subsubsection{\setlength{\rightskip}{0pt plus 5cm}enum {\bf Ali\-HLTComponent::Ali\-HLTStopwatch\-Type}}\label{classAliHLTComponent_w10}


Stopwatch type for benchmarking. \begin{Desc}
\item[Enumeration values: ]\par
\begin{description}
\index{kSWBase@{kSWBase}!AliHLTComponent@{AliHLTComponent}}\index{AliHLTComponent@{AliHLTComponent}!kSWBase@{kSWBase}}\item[{\em 
k\-SWBase\label{classAliHLTComponent_w10w4}
}]total time for event processing \index{kSWDA@{kSWDA}!AliHLTComponent@{AliHLTComponent}}\index{AliHLTComponent@{AliHLTComponent}!kSWDA@{kSWDA}}\item[{\em 
k\-SWDA\label{classAliHLTComponent_w10w5}
}]detector algorithm w/o interface callbacks \index{kSWInput@{kSWInput}!AliHLTComponent@{AliHLTComponent}}\index{AliHLTComponent@{AliHLTComponent}!kSWInput@{kSWInput}}\item[{\em 
k\-SWInput\label{classAliHLTComponent_w10w6}
}]data sources \index{kSWOutput@{kSWOutput}!AliHLTComponent@{AliHLTComponent}}\index{AliHLTComponent@{AliHLTComponent}!kSWOutput@{kSWOutput}}\item[{\em 
k\-SWOutput\label{classAliHLTComponent_w10w7}
}]data sinks \index{kSWTypeCount@{kSWTypeCount}!AliHLTComponent@{AliHLTComponent}}\index{AliHLTComponent@{AliHLTComponent}!kSWTypeCount@{kSWTypeCount}}\item[{\em 
k\-SWType\-Count\label{classAliHLTComponent_w10w8}
}]number of types \end{description}
\end{Desc}



Definition at line 374 of file Ali\-HLTComponent.h.

\footnotesize\begin{verbatim}374                            {
376     kSWBase,
378     kSWDA,
380     kSWInput,
382     kSWOutput,
384     kSWTypeCount
385   };
\end{verbatim}\normalsize 


\index{AliHLTComponent@{Ali\-HLTComponent}!TComponentType@{TComponentType}}
\index{TComponentType@{TComponentType}!AliHLTComponent@{Ali\-HLTComponent}}
\subsubsection{\setlength{\rightskip}{0pt plus 5cm}enum {\bf Ali\-HLTComponent::TComponent\-Type}}\label{classAliHLTComponent_w9}


component type definitions \begin{Desc}
\item[Enumeration values: ]\par
\begin{description}
\index{kUnknown@{kUnknown}!AliHLTComponent@{AliHLTComponent}}\index{AliHLTComponent@{AliHLTComponent}!kUnknown@{kUnknown}}\item[{\em 
k\-Unknown\label{classAliHLTComponent_w9w0}
}]\index{kSource@{kSource}!AliHLTComponent@{AliHLTComponent}}\index{AliHLTComponent@{AliHLTComponent}!kSource@{kSource}}\item[{\em 
k\-Source\label{classAliHLTComponent_w9w1}
}]\index{kProcessor@{kProcessor}!AliHLTComponent@{AliHLTComponent}}\index{AliHLTComponent@{AliHLTComponent}!kProcessor@{kProcessor}}\item[{\em 
k\-Processor\label{classAliHLTComponent_w9w2}
}]\index{kSink@{kSink}!AliHLTComponent@{AliHLTComponent}}\index{AliHLTComponent@{AliHLTComponent}!kSink@{kSink}}\item[{\em 
k\-Sink\label{classAliHLTComponent_w9w3}
}]\end{description}
\end{Desc}



Definition at line 209 of file Ali\-HLTComponent.h.

\footnotesize\begin{verbatim}209 { kUnknown=0, kSource=1, kProcessor=2, kSink=3 };
\end{verbatim}\normalsize 




\subsection{Constructor \& Destructor Documentation}
\index{AliHLTComponent@{Ali\-HLTComponent}!AliHLTComponent@{AliHLTComponent}}
\index{AliHLTComponent@{AliHLTComponent}!AliHLTComponent@{Ali\-HLTComponent}}
\subsubsection{\setlength{\rightskip}{0pt plus 5cm}Ali\-HLTComponent::Ali\-HLTComponent ()}\label{classAliHLTComponent_a0}


standard constructor 

Definition at line 49 of file Ali\-HLTComponent.cxx.

References f\-Environment, fgp\-Component\-Handler, k\-Ali\-HLTVoid\-Data\-Type, k\-HLTLog\-Default, Ali\-HLTComponent\-Handler::Schedule\-Register(), and Ali\-HLTLogging::Set\-Local\-Logging\-Level().

\footnotesize\begin{verbatim}50   :
51   fEnvironment(),
52   fCurrentEvent(0),
53   fEventCount(-1),
54   fFailedEvents(0),
55   fCurrentEventData(),
56   fpInputBlocks(NULL),
57   fCurrentInputBlock(-1),
58   fSearchDataType(kAliHLTVoidDataType),
59   fClassName(),
60   fpInputObjects(NULL),
61   fpOutputBuffer(NULL),
62   fOutputBufferSize(0),
63   fOutputBufferFilled(0),
64   fOutputBlocks(),
65   fpStopwatches(new TObjArray(kSWTypeCount))
66 {
67   // see header file for class documentation
68   // or
69   // refer to README to build package
70   // or
71   // visit http://web.ift.uib.no/~kjeks/doc/alice-hlt
72   memset(&fEnvironment, 0, sizeof(AliHLTComponentEnvironment));
73   if (fgpComponentHandler)
74     fgpComponentHandler->ScheduleRegister(this);
75   SetLocalLoggingLevel(kHLTLogDefault);
76 }

\end{verbatim}\normalsize 


\index{AliHLTComponent@{Ali\-HLTComponent}!AliHLTComponent@{AliHLTComponent}}
\index{AliHLTComponent@{AliHLTComponent}!AliHLTComponent@{Ali\-HLTComponent}}
\subsubsection{\setlength{\rightskip}{0pt plus 5cm}Ali\-HLTComponent::Ali\-HLTComponent (const {\bf Ali\-HLTComponent} \&)}\label{classAliHLTComponent_a1}


not a valid copy constructor, defined according to effective C++ style 

Definition at line 78 of file Ali\-HLTComponent.cxx.

References HLTFatal, and k\-Ali\-HLTVoid\-Data\-Type.

\footnotesize\begin{verbatim}79   :
80   AliHLTLogging(),
81   fEnvironment(),
82   fCurrentEvent(0),
83   fEventCount(-1),
84   fFailedEvents(0),
85   fCurrentEventData(),
86   fpInputBlocks(NULL),
87   fCurrentInputBlock(-1),
88   fSearchDataType(kAliHLTVoidDataType),
89   fClassName(),
90   fpInputObjects(NULL),
91   fpOutputBuffer(NULL),
92   fOutputBufferSize(0),
93   fOutputBufferFilled(0),
94   fOutputBlocks(),
95   fpStopwatches(NULL)
96 {
97   // see header file for class documentation
98   HLTFatal("copy constructor untested");
99 }

\end{verbatim}\normalsize 


\index{AliHLTComponent@{Ali\-HLTComponent}!~AliHLTComponent@{$\sim$AliHLTComponent}}
\index{~AliHLTComponent@{$\sim$AliHLTComponent}!AliHLTComponent@{Ali\-HLTComponent}}
\subsubsection{\setlength{\rightskip}{0pt plus 5cm}Ali\-HLTComponent::$\sim${\bf Ali\-HLTComponent} ()\hspace{0.3cm}{\tt  [virtual]}}\label{classAliHLTComponent_a3}


standard destructor 

Definition at line 108 of file Ali\-HLTComponent.cxx.

References Cleanup\-Input\-Objects(), and fp\-Stopwatches.

\footnotesize\begin{verbatim}109 {
110   // see header file for function documentation
111   CleanupInputObjects();
112   if (fpStopwatches!=NULL) delete fpStopwatches;
113   fpStopwatches=NULL;
114 }
\end{verbatim}\normalsize 




\subsection{Member Function Documentation}
\index{AliHLTComponent@{Ali\-HLTComponent}!AllocMemory@{AllocMemory}}
\index{AllocMemory@{AllocMemory}!AliHLTComponent@{Ali\-HLTComponent}}
\subsubsection{\setlength{\rightskip}{0pt plus 5cm}void $\ast$ Ali\-HLTComponent::Alloc\-Memory (unsigned long {\em size})\hspace{0.3cm}{\tt  [protected]}}\label{classAliHLTComponent_b7}


General memory allocation method. All memory which is going to be used 'outside' of the interface must be provided by the framework (online or offline). The method is redirected to a function provided by the current framework. Function pointers are transferred via the {\bf Ali\-HLTComponent\-Environment}{\rm (p.\,\pageref{structAliHLTComponentEnvironment})} structure. 

Definition at line 249 of file Ali\-HLTComponent.cxx.

References Ali\-HLTComponent\-Environment::f\-Alloc\-Memory\-Func, f\-Environment, Ali\-HLTComponent\-Environment::f\-Param, and HLTFatal.

Referenced by Make\-Output\-Data\-Block\-List().

\footnotesize\begin{verbatim}250 {
251   // see header file for function documentation
252   if (fEnvironment.fAllocMemoryFunc)
253     return (*fEnvironment.fAllocMemoryFunc)(fEnvironment.fParam, size );
254   HLTFatal("no memory allocation handler registered");
255   return NULL;
256 }
\end{verbatim}\normalsize 


\index{AliHLTComponent@{Ali\-HLTComponent}!CleanupInputObjects@{CleanupInputObjects}}
\index{CleanupInputObjects@{CleanupInputObjects}!AliHLTComponent@{Ali\-HLTComponent}}
\subsubsection{\setlength{\rightskip}{0pt plus 5cm}int Ali\-HLTComponent::Cleanup\-Input\-Objects ()\hspace{0.3cm}{\tt  [private]}}\label{classAliHLTComponent_d5}


Clean the list of input objects. Cleanup is done at the end of each event processing. 

Definition at line 564 of file Ali\-HLTComponent.cxx.

References fp\-Input\-Objects.

Referenced by Process\-Event(), and $\sim$Ali\-HLTComponent().

\footnotesize\begin{verbatim}565 {
566   // see header file for function documentation
567   if (!fpInputObjects) return 0;
568   TObjArray* array=fpInputObjects;
569   fpInputObjects=NULL;
570   for (int i=0; i<array->GetEntries(); i++) {
571     TObject* pObj=array->At(i);
572     if (pObj) delete pObj;
573   }
574   delete array;
575   return 0;
576 }
\end{verbatim}\normalsize 


\index{AliHLTComponent@{Ali\-HLTComponent}!CopyDataType@{CopyDataType}}
\index{CopyDataType@{CopyDataType}!AliHLTComponent@{Ali\-HLTComponent}}
\subsubsection{\setlength{\rightskip}{0pt plus 5cm}void Ali\-HLTComponent::Copy\-Data\-Type ({\bf Ali\-HLTComponent\-Data\-Type} \& {\em tgtdt}, const {\bf Ali\-HLTComponent\-Data\-Type} \& {\em srcdt})\hspace{0.3cm}{\tt  [protected]}}\label{classAliHLTComponent_b3}


Copy data type structure Copies the value an {\bf Ali\-HLTComponent\-Data\-Type}{\rm (p.\,\pageref{structAliHLTComponentDataType})} structure to another one \begin{Desc}
\item[Parameters:]
\begin{description}
\item[\mbox{$\rightarrow$} {\em tgtdt}]target structure \item[\mbox{$\leftarrow$} {\em srcdt}]source structure \end{description}
\end{Desc}


Definition at line 374 of file Ali\-HLTComponent.cxx.

References Ali\-HLTComponent\-Data\-Type::f\-ID, Ali\-HLTComponent\-Data\-Type::f\-Origin, k\-Ali\-HLTComponent\-Data\-Typef\-IDsize, and k\-Ali\-HLTComponent\-Data\-Typef\-Origin\-Size.

\footnotesize\begin{verbatim}375 {
376   // see header file for function documentation
377   memcpy(&tgtdt.fID[0], &srcdt.fID[0], kAliHLTComponentDataTypefIDsize);
378   memcpy(&tgtdt.fOrigin[0], &srcdt.fOrigin[0], kAliHLTComponentDataTypefOriginSize);
379 }
\end{verbatim}\normalsize 


\index{AliHLTComponent@{Ali\-HLTComponent}!CreateEventDoneData@{CreateEventDoneData}}
\index{CreateEventDoneData@{CreateEventDoneData}!AliHLTComponent@{Ali\-HLTComponent}}
\subsubsection{\setlength{\rightskip}{0pt plus 5cm}int Ali\-HLTComponent::Create\-Event\-Done\-Data ({\bf Ali\-HLTComponent\-Event\-Done\-Data} {\em edd})\hspace{0.3cm}{\tt  [protected]}}\label{classAliHLTComponent_b28}


Insert event-done data information into the output. \begin{Desc}
\item[Parameters:]
\begin{description}
\item[{\em edd}]event-done data information \end{description}
\end{Desc}


Definition at line 790 of file Ali\-HLTComponent.cxx.

References HLTWarning.

\footnotesize\begin{verbatim}791 {
792   // see header file for function documentation
793   int iResult=-ENOSYS;
794   //#warning  function not yet implemented
795   HLTWarning("function not yet implemented");
796   return iResult;
797 }
\end{verbatim}\normalsize 


\index{AliHLTComponent@{Ali\-HLTComponent}!CreateInputObject@{CreateInputObject}}
\index{CreateInputObject@{CreateInputObject}!AliHLTComponent@{Ali\-HLTComponent}}
\subsubsection{\setlength{\rightskip}{0pt plus 5cm}TObject $\ast$ Ali\-HLTComponent::Create\-Input\-Object (int {\em idx}, int {\em b\-Force} = {\tt 0})\hspace{0.3cm}{\tt  [private]}}\label{classAliHLTComponent_d3}


Create an object from a specified input block. \begin{Desc}
\item[Parameters:]
\begin{description}
\item[{\em idx}]index of the input block \item[{\em b\-Force}]force the retrieval of an object, error messages are suppressed if {\em b\-Force\/} is not set \end{description}
\end{Desc}
\begin{Desc}
\item[Returns:]pointer to TObject, caller must delete the object after use \end{Desc}


Definition at line 509 of file Ali\-HLTComponent.cxx.

References Ali\-HLTUInt32\_\-t, Ali\-HLTComponent\-Event\-Data::f\-Block\-Cnt, f\-Current\-Event\-Data, fp\-Input\-Blocks, Ali\-HLTComponent\-Block\-Data::f\-Ptr, Ali\-HLTComponent\-Block\-Data::f\-Size, Ali\-HLTMessage::Get\-Class(), HLTDebug, HLTError, and HLTFatal.

Referenced by Get\-Input\-Object().

\footnotesize\begin{verbatim}510 {
511   // see header file for function documentation
512   TObject* pObj=NULL;
513   if (fpInputBlocks!=NULL) {
514     if ((UInt_t)idx<fCurrentEventData.fBlockCnt) {
515       if (fpInputBlocks[idx].fPtr) {
516         AliHLTUInt32_t firstWord=*((AliHLTUInt32_t*)fpInputBlocks[idx].fPtr);
517         if (firstWord==fpInputBlocks[idx].fSize-sizeof(AliHLTUInt32_t)) {
518           HLTDebug("create object from block %d size %d", idx, fpInputBlocks[idx].fSize);
519           AliHLTMessage msg(fpInputBlocks[idx].fPtr, fpInputBlocks[idx].fSize);
520           TClass* objclass=msg.GetClass();
521           pObj=msg.ReadObject(objclass);
522           if (pObj && objclass) {
523             HLTDebug("object %p type %s created", pObj, objclass->GetName());
524           } else {
525           }
526         } else {
527           //    } else if (bForce!=0) {
528           HLTError("size missmatch: block size %d, indicated %d", fpInputBlocks[idx].fSize, firstWord+sizeof(AliHLTUInt32_t));
529         }
530       } else {
531         HLTFatal("block descriptor empty");
532       }
533     } else {
534       HLTError("index %d out of range %d", idx, fCurrentEventData.fBlockCnt);
535     }
536   } else {
537     HLTError("no input blocks available");
538   }
539   
540   return pObj;
541 }
\end{verbatim}\normalsize 


\index{AliHLTComponent@{Ali\-HLTComponent}!DataType2Text@{DataType2Text}}
\index{DataType2Text@{DataType2Text}!AliHLTComponent@{Ali\-HLTComponent}}
\subsubsection{\setlength{\rightskip}{0pt plus 5cm}void Ali\-HLTComponent::Data\-Type2Text (const {\bf Ali\-HLTComponent\-Data\-Type} \& {\em type}, char {\em output}[k\-Ali\-HLTComponent\-Data\-Typef\-IDsize+k\-Ali\-HLTComponent\-Data\-Typef\-Origin\-Size+2]) const\hspace{0.3cm}{\tt  [protected]}}\label{classAliHLTComponent_b10}


Helper function to convert the data type to a string. 

Definition at line 216 of file Ali\-HLTComponent.cxx.

References Ali\-HLTComponent\-Data\-Type::f\-ID, Ali\-HLTComponent\-Data\-Type::f\-Origin, k\-Ali\-HLTComponent\-Data\-Typef\-IDsize, and k\-Ali\-HLTComponent\-Data\-Typef\-Origin\-Size.

\footnotesize\begin{verbatim}217 {
218   // see header file for function documentation
219   memset( output, 0, kAliHLTComponentDataTypefIDsize+kAliHLTComponentDataTypefOriginSize+2 );
220   strncat( output, type.fOrigin, kAliHLTComponentDataTypefOriginSize );
221   strcat( output, ":" );
222   strncat( output, type.fID, kAliHLTComponentDataTypefIDsize );
223 }
\end{verbatim}\normalsize 


\index{AliHLTComponent@{Ali\-HLTComponent}!DataType2Text@{DataType2Text}}
\index{DataType2Text@{DataType2Text}!AliHLTComponent@{Ali\-HLTComponent}}
\subsubsection{\setlength{\rightskip}{0pt plus 5cm}string Ali\-HLTComponent::Data\-Type2Text (const {\bf Ali\-HLTComponent\-Data\-Type} \& {\em type})\hspace{0.3cm}{\tt  [static]}}\label{classAliHLTComponent_e2}


Helper function to convert the data type to a string. 

Definition at line 225 of file Ali\-HLTComponent.cxx.

References Ali\-HLTComponent\-Data\-Type::f\-ID, Ali\-HLTComponent\-Data\-Type::f\-Origin, k\-Ali\-HLTComponent\-Data\-Typef\-IDsize, and k\-Ali\-HLTComponent\-Data\-Typef\-Origin\-Size.

Referenced by Get\-First\-Input\-Object(), and Print\-Data\-Type\-Content().

\footnotesize\begin{verbatim}226 {
227   // see header file for function documentation
228   string out("");
229   
230   if (type==kAliHLTVoidDataType) {
231     out="VOID:VOID";
232   } else {
233     // some gymnastics in order to avoid a '0' which is part of either or both
234     // ID and origin terminating the whole string. Unfortunately, string doesn't
235     // stop appending at the '0' if the number of elements to append was 
236     // explicitely specified
237     string tmp("");
238     tmp.append(type.fOrigin, kAliHLTComponentDataTypefOriginSize);
239     out.append(tmp.c_str());
240     out.append(":");
241     tmp="";
242     tmp.append(type.fID, kAliHLTComponentDataTypefIDsize);
243     out.append(tmp.c_str());
244   }
245   return out;
246 }
\end{verbatim}\normalsize 


\index{AliHLTComponent@{Ali\-HLTComponent}!Deinit@{Deinit}}
\index{Deinit@{Deinit}!AliHLTComponent@{Ali\-HLTComponent}}
\subsubsection{\setlength{\rightskip}{0pt plus 5cm}int Ali\-HLTComponent::Deinit ()\hspace{0.3cm}{\tt  [virtual]}}\label{classAliHLTComponent_a5}


Clean-up function to terminate data processing. Clean-up of common data structures after data processing. The call is redirected to the internal method {\bf Do\-Deinit}{\rm (p.\,\pageref{classAliHLTComponent_b6})} which can be overridden by the child class. 

Reimplemented in {\bf Ali\-HLTProcessor} {\rm (p.\,\pageref{classAliHLTProcessor_a3})}, {\bf Ali\-HLTPHOSDDLDecoder\-Component} {\rm (p.\,\pageref{classAliHLTPHOSDDLDecoderComponent_a5})}, {\bf Ali\-HLTPHOSHistogram\-Producer\-Component} {\rm (p.\,\pageref{classAliHLTPHOSHistogramProducerComponent_a5})}, {\bf Ali\-HLTPHOSModule\-Merger\-Component} {\rm (p.\,\pageref{classAliHLTPHOSModuleMergerComponent_a5})}, {\bf Ali\-HLTPHOSRaw\-Analyzer\-Component} {\rm (p.\,\pageref{classAliHLTPHOSRawAnalyzerComponent_a5})}, and {\bf Ali\-HLTPHOSRcu\-Histogram\-Producer\-Component} {\rm (p.\,\pageref{classAliHLTPHOSRcuHistogramProducerComponent_a3})}.

Definition at line 191 of file Ali\-HLTComponent.cxx.

References Do\-Deinit().

Referenced by Ali\-HLT\_\-C\_\-Destroy\-Component(), Ali\-HLTProcessor::Deinit(), and Ali\-HLTTask::Deinit().

\footnotesize\begin{verbatim}192 {
193   // see header file for function documentation
194   int iResult=0;
195   iResult=DoDeinit();
196   return iResult;
197 }
\end{verbatim}\normalsize 


\index{AliHLTComponent@{Ali\-HLTComponent}!DoDeinit@{DoDeinit}}
\index{DoDeinit@{DoDeinit}!AliHLTComponent@{Ali\-HLTComponent}}
\subsubsection{\setlength{\rightskip}{0pt plus 5cm}int Ali\-HLTComponent::Do\-Deinit ()\hspace{0.3cm}{\tt  [protected, virtual]}}\label{classAliHLTComponent_b6}


Default method for the internal clean-up. The method is called by {\bf Deinit}{\rm (p.\,\pageref{classAliHLTComponent_a5})} 

Reimplemented in {\bf Ali\-HLTPHOSDDLDecoder\-Component} {\rm (p.\,\pageref{classAliHLTPHOSDDLDecoderComponent_a6})}, {\bf Ali\-HLTPHOSHistogram\-Producer\-Component} {\rm (p.\,\pageref{classAliHLTPHOSHistogramProducerComponent_a6})}, {\bf Ali\-HLTPHOSModule\-Merger\-Component} {\rm (p.\,\pageref{classAliHLTPHOSModuleMergerComponent_a6})}, {\bf Ali\-HLTPHOSRaw\-Analyzer\-Component} {\rm (p.\,\pageref{classAliHLTPHOSRawAnalyzerComponent_a6})}, and {\bf Ali\-HLTPHOSRcu\-Histogram\-Producer\-Component} {\rm (p.\,\pageref{classAliHLTPHOSRcuHistogramProducerComponent_a4})}.

Definition at line 209 of file Ali\-HLTComponent.cxx.

References f\-Event\-Count.

Referenced by Deinit().

\footnotesize\begin{verbatim}210 {
211   // see header file for function documentation
212   fEventCount=0;
213   return 0;
214 }
\end{verbatim}\normalsize 


\index{AliHLTComponent@{Ali\-HLTComponent}!DoInit@{DoInit}}
\index{DoInit@{DoInit}!AliHLTComponent@{Ali\-HLTComponent}}
\subsubsection{\setlength{\rightskip}{0pt plus 5cm}int Ali\-HLTComponent::Do\-Init (int {\em argc}, const char $\ast$$\ast$ {\em argv})\hspace{0.3cm}{\tt  [protected, virtual]}}\label{classAliHLTComponent_b5}


Default method for the internal initialization. The method is called by {\bf Init}{\rm (p.\,\pageref{classAliHLTComponent_a4})} 

Reimplemented in {\bf Ali\-HLTPHOSDDLDecoder\-Component} {\rm (p.\,\pageref{classAliHLTPHOSDDLDecoderComponent_a4})}, {\bf Ali\-HLTPHOSHistogram\-Producer\-Component} {\rm (p.\,\pageref{classAliHLTPHOSHistogramProducerComponent_a4})}, {\bf Ali\-HLTPHOSModule\-Merger\-Component} {\rm (p.\,\pageref{classAliHLTPHOSModuleMergerComponent_a4})}, {\bf Ali\-HLTPHOSRaw\-Analyzer\-Component} {\rm (p.\,\pageref{classAliHLTPHOSRawAnalyzerComponent_a4})}, and {\bf Ali\-HLTPHOSRcu\-Histogram\-Producer\-Component} {\rm (p.\,\pageref{classAliHLTPHOSRcuHistogramProducerComponent_a2})}.

Definition at line 199 of file Ali\-HLTComponent.cxx.

References f\-Event\-Count.

Referenced by Init().

\footnotesize\begin{verbatim}200 {
201   // see header file for function documentation
202   if (argc==0 && argv==NULL) {
203     // this is currently just to get rid of the warning "unused parameter"
204   }
205   fEventCount=0;
206   return 0;
207 }
\end{verbatim}\normalsize 


\index{AliHLTComponent@{Ali\-HLTComponent}!DoProcessing@{DoProcessing}}
\index{DoProcessing@{DoProcessing}!AliHLTComponent@{Ali\-HLTComponent}}
\subsubsection{\setlength{\rightskip}{0pt plus 5cm}virtual int Ali\-HLTComponent::Do\-Processing (const {\bf Ali\-HLTComponent\-Event\-Data} \& {\em evt\-Data}, const {\bf Ali\-HLTComponent\-Block\-Data} $\ast$ {\em blocks}, {\bf Ali\-HLTComponent\-Trigger\-Data} \& {\em trig\-Data}, {\bf Ali\-HLTUInt8\_\-t} $\ast$ {\em output\-Ptr}, {\bf Ali\-HLTUInt32\_\-t} \& {\em size}, vector$<$ {\bf Ali\-HLTComponent\-Block\-Data} $>$ \& {\em output\-Blocks}, {\bf Ali\-HLTComponent\-Event\-Done\-Data} $\ast$\& {\em edd})\hspace{0.3cm}{\tt  [pure virtual]}}\label{classAliHLTComponent_a7}


Internal processing of one event. The method is pure virtual and implemented by the child classes\begin{itemize}
\item {\bf Ali\-HLTProcessor}{\rm (p.\,\pageref{classAliHLTProcessor})}\item {\bf Ali\-HLTData\-Source}{\rm (p.\,\pageref{classAliHLTDataSource})}\item {\bf Ali\-HLTData\-Sink}{\rm (p.\,\pageref{classAliHLTDataSink})}\end{itemize}


\begin{Desc}
\item[Parameters:]
\begin{description}
\item[{\em evt\-Data}]\item[{\em blocks}]\item[{\em trig\-Data}]\item[{\em output\-Ptr}]\item[{\em size}]\item[{\em output\-Blocks}]out: the output block array is allocated internally \item[{\em edd}]\end{description}
\end{Desc}
\begin{Desc}
\item[Returns:]neg. error code if failed \end{Desc}


Implemented in {\bf Ali\-HLTData\-Sink} {\rm (p.\,\pageref{classAliHLTDataSink_a2})}, {\bf Ali\-HLTData\-Source} {\rm (p.\,\pageref{classAliHLTDataSource_a2})}, and {\bf Ali\-HLTProcessor} {\rm (p.\,\pageref{classAliHLTProcessor_a4})}.

Referenced by Process\-Event().\index{AliHLTComponent@{Ali\-HLTComponent}!EstimateObjectSize@{EstimateObjectSize}}
\index{EstimateObjectSize@{EstimateObjectSize}!AliHLTComponent@{Ali\-HLTComponent}}
\subsubsection{\setlength{\rightskip}{0pt plus 5cm}int Ali\-HLTComponent::Estimate\-Object\-Size (TObject $\ast$ {\em p\-Object}) const\hspace{0.3cm}{\tt  [protected]}}\label{classAliHLTComponent_b27}


Estimate size of a TObject \begin{Desc}
\item[Parameters:]
\begin{description}
\item[{\em p\-Object}]\end{description}
\end{Desc}
\begin{Desc}
\item[Returns:]buffer size in byte \end{Desc}


Definition at line 781 of file Ali\-HLTComponent.cxx.

\footnotesize\begin{verbatim}782 {
783   // see header file for function documentation
784   if (!pObject) return -EINVAL;
785     AliHLTMessage msg(kMESS_OBJECT);
786     msg.WriteObject(pObject);
787     return msg.Length();  
788 }
\end{verbatim}\normalsize 


\index{AliHLTComponent@{Ali\-HLTComponent}!FillBlockData@{FillBlockData}}
\index{FillBlockData@{FillBlockData}!AliHLTComponent@{Ali\-HLTComponent}}
\subsubsection{\setlength{\rightskip}{0pt plus 5cm}void Ali\-HLTComponent::Fill\-Block\-Data ({\bf Ali\-HLTComponent\-Block\-Data} \& {\em block\-Data}) const\hspace{0.3cm}{\tt  [protected]}}\label{classAliHLTComponent_b0}


Fill {\bf Ali\-HLTComponent\-Block\-Data}{\rm (p.\,\pageref{structAliHLTComponentBlockData})} structure with default values. \begin{Desc}
\item[Parameters:]
\begin{description}
\item[{\em block\-Data}]reference to data structure \end{description}
\end{Desc}


Definition at line 348 of file Ali\-HLTComponent.cxx.

References Ali\-HLTComponent\-Block\-Data::f\-Data\-Type, Fill\-Data\-Type(), Fill\-Shm\-Data(), Ali\-HLTComponent\-Block\-Data::f\-Offset, Ali\-HLTComponent\-Block\-Data::f\-Ptr, Ali\-HLTComponent\-Block\-Data::f\-Shm\-Key, Ali\-HLTComponent\-Block\-Data::f\-Size, Ali\-HLTComponent\-Block\-Data::f\-Specification, and Ali\-HLTComponent\-Block\-Data::f\-Struct\-Size.

Referenced by Ali\-HLTPHOSRcu\-Histogram\-Producer\-Component::Do\-Event(), Ali\-HLTPHOSRaw\-Analyzer\-Component::Do\-Event(), Ali\-HLTPHOSHistogram\-Producer\-Component::Do\-Event(), Ali\-HLTPHOSDDLDecoder\-Component::Do\-Event(), and Insert\-Output\-Block().

\footnotesize\begin{verbatim}349 {
350   // see header file for function documentation
351   blockData.fStructSize = sizeof(blockData);
352   FillShmData( blockData.fShmKey );
353   blockData.fOffset = ~(AliHLTUInt32_t)0;
354   blockData.fPtr = NULL;
355   blockData.fSize = 0;
356   FillDataType( blockData.fDataType );
357   blockData.fSpecification = kAliHLTVoidDataSpec;
358 }
\end{verbatim}\normalsize 


\index{AliHLTComponent@{Ali\-HLTComponent}!FillDataType@{FillDataType}}
\index{FillDataType@{FillDataType}!AliHLTComponent@{Ali\-HLTComponent}}
\subsubsection{\setlength{\rightskip}{0pt plus 5cm}void Ali\-HLTComponent::Fill\-Data\-Type ({\bf Ali\-HLTComponent\-Data\-Type} \& {\em data\-Type}) const\hspace{0.3cm}{\tt  [protected]}}\label{classAliHLTComponent_b2}


Fill {\bf Ali\-HLTComponent\-Data\-Type}{\rm (p.\,\pageref{structAliHLTComponentDataType})} structure with default values. \begin{Desc}
\item[Parameters:]
\begin{description}
\item[{\em data\-Type}]reference to data structure \end{description}
\end{Desc}


Definition at line 368 of file Ali\-HLTComponent.cxx.

Referenced by Fill\-Block\-Data().

\footnotesize\begin{verbatim}369 {
370   // see header file for function documentation
371   dataType=kAliHLTAnyDataType;
372 }
\end{verbatim}\normalsize 


\index{AliHLTComponent@{Ali\-HLTComponent}!FillEventData@{FillEventData}}
\index{FillEventData@{FillEventData}!AliHLTComponent@{Ali\-HLTComponent}}
\subsubsection{\setlength{\rightskip}{0pt plus 5cm}void Ali\-HLTComponent::Fill\-Event\-Data ({\bf Ali\-HLTComponent\-Event\-Data} \& {\em evt\-Data})\hspace{0.3cm}{\tt  [static]}}\label{classAliHLTComponent_e3}


helper function to initialize {\bf Ali\-HLTComponent\-Event\-Data}{\rm (p.\,\pageref{structAliHLTComponentEventData})} structure 

Definition at line 399 of file Ali\-HLTComponent.cxx.

References Ali\-HLTComponent\-Event\-Data::f\-Struct\-Size.

Referenced by Ali\-HLTTask::Process\-Task().

\footnotesize\begin{verbatim}400 {
401   // see header file for function documentation
402   memset(&evtData, 0, sizeof(AliHLTComponentEventData));
403   evtData.fStructSize=sizeof(AliHLTComponentEventData);
404 }
\end{verbatim}\normalsize 


\index{AliHLTComponent@{Ali\-HLTComponent}!FillShmData@{FillShmData}}
\index{FillShmData@{FillShmData}!AliHLTComponent@{Ali\-HLTComponent}}
\subsubsection{\setlength{\rightskip}{0pt plus 5cm}void Ali\-HLTComponent::Fill\-Shm\-Data ({\bf Ali\-HLTComponent\-Shm\-Data} \& {\em shm\-Data}) const\hspace{0.3cm}{\tt  [protected]}}\label{classAliHLTComponent_b1}


Fill {\bf Ali\-HLTComponent\-Shm\-Data}{\rm (p.\,\pageref{structAliHLTComponentShmData})} structure with default values. \begin{Desc}
\item[Parameters:]
\begin{description}
\item[{\em shm\-Data}]reference to data structure \end{description}
\end{Desc}


Definition at line 360 of file Ali\-HLTComponent.cxx.

References Ali\-HLTComponent\-Shm\-Data::f\-Shm\-ID, Ali\-HLTComponent\-Shm\-Data::f\-Shm\-Type, and Ali\-HLTComponent\-Shm\-Data::f\-Struct\-Size.

Referenced by Fill\-Block\-Data().

\footnotesize\begin{verbatim}361 {
362   // see header file for function documentation
363   shmData.fStructSize = sizeof(shmData);
364   shmData.fShmType = gkAliHLTComponentInvalidShmType;
365   shmData.fShmID = gkAliHLTComponentInvalidShmID;
366 }
\end{verbatim}\normalsize 


\index{AliHLTComponent@{Ali\-HLTComponent}!FindInputBlock@{FindInputBlock}}
\index{FindInputBlock@{FindInputBlock}!AliHLTComponent@{Ali\-HLTComponent}}
\subsubsection{\setlength{\rightskip}{0pt plus 5cm}int Ali\-HLTComponent::Find\-Input\-Block (const {\bf Ali\-HLTComponent\-Block\-Data} $\ast$ {\em p\-Block}) const\hspace{0.3cm}{\tt  [private]}}\label{classAliHLTComponent_d2}


Get index in the array of input bocks. Calculate index and check integrety of a block data structure pointer. \begin{Desc}
\item[Parameters:]
\begin{description}
\item[{\em p\-Block}]pointer to block data \end{description}
\end{Desc}
\begin{Desc}
\item[Returns:]index of the block, -ENOENT if no block found \end{Desc}


Definition at line 660 of file Ali\-HLTComponent.cxx.

References Ali\-HLTComponent\-Event\-Data::f\-Block\-Cnt, f\-Current\-Event\-Data, and fp\-Input\-Blocks.

\footnotesize\begin{verbatim}661 {
662   // see header file for function documentation
663   int iResult=-ENOENT;
664   if (fpInputBlocks!=NULL) {
665     if (pBlock) {
666       if (pBlock>=fpInputBlocks && pBlock<fpInputBlocks+fCurrentEventData.fBlockCnt) {
667         iResult=(int)(pBlock-fpInputBlocks);
668       }
669     } else {
670       iResult=-EINVAL;
671     }
672   }
673   return iResult;
674 }
\end{verbatim}\normalsize 


\index{AliHLTComponent@{Ali\-HLTComponent}!FindInputBlock@{FindInputBlock}}
\index{FindInputBlock@{FindInputBlock}!AliHLTComponent@{Ali\-HLTComponent}}
\subsubsection{\setlength{\rightskip}{0pt plus 5cm}int Ali\-HLTComponent::Find\-Input\-Block (const {\bf Ali\-HLTComponent\-Data\-Type} \& {\em dt}, int {\em start\-Idx} = {\tt -1}) const\hspace{0.3cm}{\tt  [private]}}\label{classAliHLTComponent_d1}


Find the first input block of specified data type beginning at index. \begin{Desc}
\item[Parameters:]
\begin{description}
\item[{\em dt}]data type \item[{\em start\-Idx}]index to start the search \end{description}
\end{Desc}
\begin{Desc}
\item[Returns:]index of the block, -ENOENT if no block found \end{Desc}


Definition at line 494 of file Ali\-HLTComponent.cxx.

References Ali\-HLTComponent\-Event\-Data::f\-Block\-Cnt, f\-Current\-Event\-Data, Ali\-HLTComponent\-Block\-Data::f\-Data\-Type, fp\-Input\-Blocks, and k\-Ali\-HLTAny\-Data\-Type.

Referenced by Get\-First\-Input\-Block(), Get\-First\-Input\-Object(), Get\-Next\-Input\-Block(), Get\-Next\-Input\-Object(), and Get\-Specification().

\footnotesize\begin{verbatim}495 {
496   // see header file for function documentation
497   int iResult=-ENOENT;
498   if (fpInputBlocks!=NULL) {
499     int idx=startIdx<0?0:startIdx;
500     for ( ; (UInt_t)idx<fCurrentEventData.fBlockCnt && iResult==-ENOENT; idx++) {
501       if (dt == kAliHLTAnyDataType || fpInputBlocks[idx].fDataType == dt) {
502         iResult=idx;
503       }
504     }
505   }
506   return iResult;
507 }
\end{verbatim}\normalsize 


\index{AliHLTComponent@{Ali\-HLTComponent}!FindMatchingDataTypes@{FindMatchingDataTypes}}
\index{FindMatchingDataTypes@{FindMatchingDataTypes}!AliHLTComponent@{Ali\-HLTComponent}}
\subsubsection{\setlength{\rightskip}{0pt plus 5cm}int Ali\-HLTComponent::Find\-Matching\-Data\-Types ({\bf Ali\-HLTComponent} $\ast$ {\em p\-Consumer}, vector$<$ {\bf Ali\-HLTComponent\-Data\-Type} $>$ $\ast$ {\em tgt\-List})}\label{classAliHLTComponent_a14}


Find matching data types between this component and a consumer component. Currently, a component can produce only one type of data. This restriction is most likely to be abolished in the future. \begin{Desc}
\item[Parameters:]
\begin{description}
\item[{\em p\-Consumer}]a component and consumer of the data produced by this component \item[{\em tgt\-List}]reference to a vector list to receive the matching data types. \end{description}
\end{Desc}
\begin{Desc}
\item[Returns:]$>$= 0 success, neg. error code if failed \end{Desc}


Definition at line 300 of file Ali\-HLTComponent.cxx.

References Get\-Output\-Data\-Type().

Referenced by Ali\-HLTTask::Get\-Nof\-Matching\-Data\-Types().

\footnotesize\begin{verbatim}301 {
302   // see header file for function documentation
303   int iResult=0;
304   if (pConsumer) {
305     vector<AliHLTComponentDataType> ctlist;
306     ((AliHLTComponent*)pConsumer)->GetInputDataTypes(ctlist);
307     vector<AliHLTComponentDataType>::iterator type=ctlist.begin();
308     //AliHLTComponentDataType ouptdt=GetOutputDataType();
309     //PrintDataTypeContent(ouptdt, "publisher \'%s\'");
310     while (type!=ctlist.end() && iResult==0) {
311       //PrintDataTypeContent((*type), "consumer \'%s\'");
312       if ((*type)==GetOutputDataType() ||
313           (*type)==kAliHLTAnyDataType) {
314         if (tgtList) tgtList->push_back(*type);
315         iResult++;
316         // this loop has to be changed in case of multiple output types
317         break;
318       }
319       type++;
320     }
321   } else {
322     iResult=-EINVAL;
323   }
324   return iResult;
325 }
\end{verbatim}\normalsize 


\index{AliHLTComponent@{Ali\-HLTComponent}!GetComponentID@{GetComponentID}}
\index{GetComponentID@{GetComponentID}!AliHLTComponent@{Ali\-HLTComponent}}
\subsubsection{\setlength{\rightskip}{0pt plus 5cm}virtual const char$\ast$ Ali\-HLTComponent::Get\-Component\-ID ()\hspace{0.3cm}{\tt  [pure virtual]}}\label{classAliHLTComponent_a9}


Get the id of the component. Each component is identified by a unique id. The function is pure virtual and must be implemented by the child class. \begin{Desc}
\item[Returns:]component id (string) \end{Desc}


Implemented in {\bf Ali\-HLTPHOSDDLDecoder\-Component} {\rm (p.\,\pageref{classAliHLTPHOSDDLDecoderComponent_a10})}, {\bf Ali\-HLTPHOSHistogram\-Producer\-Component} {\rm (p.\,\pageref{classAliHLTPHOSHistogramProducerComponent_a10})}, {\bf Ali\-HLTPHOSModule\-Merger\-Component} {\rm (p.\,\pageref{classAliHLTPHOSModuleMergerComponent_a10})}, {\bf Ali\-HLTPHOSRaw\-Analyzer\-Component} {\rm (p.\,\pageref{classAliHLTPHOSRawAnalyzerComponent_a12})}, {\bf Ali\-HLTPHOSRaw\-Analyzer\-Crude\-Component} {\rm (p.\,\pageref{classAliHLTPHOSRawAnalyzerCrudeComponent_a4})}, {\bf Ali\-HLTPHOSRaw\-Analyzer\-Peak\-Finder\-Component} {\rm (p.\,\pageref{classAliHLTPHOSRawAnalyzerPeakFinderComponent_a2})}, and {\bf Ali\-HLTPHOSRcu\-Histogram\-Producer\-Component} {\rm (p.\,\pageref{classAliHLTPHOSRcuHistogramProducerComponent_a10})}.

Referenced by Ali\-HLTData\-Source::Do\-Processing(), Ali\-HLTTask::Print\-Status(), Process\-Event(), Ali\-HLTTask::Process\-Task(), Ali\-HLTComponent\-Handler::Register\-Component(), and Ali\-HLTData\-Buffer::Set\-Consumer().\index{AliHLTComponent@{Ali\-HLTComponent}!GetComponentType@{GetComponentType}}
\index{GetComponentType@{GetComponentType}!AliHLTComponent@{Ali\-HLTComponent}}
\subsubsection{\setlength{\rightskip}{0pt plus 5cm}virtual {\bf TComponent\-Type} Ali\-HLTComponent::Get\-Component\-Type ()\hspace{0.3cm}{\tt  [pure virtual]}}\label{classAliHLTComponent_a8}


Get the type of the component. The function is pure virtual and must be implemented by the child class. \begin{Desc}
\item[Returns:]component type id \end{Desc}


Implemented in {\bf Ali\-HLTData\-Sink} {\rm (p.\,\pageref{classAliHLTDataSink_a3})}, {\bf Ali\-HLTData\-Source} {\rm (p.\,\pageref{classAliHLTDataSource_a3})}, and {\bf Ali\-HLTProcessor} {\rm (p.\,\pageref{classAliHLTProcessor_a5})}.

Referenced by Ali\-HLTSystem::Init\-Benchmarking(), and Ali\-HLTTask::Process\-Task().\index{AliHLTComponent@{Ali\-HLTComponent}!GetDataType@{GetDataType}}
\index{GetDataType@{GetDataType}!AliHLTComponent@{Ali\-HLTComponent}}
\subsubsection{\setlength{\rightskip}{0pt plus 5cm}{\bf Ali\-HLTComponent\-Data\-Type} Ali\-HLTComponent::Get\-Data\-Type (const TObject $\ast$ {\em p\-Object} = {\tt NULL})\hspace{0.3cm}{\tt  [protected]}}\label{classAliHLTComponent_b16}


Get data type of an input block. Get data type of the object previously fetched via Get\-First\-Input\-Object/Next\-Input\-Object or the last one if no object specified. \begin{Desc}
\item[Parameters:]
\begin{description}
\item[{\em p\-Object}]pointer to TObject \end{description}
\end{Desc}
\begin{Desc}
\item[Returns:]data specification, k\-Ali\-HLTVoid\-Data\-Spec if failed \end{Desc}


Definition at line 578 of file Ali\-HLTComponent.cxx.

References ALIHLTCOMPONENT\_\-BASE\_\-STOPWATCH, Ali\-HLTComponent\-Event\-Data::f\-Block\-Cnt, f\-Current\-Event\-Data, Ali\-HLTComponent\-Block\-Data::f\-Data\-Type, fp\-Input\-Blocks, fp\-Input\-Objects, HLTError, and HLTFatal.

\footnotesize\begin{verbatim}579 {
580   // see header file for function documentation
581   ALIHLTCOMPONENT_BASE_STOPWATCH();
582   AliHLTComponentDataType dt=kAliHLTVoidDataType;
583   int idx=fCurrentInputBlock;
584   if (pObject) {
585     if (fpInputObjects==NULL || (idx=fpInputObjects->IndexOf(pObject))>=0) {
586     } else {
587       HLTError("unknown object %p", pObject);
588     }
589   }
590   if (idx>=0) {
591     if ((UInt_t)idx<fCurrentEventData.fBlockCnt) {
592       dt=fpInputBlocks[idx].fDataType;
593     } else {
594       HLTFatal("severe internal error, index out of range");
595     }
596   }
597   return dt;
598 }
\end{verbatim}\normalsize 


\index{AliHLTComponent@{Ali\-HLTComponent}!GetEventCount@{GetEventCount}}
\index{GetEventCount@{GetEventCount}!AliHLTComponent@{Ali\-HLTComponent}}
\subsubsection{\setlength{\rightskip}{0pt plus 5cm}int Ali\-HLTComponent::Get\-Event\-Count () const\hspace{0.3cm}{\tt  [protected]}}\label{classAliHLTComponent_b11}


Get event number. \begin{Desc}
\item[Returns:]value of the internal event counter \end{Desc}


Definition at line 424 of file Ali\-HLTComponent.cxx.

\footnotesize\begin{verbatim}425 {
426   // see header file for function documentation
427   return fEventCount;
428 }
\end{verbatim}\normalsize 


\index{AliHLTComponent@{Ali\-HLTComponent}!GetEventDoneData@{GetEventDoneData}}
\index{GetEventDoneData@{GetEventDoneData}!AliHLTComponent@{Ali\-HLTComponent}}
\subsubsection{\setlength{\rightskip}{0pt plus 5cm}int Ali\-HLTComponent::Get\-Event\-Done\-Data (unsigned long {\em size}, {\bf Ali\-HLTComponent\-Event\-Done\-Data} $\ast$$\ast$ {\em edd})\hspace{0.3cm}{\tt  [protected]}}\label{classAliHLTComponent_b9}


Fill the Event\-Done\-Data structure. The method is redirected to a function provided by the current framework. Function pointers are transferred via the {\bf Ali\-HLTComponent\-Environment}{\rm (p.\,\pageref{structAliHLTComponentEnvironment})} structure. 

Definition at line 292 of file Ali\-HLTComponent.cxx.

References f\-Environment, Ali\-HLTComponent\-Environment::f\-Get\-Event\-Done\-Data\-Func, and Ali\-HLTComponent\-Environment::f\-Param.

\footnotesize\begin{verbatim}293 {
294   // see header file for function documentation
295   if (fEnvironment.fGetEventDoneDataFunc)
296     return (*fEnvironment.fGetEventDoneDataFunc)(fEnvironment.fParam, fCurrentEvent, size, edd );
297   return -ENOSYS;
298 }
\end{verbatim}\normalsize 


\index{AliHLTComponent@{Ali\-HLTComponent}!GetFirstInputBlock@{GetFirstInputBlock}}
\index{GetFirstInputBlock@{GetFirstInputBlock}!AliHLTComponent@{Ali\-HLTComponent}}
\subsubsection{\setlength{\rightskip}{0pt plus 5cm}const {\bf Ali\-HLTComponent\-Block\-Data} $\ast$ Ali\-HLTComponent::Get\-First\-Input\-Block (const char $\ast$ {\em dt\-ID}, const char $\ast$ {\em dt\-Origin})\hspace{0.3cm}{\tt  [protected]}}\label{classAliHLTComponent_b19}


Get the first block of a specific data type from the input data. The method looks for the first block of type specified by the ID and Origin strings in the input stream. It is intended to be used within the high-level interface.\par
 {\em Note:\/} THE BLOCK DESCRIPTOR MUST NOT BE DELETED by the caller. \begin{Desc}
\item[Parameters:]
\begin{description}
\item[{\em dt\-ID}]data type ID of the block \item[{\em dt\-Origin}]data type origin of the block \end{description}
\end{Desc}
\begin{Desc}
\item[Returns:]pointer to {\bf Ali\-HLTComponent\-Block\-Data}{\rm (p.\,\pageref{structAliHLTComponentBlockData})} \end{Desc}


Definition at line 637 of file Ali\-HLTComponent.cxx.

References ALIHLTCOMPONENT\_\-BASE\_\-STOPWATCH, Get\-First\-Input\-Block(), and Set\-Data\-Type().

\footnotesize\begin{verbatim}639 {
640   // see header file for function documentation
641   ALIHLTCOMPONENT_BASE_STOPWATCH();
642   AliHLTComponentDataType dt;
643   SetDataType(dt, dtID, dtOrigin);
644   return GetFirstInputBlock(dt);
645 }
\end{verbatim}\normalsize 


\index{AliHLTComponent@{Ali\-HLTComponent}!GetFirstInputBlock@{GetFirstInputBlock}}
\index{GetFirstInputBlock@{GetFirstInputBlock}!AliHLTComponent@{Ali\-HLTComponent}}
\subsubsection{\setlength{\rightskip}{0pt plus 5cm}const {\bf Ali\-HLTComponent\-Block\-Data} $\ast$ Ali\-HLTComponent::Get\-First\-Input\-Block (const {\bf Ali\-HLTComponent\-Data\-Type} \& {\em dt} = {\tt {\bf k\-Ali\-HLTAny\-Data\-Type}})\hspace{0.3cm}{\tt  [protected]}}\label{classAliHLTComponent_b18}


Get the first block of a specific data type from the input data. The method looks for the first block of type dt in the input stream. It is intended to be used within the high-level interface.\par
 {\em Note:\/} THE BLOCK DESCRIPTOR MUST NOT BE DELETED by the caller. \begin{Desc}
\item[Parameters:]
\begin{description}
\item[{\em dt}]data type of the block \end{description}
\end{Desc}
\begin{Desc}
\item[Returns:]pointer to {\bf Ali\-HLTComponent\-Block\-Data}{\rm (p.\,\pageref{structAliHLTComponentBlockData})} \end{Desc}


Definition at line 622 of file Ali\-HLTComponent.cxx.

References ALIHLTCOMPONENT\_\-BASE\_\-STOPWATCH, f\-Class\-Name, Find\-Input\-Block(), fp\-Input\-Blocks, and f\-Search\-Data\-Type.

Referenced by Get\-First\-Input\-Block().

\footnotesize\begin{verbatim}623 {
624   // see header file for function documentation
625   ALIHLTCOMPONENT_BASE_STOPWATCH();
626   fSearchDataType=dt;
627   fClassName.clear();
628   int idx=FindInputBlock(fSearchDataType, 0);
629   const AliHLTComponentBlockData* pBlock=NULL;
630   if (idx>=0) {
631     // check for fpInputBlocks pointer done in FindInputBlock
632     pBlock=&fpInputBlocks[idx];
633   }
634   return pBlock;
635 }
\end{verbatim}\normalsize 


\index{AliHLTComponent@{Ali\-HLTComponent}!GetFirstInputObject@{GetFirstInputObject}}
\index{GetFirstInputObject@{GetFirstInputObject}!AliHLTComponent@{Ali\-HLTComponent}}
\subsubsection{\setlength{\rightskip}{0pt plus 5cm}const TObject $\ast$ Ali\-HLTComponent::Get\-First\-Input\-Object (const char $\ast$ {\em dt\-ID}, const char $\ast$ {\em dt\-Origin}, const char $\ast$ {\em classname} = {\tt NULL}, int {\em b\-Force} = {\tt 0})\hspace{0.3cm}{\tt  [protected]}}\label{classAliHLTComponent_b14}


Get the first object of a specific data type from the input data. The hight-level methods provide functionality to transfer ROOT data structures which inherit from TObject. The method looks for the first ROOT object of type specified by the ID and Origin strings in the input stream. If also the class name is provided, the object is checked for the right class type. The input data block needs a certain structure, namely the buffer size as first word. If the cross check fails, the retrieval is silently abondoned, unless the {\em b\-Force\/} parameter is set.\par
 {\em Note:\/} THE OBJECT MUST NOT BE DELETED by the caller. \begin{Desc}
\item[Parameters:]
\begin{description}
\item[{\em dt\-ID}]data type ID of the object \item[{\em dt\-Origin}]data type origin of the object \item[{\em classname}]class name of the object \item[{\em b\-Force}]force the retrieval of an object, error messages are suppressed if {\em b\-Force\/} is not set \end{description}
\end{Desc}
\begin{Desc}
\item[Returns:]pointer to TObject, NULL if no objects of specified type available \end{Desc}


Definition at line 467 of file Ali\-HLTComponent.cxx.

References ALIHLTCOMPONENT\_\-BASE\_\-STOPWATCH, Get\-First\-Input\-Object(), and Set\-Data\-Type().

\footnotesize\begin{verbatim}471 {
472   // see header file for function documentation
473   ALIHLTCOMPONENT_BASE_STOPWATCH();
474   AliHLTComponentDataType dt;
475   SetDataType(dt, dtID, dtOrigin);
476   return GetFirstInputObject(dt, classname, bForce);
477 }
\end{verbatim}\normalsize 


\index{AliHLTComponent@{Ali\-HLTComponent}!GetFirstInputObject@{GetFirstInputObject}}
\index{GetFirstInputObject@{GetFirstInputObject}!AliHLTComponent@{Ali\-HLTComponent}}
\subsubsection{\setlength{\rightskip}{0pt plus 5cm}const TObject $\ast$ Ali\-HLTComponent::Get\-First\-Input\-Object (const {\bf Ali\-HLTComponent\-Data\-Type} \& {\em dt} = {\tt {\bf k\-Ali\-HLTAny\-Data\-Type}}, const char $\ast$ {\em classname} = {\tt NULL}, int {\em b\-Force} = {\tt 0})\hspace{0.3cm}{\tt  [protected]}}\label{classAliHLTComponent_b13}


Get the first object of a specific data type from the input data. The hight-level methods provide functionality to transfer ROOT data structures which inherit from TObject. The method looks for the first ROOT object of type dt in the input stream. If also the class name is provided, the object is checked for the right class type. The input data block needs a certain structure, namely the buffer size as first word. If the cross check fails, the retrieval is silently abondoned, unless the {\em b\-Force\/} parameter is set.\par
 {\em Note:\/} THE OBJECT MUST NOT BE DELETED by the caller. \begin{Desc}
\item[Parameters:]
\begin{description}
\item[{\em dt}]data type of the object \item[{\em classname}]class name of the object \item[{\em b\-Force}]force the retrieval of an object, error messages are suppressed if {\em b\-Force\/} is not set \end{description}
\end{Desc}
\begin{Desc}
\item[Returns:]pointer to TObject, NULL if no objects of specified type available \end{Desc}


Definition at line 446 of file Ali\-HLTComponent.cxx.

References ALIHLTCOMPONENT\_\-BASE\_\-STOPWATCH, Data\-Type2Text(), f\-Class\-Name, f\-Current\-Input\-Block, Find\-Input\-Block(), f\-Search\-Data\-Type, Get\-Input\-Object(), and HLTDebug.

Referenced by Get\-First\-Input\-Object().

\footnotesize\begin{verbatim}449 {
450   // see header file for function documentation
451   ALIHLTCOMPONENT_BASE_STOPWATCH();
452   fSearchDataType=dt;
453   if (classname) fClassName=classname;
454   else fClassName.clear();
455   int idx=FindInputBlock(fSearchDataType, 0);
456   HLTDebug("found block %d when searching for data type %s", idx, DataType2Text(dt).c_str());
457   TObject* pObj=NULL;
458   if (idx>=0) {
459     if ((pObj=GetInputObject(idx, fClassName.c_str(), bForce))!=NULL) {
460       fCurrentInputBlock=idx;
461     } else {
462     }
463   }
464   return pObj;
465 }
\end{verbatim}\normalsize 


\index{AliHLTComponent@{Ali\-HLTComponent}!GetInputBlock@{GetInputBlock}}
\index{GetInputBlock@{GetInputBlock}!AliHLTComponent@{Ali\-HLTComponent}}
\subsubsection{\setlength{\rightskip}{0pt plus 5cm}const {\bf Ali\-HLTComponent\-Block\-Data}$\ast$ Ali\-HLTComponent::Get\-Input\-Block (int {\em index})\hspace{0.3cm}{\tt  [protected]}}\label{classAliHLTComponent_b20}


Get input block by index.\par
 {\em Note:\/} THE BLOCK DESCRIPTOR MUST NOT BE DELETED by the caller. \begin{Desc}
\item[Returns:]pointer to {\bf Ali\-HLTComponent\-Block\-Data}{\rm (p.\,\pageref{structAliHLTComponentBlockData})}, NULL if index out of range \end{Desc}
\index{AliHLTComponent@{Ali\-HLTComponent}!GetInputDataTypes@{GetInputDataTypes}}
\index{GetInputDataTypes@{GetInputDataTypes}!AliHLTComponent@{Ali\-HLTComponent}}
\subsubsection{\setlength{\rightskip}{0pt plus 5cm}virtual void Ali\-HLTComponent::Get\-Input\-Data\-Types (vector$<$ {\bf Ali\-HLTComponent\-Data\-Type} $>$ \&)\hspace{0.3cm}{\tt  [pure virtual]}}\label{classAliHLTComponent_a10}


Get the input data types of the component. The function is pure virtual and must be implemented by the child class. \begin{Desc}
\item[Returns:]list of data types in the vector reference \end{Desc}


Implemented in {\bf Ali\-HLTData\-Source} {\rm (p.\,\pageref{classAliHLTDataSource_a4})}.\index{AliHLTComponent@{Ali\-HLTComponent}!GetInputObject@{GetInputObject}}
\index{GetInputObject@{GetInputObject}!AliHLTComponent@{Ali\-HLTComponent}}
\subsubsection{\setlength{\rightskip}{0pt plus 5cm}TObject $\ast$ Ali\-HLTComponent::Get\-Input\-Object (int {\em idx}, const char $\ast$ {\em classname} = {\tt NULL}, int {\em b\-Force} = {\tt 0})\hspace{0.3cm}{\tt  [private]}}\label{classAliHLTComponent_d4}


Get input object Get object from the input block list. The methods first checks whether the object was already created. If not, it is created by {\bf Create\-Input\-Object}{\rm (p.\,\pageref{classAliHLTComponent_d3})} and inserted into the list of objects. \begin{Desc}
\item[Parameters:]
\begin{description}
\item[{\em idx}]index in the input block list \item[{\em classname}]name of the class, object is checked for correct class name if set \item[{\em b\-Force}]force the retrieval of an object, error messages are suppressed if {\em b\-Force\/} is not set \end{description}
\end{Desc}
\begin{Desc}
\item[Returns:]pointer to TObject \end{Desc}


Definition at line 543 of file Ali\-HLTComponent.cxx.

References Create\-Input\-Object(), Ali\-HLTComponent\-Event\-Data::f\-Block\-Cnt, f\-Current\-Event\-Data, fp\-Input\-Objects, and HLTFatal.

Referenced by Get\-First\-Input\-Object(), and Get\-Next\-Input\-Object().

\footnotesize\begin{verbatim}544 {
545   // see header file for function documentation
546   if (fpInputObjects==NULL) {
547     fpInputObjects=new TObjArray(fCurrentEventData.fBlockCnt);
548   }
549   TObject* pObj=NULL;
550   if (fpInputObjects) {
551     pObj=fpInputObjects->At(idx);
552     if (pObj==NULL) {
553       pObj=CreateInputObject(idx, bForce);
554       if (pObj) {
555         fpInputObjects->AddAt(pObj, idx);
556       }
557     }
558   } else {
559     HLTFatal("memory allocation failed: TObjArray of size %d", fCurrentEventData.fBlockCnt);
560   }
561   return pObj;
562 }
\end{verbatim}\normalsize 


\index{AliHLTComponent@{Ali\-HLTComponent}!GetNextInputBlock@{GetNextInputBlock}}
\index{GetNextInputBlock@{GetNextInputBlock}!AliHLTComponent@{Ali\-HLTComponent}}
\subsubsection{\setlength{\rightskip}{0pt plus 5cm}const {\bf Ali\-HLTComponent\-Block\-Data} $\ast$ Ali\-HLTComponent::Get\-Next\-Input\-Block ()\hspace{0.3cm}{\tt  [protected]}}\label{classAliHLTComponent_b21}


Get the next block of a specific data type from the input data. The method looks for the next block of type and class specified to the previous {\bf Get\-First\-Input\-Block}{\rm (p.\,\pageref{classAliHLTComponent_b18})} call. To be used within the high-level interface.\par
 {\em Note:\/} THE BLOCK DESCRIPTOR MUST NOT BE DELETED by the caller. 

Definition at line 647 of file Ali\-HLTComponent.cxx.

References ALIHLTCOMPONENT\_\-BASE\_\-STOPWATCH, f\-Current\-Input\-Block, Find\-Input\-Block(), fp\-Input\-Blocks, and f\-Search\-Data\-Type.

\footnotesize\begin{verbatim}648 {
649   // see header file for function documentation
650   ALIHLTCOMPONENT_BASE_STOPWATCH();
651   int idx=FindInputBlock(fSearchDataType, fCurrentInputBlock+1);
652   const AliHLTComponentBlockData* pBlock=NULL;
653   if (idx>=0) {
654     // check for fpInputBlocks pointer done in FindInputBlock
655     pBlock=&fpInputBlocks[idx];
656   }
657   return pBlock;
658 }
\end{verbatim}\normalsize 


\index{AliHLTComponent@{Ali\-HLTComponent}!GetNextInputObject@{GetNextInputObject}}
\index{GetNextInputObject@{GetNextInputObject}!AliHLTComponent@{Ali\-HLTComponent}}
\subsubsection{\setlength{\rightskip}{0pt plus 5cm}const TObject $\ast$ Ali\-HLTComponent::Get\-Next\-Input\-Object (int {\em b\-Force} = {\tt 0})\hspace{0.3cm}{\tt  [protected]}}\label{classAliHLTComponent_b15}


Get the next object of a specific data type from the input data. The hight-level methods provide functionality to transfer ROOT data structures which inherit from TObject. The method looks for the next ROOT object of type and class specified to the previous {\bf Get\-First\-Input\-Object}{\rm (p.\,\pageref{classAliHLTComponent_b13})} call.\par
 {\em Note:\/} THE OBJECT MUST NOT BE DELETED by the caller. \begin{Desc}
\item[Parameters:]
\begin{description}
\item[{\em b\-Force}]force the retrieval of an object, error messages are suppressed if {\em b\-Force\/} is not set \end{description}
\end{Desc}
\begin{Desc}
\item[Returns:]pointer to TObject, NULL if no more objects available \end{Desc}


Definition at line 479 of file Ali\-HLTComponent.cxx.

References ALIHLTCOMPONENT\_\-BASE\_\-STOPWATCH, f\-Class\-Name, f\-Current\-Input\-Block, Find\-Input\-Block(), f\-Search\-Data\-Type, and Get\-Input\-Object().

\footnotesize\begin{verbatim}480 {
481   // see header file for function documentation
482   ALIHLTCOMPONENT_BASE_STOPWATCH();
483   int idx=FindInputBlock(fSearchDataType, fCurrentInputBlock+1);
484   //HLTDebug("found block %d when searching for data type %s", idx, DataType2Text(fSearchDataType).c_str());
485   TObject* pObj=NULL;
486   if (idx>=0) {
487     if ((pObj=GetInputObject(idx, fClassName.c_str(), bForce))!=NULL) {
488       fCurrentInputBlock=idx;
489     }
490   }
491   return pObj;
492 }
\end{verbatim}\normalsize 


\index{AliHLTComponent@{Ali\-HLTComponent}!GetNumberOfInputBlocks@{GetNumberOfInputBlocks}}
\index{GetNumberOfInputBlocks@{GetNumberOfInputBlocks}!AliHLTComponent@{Ali\-HLTComponent}}
\subsubsection{\setlength{\rightskip}{0pt plus 5cm}int Ali\-HLTComponent::Get\-Number\-Of\-Input\-Blocks () const\hspace{0.3cm}{\tt  [protected]}}\label{classAliHLTComponent_b12}


Get the number of input blocks. \begin{Desc}
\item[Returns:]number of input blocks \end{Desc}


Definition at line 437 of file Ali\-HLTComponent.cxx.

References Ali\-HLTComponent\-Event\-Data::f\-Block\-Cnt, f\-Current\-Event\-Data, and fp\-Input\-Blocks.

\footnotesize\begin{verbatim}438 {
439   // see header file for function documentation
440   if (fpInputBlocks!=NULL) {
441     return fCurrentEventData.fBlockCnt;
442   }
443   return 0;
444 }
\end{verbatim}\normalsize 


\index{AliHLTComponent@{Ali\-HLTComponent}!GetOutputDataSize@{GetOutputDataSize}}
\index{GetOutputDataSize@{GetOutputDataSize}!AliHLTComponent@{Ali\-HLTComponent}}
\subsubsection{\setlength{\rightskip}{0pt plus 5cm}virtual void Ali\-HLTComponent::Get\-Output\-Data\-Size (unsigned long \& {\em const\-Base}, double \& {\em input\-Multiplier})\hspace{0.3cm}{\tt  [pure virtual]}}\label{classAliHLTComponent_a12}


Get a ratio by how much the data volume is shrinked or enhanced. The function is pure virtual and must be implemented by the child class. \begin{Desc}
\item[Parameters:]
\begin{description}
\item[{\em const\-Base}]{\em return\/}: additive part, independent of the input data volume \item[{\em input\-Multiplier}]{\em return\/}: multiplication ratio \end{description}
\end{Desc}
\begin{Desc}
\item[Returns:]values in the reference variables \end{Desc}


Implemented in {\bf Ali\-HLTData\-Sink} {\rm (p.\,\pageref{classAliHLTDataSink_a5})}, {\bf Ali\-HLTPHOSDDLDecoder\-Component} {\rm (p.\,\pageref{classAliHLTPHOSDDLDecoderComponent_a13})}, {\bf Ali\-HLTPHOSHistogram\-Producer\-Component} {\rm (p.\,\pageref{classAliHLTPHOSHistogramProducerComponent_a13})}, {\bf Ali\-HLTPHOSModule\-Merger\-Component} {\rm (p.\,\pageref{classAliHLTPHOSModuleMergerComponent_a13})}, {\bf Ali\-HLTPHOSRaw\-Analyzer\-Component} {\rm (p.\,\pageref{classAliHLTPHOSRawAnalyzerComponent_a15})}, and {\bf Ali\-HLTPHOSRcu\-Histogram\-Producer\-Component} {\rm (p.\,\pageref{classAliHLTPHOSRcuHistogramProducerComponent_a8})}.

Referenced by Ali\-HLT\_\-C\_\-Get\-Output\-Size(), and Ali\-HLTTask::Process\-Task().\index{AliHLTComponent@{Ali\-HLTComponent}!GetOutputDataType@{GetOutputDataType}}
\index{GetOutputDataType@{GetOutputDataType}!AliHLTComponent@{Ali\-HLTComponent}}
\subsubsection{\setlength{\rightskip}{0pt plus 5cm}virtual {\bf Ali\-HLTComponent\-Data\-Type} Ali\-HLTComponent::Get\-Output\-Data\-Type ()\hspace{0.3cm}{\tt  [pure virtual]}}\label{classAliHLTComponent_a11}


Get the output data type of the component. The function is pure virtual and must be implemented by the child class. \begin{Desc}
\item[Returns:]output data type \end{Desc}


Implemented in {\bf Ali\-HLTData\-Sink} {\rm (p.\,\pageref{classAliHLTDataSink_a4})}, {\bf Ali\-HLTPHOSDDLDecoder\-Component} {\rm (p.\,\pageref{classAliHLTPHOSDDLDecoderComponent_a12})}, {\bf Ali\-HLTPHOSHistogram\-Producer\-Component} {\rm (p.\,\pageref{classAliHLTPHOSHistogramProducerComponent_a12})}, {\bf Ali\-HLTPHOSModule\-Merger\-Component} {\rm (p.\,\pageref{classAliHLTPHOSModuleMergerComponent_a12})}, {\bf Ali\-HLTPHOSRaw\-Analyzer\-Component} {\rm (p.\,\pageref{classAliHLTPHOSRawAnalyzerComponent_a14})}, and {\bf Ali\-HLTPHOSRcu\-Histogram\-Producer\-Component} {\rm (p.\,\pageref{classAliHLTPHOSRcuHistogramProducerComponent_a7})}.

Referenced by Ali\-HLT\_\-C\_\-Get\-Output\-Data\-Type(), Find\-Matching\-Data\-Types(), and Make\-Output\-Data\-Block\-List().\index{AliHLTComponent@{Ali\-HLTComponent}!GetSpecification@{GetSpecification}}
\index{GetSpecification@{GetSpecification}!AliHLTComponent@{Ali\-HLTComponent}}
\subsubsection{\setlength{\rightskip}{0pt plus 5cm}{\bf Ali\-HLTUInt32\_\-t} Ali\-HLTComponent::Get\-Specification (const {\bf Ali\-HLTComponent\-Block\-Data} $\ast$ {\em p\-Block} = {\tt NULL})\hspace{0.3cm}{\tt  [protected]}}\label{classAliHLTComponent_b22}


Get data specification of an input block. Get data specification of the input bblock previously fetched via Get\-First\-Input\-Object/Next\-Input\-Object or the last one if no block specified. \begin{Desc}
\item[Parameters:]
\begin{description}
\item[{\em p\-Block}]pointer to input block \end{description}
\end{Desc}
\begin{Desc}
\item[Returns:]data specification, k\-Ali\-HLTVoid\-Data\-Spec if failed \end{Desc}


Definition at line 676 of file Ali\-HLTComponent.cxx.

References ALIHLTCOMPONENT\_\-BASE\_\-STOPWATCH, Ali\-HLTUInt32\_\-t, Find\-Input\-Block(), fp\-Input\-Blocks, fp\-Input\-Objects, Ali\-HLTComponent\-Block\-Data::f\-Specification, and HLTError.

\footnotesize\begin{verbatim}677 {
678   // see header file for function documentation
679   ALIHLTCOMPONENT_BASE_STOPWATCH();
680   AliHLTUInt32_t iSpec=kAliHLTVoidDataSpec;
681   int idx=fCurrentInputBlock;
682   if (pBlock) {
683     if (fpInputObjects==NULL || (idx=FindInputBlock(pBlock))>=0) {
684     } else {
685       HLTError("unknown Block %p", pBlock);
686     }
687   }
688   if (idx>=0) {
689     // check for fpInputBlocks pointer done in FindInputBlock
690     iSpec=fpInputBlocks[idx].fSpecification;
691   }
692   return iSpec;
693 }
\end{verbatim}\normalsize 


\index{AliHLTComponent@{Ali\-HLTComponent}!GetSpecification@{GetSpecification}}
\index{GetSpecification@{GetSpecification}!AliHLTComponent@{Ali\-HLTComponent}}
\subsubsection{\setlength{\rightskip}{0pt plus 5cm}{\bf Ali\-HLTUInt32\_\-t} Ali\-HLTComponent::Get\-Specification (const TObject $\ast$ {\em p\-Object} = {\tt NULL})\hspace{0.3cm}{\tt  [protected]}}\label{classAliHLTComponent_b17}


Get data specification of an input block. Get data specification of the object previously fetched via Get\-First\-Input\-Object/Next\-Input\-Object or the last one if no object specified. \begin{Desc}
\item[Parameters:]
\begin{description}
\item[{\em p\-Object}]pointer to TObject \end{description}
\end{Desc}
\begin{Desc}
\item[Returns:]data specification, k\-Ali\-HLTVoid\-Data\-Spec if failed \end{Desc}


Definition at line 600 of file Ali\-HLTComponent.cxx.

References ALIHLTCOMPONENT\_\-BASE\_\-STOPWATCH, Ali\-HLTUInt32\_\-t, Ali\-HLTComponent\-Event\-Data::f\-Block\-Cnt, f\-Current\-Event\-Data, fp\-Input\-Blocks, fp\-Input\-Objects, Ali\-HLTComponent\-Block\-Data::f\-Specification, HLTError, and HLTFatal.

\footnotesize\begin{verbatim}601 {
602   // see header file for function documentation
603   ALIHLTCOMPONENT_BASE_STOPWATCH();
604   AliHLTUInt32_t iSpec=kAliHLTVoidDataSpec;
605   int idx=fCurrentInputBlock;
606   if (pObject) {
607     if (fpInputObjects==NULL || (idx=fpInputObjects->IndexOf(pObject))>=0) {
608     } else {
609       HLTError("unknown object %p", pObject);
610     }
611   }
612   if (idx>=0) {
613     if ((UInt_t)idx<fCurrentEventData.fBlockCnt) {
614       iSpec=fpInputBlocks[idx].fSpecification;
615     } else {
616       HLTFatal("severe internal error, index out of range");
617     }
618   }
619   return iSpec;
620 }
\end{verbatim}\normalsize 


\index{AliHLTComponent@{Ali\-HLTComponent}!IncrementEventCounter@{IncrementEventCounter}}
\index{IncrementEventCounter@{IncrementEventCounter}!AliHLTComponent@{Ali\-HLTComponent}}
\subsubsection{\setlength{\rightskip}{0pt plus 5cm}int Ali\-HLTComponent::Increment\-Event\-Counter ()\hspace{0.3cm}{\tt  [private]}}\label{classAliHLTComponent_d0}


Increment the internal event counter. To be used by the friend classes {\bf Ali\-HLTProcessor}{\rm (p.\,\pageref{classAliHLTProcessor})}, {\bf Ali\-HLTData\-Source}{\rm (p.\,\pageref{classAliHLTDataSource})} and {\bf Ali\-HLTData\-Sink}{\rm (p.\,\pageref{classAliHLTDataSink})}. \begin{Desc}
\item[Returns:]new value of the internal event counter \end{Desc}


Definition at line 430 of file Ali\-HLTComponent.cxx.

References f\-Event\-Count.

Referenced by Process\-Event().

\footnotesize\begin{verbatim}431 {
432   // see header file for function documentation
433   if (fEventCount>=0) fEventCount++;
434   return fEventCount;
435 }
\end{verbatim}\normalsize 


\index{AliHLTComponent@{Ali\-HLTComponent}!Init@{Init}}
\index{Init@{Init}!AliHLTComponent@{Ali\-HLTComponent}}
\subsubsection{\setlength{\rightskip}{0pt plus 5cm}int Ali\-HLTComponent::Init ({\bf Ali\-HLTComponent\-Environment} $\ast$ {\em environ}, void $\ast$ {\em environ\-Param}, int {\em argc}, const char $\ast$$\ast$ {\em argv})\hspace{0.3cm}{\tt  [virtual]}}\label{classAliHLTComponent_a4}


Init function to prepare data processing. Initialization of common data structures for a sequence of events. The call is redirected to the internal method {\bf Do\-Init}{\rm (p.\,\pageref{classAliHLTComponent_b5})} which can be overridden by the child class.\par
 During Init also the environment structure is passed to the component. \begin{Desc}
\item[Parameters:]
\begin{description}
\item[{\em environ}]environment pointer with environment dependend function calls \item[{\em environ\-Param}]additionel parameter for function calls, the pointer is passed as it is \item[{\em argc}]size of the argument array \item[{\em argv}]agument array for component initialization \end{description}
\end{Desc}


Reimplemented in {\bf Ali\-HLTProcessor} {\rm (p.\,\pageref{classAliHLTProcessor_a2})}.

Definition at line 135 of file Ali\-HLTComponent.cxx.

References Ali\-HLTComponent\-Log\-Severity, Do\-Init(), f\-Environment, f\-Event\-Count, Ali\-HLTComponent\-Environment::f\-Param, HLTError, and Ali\-HLTLogging::Set\-Local\-Logging\-Level().

Referenced by Ali\-HLTComponent\-Handler::Create\-Component(), and Ali\-HLTProcessor::Init().

\footnotesize\begin{verbatim}136 {
137   // see header file for function documentation
138   int iResult=0;
139   if (environ) {
140     memcpy(&fEnvironment, environ, sizeof(AliHLTComponentEnvironment));
141     fEnvironment.fParam=environParam;
142   }
143   const char** pArguments=NULL;
144   int iNofChildArgs=0;
145   TString argument="";
146   int bMissingParam=0;
147   if (argc>0) {
148     pArguments=new const char*[argc];
149     if (pArguments) {
150       for (int i=0; i<argc && iResult>=0; i++) {
151         argument=argv[i];
152         if (argument.IsNull()) continue;
153 
154         // benchmark
155         if (argument.CompareTo("benchmark")==0) {
156 
157           // loglevel
158         } else if (argument.CompareTo("loglevel")==0) {
159           if ((bMissingParam=(++i>=argc))) break;
160           TString parameter(argv[i]);
161           parameter.Remove(TString::kLeading, ' '); // remove all blanks
162           if (parameter.BeginsWith("0x") &&
163               parameter.Replace(0,2,"",0).IsHex()) {
164             AliHLTComponentLogSeverity loglevel=kHLTLogNone;
165             sscanf(parameter.Data(),"%x", (unsigned int*)&loglevel);
166             SetLocalLoggingLevel(loglevel);
167           } else {
168             HLTError("wrong parameter for argument %s, hex number expected", argument.Data());
169             iResult=-EINVAL;
170           }
171         } else {
172           pArguments[iNofChildArgs++]=argv[i];
173         }
174       }
175     } else {
176       iResult=-ENOMEM;
177     }
178   }
179   if (bMissingParam) {
180     HLTError("missing parameter for argument %s", argument.Data());
181     iResult=-EINVAL;
182   }
183   if (iResult>=0) {
184     iResult=DoInit(iNofChildArgs, pArguments);
185   }
186   if (iResult>=0) fEventCount=0;
187   if (pArguments) delete [] pArguments;
188   return iResult;
189 }
\end{verbatim}\normalsize 


\index{AliHLTComponent@{Ali\-HLTComponent}!InsertOutputBlock@{InsertOutputBlock}}
\index{InsertOutputBlock@{InsertOutputBlock}!AliHLTComponent@{Ali\-HLTComponent}}
\subsubsection{\setlength{\rightskip}{0pt plus 5cm}int Ali\-HLTComponent::Insert\-Output\-Block (void $\ast$ {\em p\-Buffer}, int {\em i\-Size}, const {\bf Ali\-HLTComponent\-Data\-Type} \& {\em dt}, {\bf Ali\-HLTUInt32\_\-t} {\em spec})\hspace{0.3cm}{\tt  [private]}}\label{classAliHLTComponent_d6}


Insert a buffer into the output block stream. This is the only method to insert blocks into the output stream, called from all types of the Pushback method. The actual data might have been written to the output buffer already. In that case NULL can be provided as buffer, only the block descriptor will be build. \begin{Desc}
\item[Parameters:]
\begin{description}
\item[{\em p\-Buffer}]pointer to buffer \item[{\em i\-Size}]size of the buffer in byte \item[{\em dt}]data type \item[{\em spec}]data specification \end{description}
\end{Desc}


Definition at line 745 of file Ali\-HLTComponent.cxx.

References Ali\-HLTUInt8\_\-t, Ali\-HLTComponent\-Block\-Data::f\-Data\-Type, Fill\-Block\-Data(), Ali\-HLTComponent\-Block\-Data::f\-Offset, f\-Output\-Blocks, f\-Output\-Buffer\-Filled, f\-Output\-Buffer\-Size, fp\-Output\-Buffer, Ali\-HLTComponent\-Block\-Data::f\-Ptr, Ali\-HLTComponent\-Block\-Data::f\-Size, Ali\-HLTComponent\-Block\-Data::f\-Specification, and HLTError.

Referenced by Push\-Back().

\footnotesize\begin{verbatim}746 {
747   // see header file for function documentation
748   int iResult=0;
749   if (pBuffer) {
750     if (fpOutputBuffer && iSize<=(int)(fOutputBufferSize-fOutputBufferFilled)) {
751       AliHLTUInt8_t* pTgt=fpOutputBuffer+fOutputBufferFilled;
752       AliHLTComponentBlockData bd;
753       FillBlockData( bd );
754       bd.fOffset        = fOutputBufferFilled;
755       bd.fPtr           = pTgt;
756       bd.fSize          = iSize;
757       bd.fDataType      = dt;
758       bd.fSpecification = spec;
759       if (pBuffer!=NULL && pBuffer!=pTgt) {
760         memcpy(pTgt, pBuffer, iSize);
761         //AliHLTUInt32_t firstWord=*((AliHLTUInt32_t*)pBuffer); 
762         //HLTDebug("copy %d bytes from %p to output buffer %p, first word %#x", iSize, pBuffer, pTgt, firstWord);
763       }
764       fOutputBufferFilled+=bd.fSize;
765       fOutputBlocks.push_back( bd );
766       //HLTDebug("buffer inserted to output: size %d data type %s spec %#x", iSize, DataType2Text(dt).c_str(), spec);
767     } else {
768       if (fpOutputBuffer) {
769         HLTError("too little space in output buffer: %d, required %d", fOutputBufferSize-fOutputBufferFilled, iSize);
770       } else {
771         HLTError("output buffer not available");
772       }
773       iResult=-ENOSPC;
774     }
775   } else {
776     iResult=-EINVAL;
777   }
778   return iResult;
779 }
\end{verbatim}\normalsize 


\index{AliHLTComponent@{Ali\-HLTComponent}!MakeOutputDataBlockList@{MakeOutputDataBlockList}}
\index{MakeOutputDataBlockList@{MakeOutputDataBlockList}!AliHLTComponent@{Ali\-HLTComponent}}
\subsubsection{\setlength{\rightskip}{0pt plus 5cm}int Ali\-HLTComponent::Make\-Output\-Data\-Block\-List (const vector$<$ {\bf Ali\-HLTComponent\-Block\-Data} $>$ \& {\em blocks}, {\bf Ali\-HLTUInt32\_\-t} $\ast$ {\em block\-Count}, {\bf Ali\-HLTComponent\-Block\-Data} $\ast$$\ast$ {\em output\-Blocks})\hspace{0.3cm}{\tt  [protected]}}\label{classAliHLTComponent_b8}


Helper function to create a monolithic Block\-Data description block out of a list Block\-Data descriptors. For convenience, inside the interface vector lists are used, to make the interface pure C style, monilithic blocks must be exchanged. The method is redirected to a function provided by the current framework. Function pointers are transferred via the {\bf Ali\-HLTComponent\-Environment}{\rm (p.\,\pageref{structAliHLTComponentEnvironment})} structure. 

Definition at line 258 of file Ali\-HLTComponent.cxx.

References Ali\-HLTUInt32\_\-t, Alloc\-Memory(), Ali\-HLTComponent\-Block\-Data::f\-Data\-Type, and Get\-Output\-Data\-Type().

Referenced by Process\-Event().

\footnotesize\begin{verbatim}260 {
261   // see header file for function documentation
262     if ( blockCount==NULL || outputBlocks==NULL )
263         return -EFAULT;
264     AliHLTUInt32_t count = blocks.size();
265     if ( !count )
266         {
267         *blockCount = 0;
268         *outputBlocks = NULL;
269         return 0;
270         }
271     *outputBlocks = reinterpret_cast<AliHLTComponentBlockData*>( AllocMemory( sizeof(AliHLTComponentBlockData)*count ) );
272     if ( !*outputBlocks )
273         return -ENOMEM;
274     for ( unsigned long i = 0; i < count; i++ ) {
275         (*outputBlocks)[i] = blocks[i];
276         if (blocks[i].fDataType==kAliHLTAnyDataType) {
277           (*outputBlocks)[i].fDataType=GetOutputDataType();
278           /* data type was set to the output data type by the PubSub AliRoot
279              Wrapper component, if data type of the block was ********:****.
280              Now handled by the component base class in order to have same
281              behavior when running embedded in AliRoot
282           memset((*outputBlocks)[i].fDataType.fID, '*', kAliHLTComponentDataTypefIDsize);
283           memset((*outputBlocks)[i].fDataType.fOrigin, '*', kAliHLTComponentDataTypefOriginSize);
284           */
285         }
286     }
287     *blockCount = count;
288     return 0;
289 
290 }
\end{verbatim}\normalsize 


\index{AliHLTComponent@{Ali\-HLTComponent}!operator=@{operator=}}
\index{operator=@{operator=}!AliHLTComponent@{Ali\-HLTComponent}}
\subsubsection{\setlength{\rightskip}{0pt plus 5cm}{\bf Ali\-HLTComponent} \& Ali\-HLTComponent::operator= (const {\bf Ali\-HLTComponent} \&)}\label{classAliHLTComponent_a2}


not a valid assignment op, but defined according to effective C++ style 

Definition at line 101 of file Ali\-HLTComponent.cxx.

References HLTFatal.

\footnotesize\begin{verbatim}102 { 
103   // see header file for class documentation
104   HLTFatal("assignment operator untested");
105   return *this;
106 }
\end{verbatim}\normalsize 


\index{AliHLTComponent@{Ali\-HLTComponent}!PrintComponentDataTypeInfo@{PrintComponentDataTypeInfo}}
\index{PrintComponentDataTypeInfo@{PrintComponentDataTypeInfo}!AliHLTComponent@{Ali\-HLTComponent}}
\subsubsection{\setlength{\rightskip}{0pt plus 5cm}void Ali\-HLTComponent::Print\-Component\-Data\-Type\-Info (const {\bf Ali\-HLTComponent\-Data\-Type} \& {\em dt})}\label{classAliHLTComponent_a16}


Print info on an {\bf Ali\-HLTComponent\-Data\-Type}{\rm (p.\,\pageref{structAliHLTComponentDataType})} structure This is just a helper function to examine an {\bf Ali\-HLTComponent\-Data\-Type}{\rm (p.\,\pageref{structAliHLTComponentDataType})} structur. 

Definition at line 406 of file Ali\-HLTComponent.cxx.

References Ali\-HLTComponent\-Data\-Type::f\-ID, Ali\-HLTComponent\-Data\-Type::f\-Origin, Ali\-HLTComponent\-Data\-Type::f\-Struct\-Size, k\-HLTLog\-None, and Ali\-HLTLogging::Message().

\footnotesize\begin{verbatim}407 {
408   // see header file for function documentation
409   TString msg;
410   msg.Form("AliHLTComponentDataType(%d): ID=\"", dt.fStructSize);
411   for ( int i = 0; i < kAliHLTComponentDataTypefIDsize; i++ ) {
412    if (dt.fID[i]!=0) msg+=dt.fID[i];
413    else msg+="\\0";
414   }
415   msg+="\" Origin=\"";
416   for ( int i = 0; i < kAliHLTComponentDataTypefOriginSize; i++ ) {
417    if (dt.fOrigin[i]!=0) msg+=dt.fOrigin[i];
418    else msg+="\\0";
419   }
420   msg+="\"";
421   AliHLTLogging::Message(NULL, kHLTLogNone, NULL , NULL, msg.Data());
422 }
\end{verbatim}\normalsize 


\index{AliHLTComponent@{Ali\-HLTComponent}!PrintDataTypeContent@{PrintDataTypeContent}}
\index{PrintDataTypeContent@{PrintDataTypeContent}!AliHLTComponent@{Ali\-HLTComponent}}
\subsubsection{\setlength{\rightskip}{0pt plus 5cm}void Ali\-HLTComponent::Print\-Data\-Type\-Content ({\bf Ali\-HLTComponent\-Data\-Type} \& {\em dt}, const char $\ast$ {\em format} = {\tt NULL}) const}\label{classAliHLTComponent_a15}


Helper function to print content of data type. 

Definition at line 327 of file Ali\-HLTComponent.cxx.

References Data\-Type2Text(), Ali\-HLTComponent\-Data\-Type::f\-ID, Ali\-HLTComponent\-Data\-Type::f\-Origin, and HLTMessage.

\footnotesize\begin{verbatim}328 {
329   // see header file for function documentation
330   const char* fmt="publisher \'%s\'";
331   if (format) fmt=format;
332   HLTMessage(fmt, (DataType2Text(dt)).c_str());
333   HLTMessage("%x %x %x %x %x %x %x %x : %x %x %x %x", 
334              dt.fID[0],
335              dt.fID[1],
336              dt.fID[2],
337              dt.fID[3],
338              dt.fID[4],
339              dt.fID[5],
340              dt.fID[6],
341              dt.fID[7],
342              dt.fOrigin[0],
343              dt.fOrigin[1],
344              dt.fOrigin[2],
345              dt.fOrigin[3]);
346 }
\end{verbatim}\normalsize 


\index{AliHLTComponent@{Ali\-HLTComponent}!ProcessEvent@{ProcessEvent}}
\index{ProcessEvent@{ProcessEvent}!AliHLTComponent@{Ali\-HLTComponent}}
\subsubsection{\setlength{\rightskip}{0pt plus 5cm}int Ali\-HLTComponent::Process\-Event (const {\bf Ali\-HLTComponent\-Event\-Data} \& {\em evt\-Data}, const {\bf Ali\-HLTComponent\-Block\-Data} $\ast$ {\em blocks}, {\bf Ali\-HLTComponent\-Trigger\-Data} \& {\em trig\-Data}, {\bf Ali\-HLTUInt8\_\-t} $\ast$ {\em output\-Ptr}, {\bf Ali\-HLTUInt32\_\-t} \& {\em size}, {\bf Ali\-HLTUInt32\_\-t} \& {\em output\-Block\-Cnt}, {\bf Ali\-HLTComponent\-Block\-Data} $\ast$\& {\em output\-Blocks}, {\bf Ali\-HLTComponent\-Event\-Done\-Data} $\ast$\& {\em edd})}\label{classAliHLTComponent_a6}


Processing of one event. The method is the entrance of the event processing. The parameters are cached for uses with the high-level interface and the Do\-Processing implementation is called.

\begin{Desc}
\item[Parameters:]
\begin{description}
\item[{\em evt\-Data}]\item[{\em blocks}]\item[{\em trig\-Data}]\item[{\em output\-Ptr}]\item[{\em size}]\item[{\em output\-Block\-Cnt}]out: size of the output block array, set by the component \item[{\em output\-Blocks}]out: the output block array is allocated internally \item[{\em edd}]\end{description}
\end{Desc}
\begin{Desc}
\item[Returns:]neg. error code if failed \end{Desc}


Definition at line 799 of file Ali\-HLTComponent.cxx.

References ALIHLTCOMPONENT\_\-BASE\_\-STOPWATCH, ALIHLTCOMPONENT\_\-DA\_\-STOPWATCH, Cleanup\-Input\-Objects(), Do\-Processing(), f\-Current\-Event, f\-Current\-Event\-Data, f\-Current\-Input\-Block, Ali\-HLTComponent\-Event\-Data::f\-Event\-ID, f\-Output\-Blocks, f\-Output\-Buffer\-Filled, f\-Output\-Buffer\-Size, fp\-Input\-Blocks, fp\-Output\-Buffer, f\-Search\-Data\-Type, Get\-Component\-ID(), HLTError, HLTFatal, Increment\-Event\-Counter(), and Make\-Output\-Data\-Block\-List().

Referenced by Ali\-HLT\_\-C\_\-Process\-Event(), and Ali\-HLTTask::Process\-Task().

\footnotesize\begin{verbatim}807 {
808   // see header file for function documentation
809   ALIHLTCOMPONENT_BASE_STOPWATCH();
810   int iResult=0;
811   fCurrentEvent=evtData.fEventID;
812   fCurrentEventData=evtData;
813   fpInputBlocks=blocks;
814   fCurrentInputBlock=-1;
815   fSearchDataType=kAliHLTAnyDataType;
816   fpOutputBuffer=outputPtr;
817   fOutputBufferSize=size;
818   fOutputBufferFilled=0;
819   fOutputBlocks.clear();
820   
821   vector<AliHLTComponentBlockData> blockData;
822   { // dont delete, sets the scope for the stopwatch guard
823     ALIHLTCOMPONENT_DA_STOPWATCH();
824     iResult=DoProcessing(evtData, blocks, trigData, outputPtr, size, blockData, edd);
825   } // end of the scope of the stopwatch guard
826   if (iResult>=0) {
827     if (fOutputBlocks.size()>0) {
828       //HLTDebug("got %d block(s) via high level interface", fOutputBlocks.size());
829       if (blockData.size()>0) {
830         HLTError("low level and high interface must not be mixed; use PushBack methods to insert data blocks");
831         iResult=-EFAULT;
832       } else {
833         iResult=MakeOutputDataBlockList(fOutputBlocks, &outputBlockCnt, &outputBlocks);
834         size=fOutputBufferFilled;
835       }
836     } else {
837       iResult=MakeOutputDataBlockList(blockData, &outputBlockCnt, &outputBlocks);
838     }
839     if (iResult<0) {
840       HLTFatal("component %s (%p): can not convert output block descriptor list", GetComponentID(), this);
841     }
842   }
843   if (iResult<0) {
844     outputBlockCnt=0;
845     outputBlocks=NULL;
846   }
847   CleanupInputObjects();
848   IncrementEventCounter();
849   return iResult;
850 }
\end{verbatim}\normalsize 


\index{AliHLTComponent@{Ali\-HLTComponent}!PushBack@{PushBack}}
\index{PushBack@{PushBack}!AliHLTComponent@{Ali\-HLTComponent}}
\subsubsection{\setlength{\rightskip}{0pt plus 5cm}int Ali\-HLTComponent::Push\-Back (void $\ast$ {\em p\-Buffer}, int {\em i\-Size}, const char $\ast$ {\em dt\-ID}, const char $\ast$ {\em dt\-Origin}, {\bf Ali\-HLTUInt32\_\-t} {\em spec} = {\tt {\bf k\-Ali\-HLTVoid\-Data\-Spec}})\hspace{0.3cm}{\tt  [protected]}}\label{classAliHLTComponent_b26}


Insert an object into the output. \begin{Desc}
\item[Parameters:]
\begin{description}
\item[{\em p\-Buffer}]pointer to buffer \item[{\em i\-Size}]size of the buffer \item[{\em dt\-ID}]data type ID of the object \item[{\em dt\-Origin}]data type origin of the object \item[{\em spec}]data specification \end{description}
\end{Desc}
\begin{Desc}
\item[Returns:]neg. error code if failed \end{Desc}


Definition at line 736 of file Ali\-HLTComponent.cxx.

References ALIHLTCOMPONENT\_\-BASE\_\-STOPWATCH, Push\-Back(), and Set\-Data\-Type().

\footnotesize\begin{verbatim}737 {
738   // see header file for function documentation
739   ALIHLTCOMPONENT_BASE_STOPWATCH();
740   AliHLTComponentDataType dt;
741   SetDataType(dt, dtID, dtOrigin);
742   return PushBack(pBuffer, iSize, dt, spec);
743 }
\end{verbatim}\normalsize 


\index{AliHLTComponent@{Ali\-HLTComponent}!PushBack@{PushBack}}
\index{PushBack@{PushBack}!AliHLTComponent@{Ali\-HLTComponent}}
\subsubsection{\setlength{\rightskip}{0pt plus 5cm}int Ali\-HLTComponent::Push\-Back (void $\ast$ {\em p\-Buffer}, int {\em i\-Size}, const {\bf Ali\-HLTComponent\-Data\-Type} \& {\em dt}, {\bf Ali\-HLTUInt32\_\-t} {\em spec} = {\tt {\bf k\-Ali\-HLTVoid\-Data\-Spec}})\hspace{0.3cm}{\tt  [protected]}}\label{classAliHLTComponent_b25}


Insert an object into the output. \begin{Desc}
\item[Parameters:]
\begin{description}
\item[{\em p\-Buffer}]pointer to buffer \item[{\em i\-Size}]size of the buffer \item[{\em dt}]data type of the object \item[{\em spec}]data specification \end{description}
\end{Desc}
\begin{Desc}
\item[Returns:]neg. error code if failed \end{Desc}


Definition at line 729 of file Ali\-HLTComponent.cxx.

References ALIHLTCOMPONENT\_\-BASE\_\-STOPWATCH, and Insert\-Output\-Block().

\footnotesize\begin{verbatim}730 {
731   // see header file for function documentation
732   ALIHLTCOMPONENT_BASE_STOPWATCH();
733   return InsertOutputBlock(pBuffer, iSize, dt, spec);
734 }
\end{verbatim}\normalsize 


\index{AliHLTComponent@{Ali\-HLTComponent}!PushBack@{PushBack}}
\index{PushBack@{PushBack}!AliHLTComponent@{Ali\-HLTComponent}}
\subsubsection{\setlength{\rightskip}{0pt plus 5cm}int Ali\-HLTComponent::Push\-Back (TObject $\ast$ {\em p\-Object}, const char $\ast$ {\em dt\-ID}, const char $\ast$ {\em dt\-Origin}, {\bf Ali\-HLTUInt32\_\-t} {\em spec} = {\tt {\bf k\-Ali\-HLTVoid\-Data\-Spec}})\hspace{0.3cm}{\tt  [protected]}}\label{classAliHLTComponent_b24}


Insert an object into the output. \begin{Desc}
\item[Parameters:]
\begin{description}
\item[{\em p\-Object}]pointer to root object \item[{\em dt\-ID}]data type ID of the object \item[{\em dt\-Origin}]data type origin of the object \item[{\em spec}]data specification \end{description}
\end{Desc}
\begin{Desc}
\item[Returns:]neg. error code if failed \end{Desc}


Definition at line 720 of file Ali\-HLTComponent.cxx.

References ALIHLTCOMPONENT\_\-BASE\_\-STOPWATCH, Push\-Back(), and Set\-Data\-Type().

\footnotesize\begin{verbatim}721 {
722   // see header file for function documentation
723   ALIHLTCOMPONENT_BASE_STOPWATCH();
724   AliHLTComponentDataType dt;
725   SetDataType(dt, dtID, dtOrigin);
726   return PushBack(pObject, dt, spec);
727 }
\end{verbatim}\normalsize 


\index{AliHLTComponent@{Ali\-HLTComponent}!PushBack@{PushBack}}
\index{PushBack@{PushBack}!AliHLTComponent@{Ali\-HLTComponent}}
\subsubsection{\setlength{\rightskip}{0pt plus 5cm}int Ali\-HLTComponent::Push\-Back (TObject $\ast$ {\em p\-Object}, const {\bf Ali\-HLTComponent\-Data\-Type} \& {\em dt}, {\bf Ali\-HLTUInt32\_\-t} {\em spec} = {\tt {\bf k\-Ali\-HLTVoid\-Data\-Spec}})\hspace{0.3cm}{\tt  [protected]}}\label{classAliHLTComponent_b23}


Insert an object into the output. \begin{Desc}
\item[Parameters:]
\begin{description}
\item[{\em p\-Object}]pointer to root object \item[{\em dt}]data type of the object \item[{\em spec}]data specification \end{description}
\end{Desc}
\begin{Desc}
\item[Returns:]neg. error code if failed \end{Desc}


Definition at line 695 of file Ali\-HLTComponent.cxx.

References ALIHLTCOMPONENT\_\-BASE\_\-STOPWATCH, HLTDebug, HLTError, Insert\-Output\-Block(), and Ali\-HLTMessage::Set\-Length().

Referenced by Push\-Back().

\footnotesize\begin{verbatim}696 {
697   // see header file for function documentation
698   ALIHLTCOMPONENT_BASE_STOPWATCH();
699   int iResult=0;
700   if (pObject) {
701     AliHLTMessage msg(kMESS_OBJECT);
702     msg.WriteObject(pObject);
703     Int_t iMsgLength=msg.Length();
704     if (iMsgLength>0) {
705       msg.SetLength(); // sets the length to the first (reserved) word
706       iResult=InsertOutputBlock(msg.Buffer(), iMsgLength, dt, spec);
707       if (iResult>=0) {
708         HLTDebug("object %s (%p) size %d inserted to output", pObject->ClassName(), pObject, iMsgLength);
709       }
710     } else {
711       HLTError("object serialization failed for object %p", pObject);
712       iResult=-ENOMSG;
713     }
714   } else {
715     iResult=-EINVAL;
716   }
717   return iResult;
718 }
\end{verbatim}\normalsize 


\index{AliHLTComponent@{Ali\-HLTComponent}!SetDataType@{SetDataType}}
\index{SetDataType@{SetDataType}!AliHLTComponent@{Ali\-HLTComponent}}
\subsubsection{\setlength{\rightskip}{0pt plus 5cm}void Ali\-HLTComponent::Set\-Data\-Type ({\bf Ali\-HLTComponent\-Data\-Type} \& {\em tgtdt}, const char $\ast$ {\em id}, const char $\ast$ {\em origin})\hspace{0.3cm}{\tt  [protected]}}\label{classAliHLTComponent_b4}


Set the ID and Origin of an {\bf Ali\-HLTComponent\-Data\-Type}{\rm (p.\,\pageref{structAliHLTComponentDataType})} structure. The function sets the f\-Structure\-Size member and copies the strings to the ID and Origin. Only characters from the valid part of the string are copied, the rest is fille with 0's. Please note that the f\-ID and f\-Origin members are not strings, just arrays of chars of size {\bf k\-Ali\-HLTComponent\-Data\-Typef\-IDsize}{\rm (p.\,\pageref{AliHLTDataTypes_8h_a12})} and {\bf k\-Ali\-HLTComponent\-Data\-Typef\-Origin\-Size}{\rm (p.\,\pageref{AliHLTDataTypes_8h_a13})} respectively and not necessarily with a terminating zero. \begin{Desc}
\item[Parameters:]
\begin{description}
\item[{\em tgtdt}]target data type structure \item[{\em id}]ID string \item[{\em origin}]Origin string \end{description}
\end{Desc}


Definition at line 381 of file Ali\-HLTComponent.cxx.

References Ali\-HLTComponent\-Data\-Type::f\-ID, Ali\-HLTComponent\-Data\-Type::f\-Origin, Ali\-HLTComponent\-Data\-Type::f\-Struct\-Size, HLTWarning, k\-Ali\-HLTComponent\-Data\-Typef\-IDsize, and k\-Ali\-HLTComponent\-Data\-Typef\-Origin\-Size.

Referenced by Get\-First\-Input\-Block(), Get\-First\-Input\-Object(), and Push\-Back().

\footnotesize\begin{verbatim}382 {
383   // see header file for function documentation
384   tgtdt.fStructSize = sizeof(AliHLTComponentDataType);
385   memset(&tgtdt.fID[0], 0, kAliHLTComponentDataTypefIDsize);
386   memset(&tgtdt.fOrigin[0], 0, kAliHLTComponentDataTypefOriginSize);
387 
388   if ((int)strlen(id)>kAliHLTComponentDataTypefIDsize) {
389     HLTWarning("data type id %s is too long, truncated to %d", id, kAliHLTComponentDataTypefIDsize);
390   }
391   strncpy(&tgtdt.fID[0], id, kAliHLTComponentDataTypefIDsize);
392 
393   if ((int)strlen(origin)>kAliHLTComponentDataTypefOriginSize) {
394     HLTWarning("data type origin %s is too long, truncated to %d", origin, kAliHLTComponentDataTypefOriginSize);
395   }
396   strncpy(&tgtdt.fOrigin[0], origin, kAliHLTComponentDataTypefOriginSize);
397 }
\end{verbatim}\normalsize 


\index{AliHLTComponent@{Ali\-HLTComponent}!SetGlobalComponentHandler@{SetGlobalComponentHandler}}
\index{SetGlobalComponentHandler@{SetGlobalComponentHandler}!AliHLTComponent@{Ali\-HLTComponent}}
\subsubsection{\setlength{\rightskip}{0pt plus 5cm}int Ali\-HLTComponent::Set\-Global\-Component\-Handler ({\bf Ali\-HLTComponent\-Handler} $\ast$ {\em p\-CH}, int {\em b\-Overwrite} = {\tt 0})\hspace{0.3cm}{\tt  [static]}}\label{classAliHLTComponent_e0}


Set the global component handler. The static method is needed for the automatic registration of components. 

Definition at line 118 of file Ali\-HLTComponent.cxx.

References fgp\-Component\-Handler.

Referenced by Ali\-HLTComponent\-Handler::Add\-Standard\-Components(), Ali\-HLTComponent\-Handler::Load\-Library(), and Unset\-Global\-Component\-Handler().

\footnotesize\begin{verbatim}119 {
120   // see header file for function documentation
121   int iResult=0;
122   if (fgpComponentHandler==NULL || bOverwrite!=0)
123     fgpComponentHandler=pCH;
124   else
125     iResult=-EPERM;
126   return iResult;
127 }
\end{verbatim}\normalsize 


\index{AliHLTComponent@{Ali\-HLTComponent}!SetStopwatch@{SetStopwatch}}
\index{SetStopwatch@{SetStopwatch}!AliHLTComponent@{Ali\-HLTComponent}}
\subsubsection{\setlength{\rightskip}{0pt plus 5cm}int Ali\-HLTComponent::Set\-Stopwatch (TObject $\ast$ {\em p\-SW}, {\bf Ali\-HLTStopwatch\-Type} {\em type} = {\tt kSWBase})}\label{classAliHLTComponent_a17}


Set a stopwatch for a given purpose. \begin{Desc}
\item[Parameters:]
\begin{description}
\item[{\em p\-SW}]stopwatch object \item[{\em type}]type of the stopwatch \end{description}
\end{Desc}


Definition at line 918 of file Ali\-HLTComponent.cxx.

References fp\-Stopwatches, and HLTWarning.

Referenced by Ali\-HLTSystem::Init\-Benchmarking(), and Set\-Stopwatches().

\footnotesize\begin{verbatim}919 {
920   // see header file for function documentation
921   int iResult=0;
922   if (pSW!=NULL && type<kSWTypeCount) {
923     if (fpStopwatches) {
924       TObject* pObj=fpStopwatches->At((int)type);
925       if (pSW==NULL        // explicit reset
926           || pObj==NULL) { // explicit set
927         fpStopwatches->AddAt(pSW, (int)type);
928       } else if (pObj!=pSW) {
929         HLTWarning("stopwatch %d already set, reset first", (int)type);
930         iResult=-EBUSY;
931       }
932     }
933   } else {
934     iResult=-EINVAL;
935   }
936   return iResult;
937 }
\end{verbatim}\normalsize 


\index{AliHLTComponent@{Ali\-HLTComponent}!SetStopwatches@{SetStopwatches}}
\index{SetStopwatches@{SetStopwatches}!AliHLTComponent@{Ali\-HLTComponent}}
\subsubsection{\setlength{\rightskip}{0pt plus 5cm}int Ali\-HLTComponent::Set\-Stopwatches (TObj\-Array $\ast$ {\em p\-Stopwatches})}\label{classAliHLTComponent_a18}


Init a set of stopwatches. \begin{Desc}
\item[Parameters:]
\begin{description}
\item[{\em p\-Stopwatches}]object array of stopwatches \end{description}
\end{Desc}


Definition at line 939 of file Ali\-HLTComponent.cxx.

References k\-SWType\-Count, and Set\-Stopwatch().

Referenced by Ali\-HLTSystem::Init\-Benchmarking().

\footnotesize\begin{verbatim}940 {
941   // see header file for function documentation
942   if (pStopwatches==NULL) return -EINVAL;
943 
944   int iResult=0;
945   for (int i=0 ; i<(int)kSWTypeCount && pStopwatches->GetEntries(); i++)
946     SetStopwatch(pStopwatches->At(i), (AliHLTStopwatchType)i);
947   return iResult;
948 }
\end{verbatim}\normalsize 


\index{AliHLTComponent@{Ali\-HLTComponent}!Spawn@{Spawn}}
\index{Spawn@{Spawn}!AliHLTComponent@{Ali\-HLTComponent}}
\subsubsection{\setlength{\rightskip}{0pt plus 5cm}virtual {\bf Ali\-HLTComponent}$\ast$ Ali\-HLTComponent::Spawn ()\hspace{0.3cm}{\tt  [pure virtual]}}\label{classAliHLTComponent_a13}


Spawn function. Each component must implement a spawn function to create a new instance of the class. Basically the function must return {\em new {\bf my\_\-class\_\-name}\/}. \begin{Desc}
\item[Returns:]new class instance \end{Desc}


Implemented in {\bf Ali\-HLTPHOSDDLDecoder\-Component} {\rm (p.\,\pageref{classAliHLTPHOSDDLDecoderComponent_a14})}, {\bf Ali\-HLTPHOSHistogram\-Producer\-Component} {\rm (p.\,\pageref{classAliHLTPHOSHistogramProducerComponent_a15})}, {\bf Ali\-HLTPHOSModule\-Merger\-Component} {\rm (p.\,\pageref{classAliHLTPHOSModuleMergerComponent_a15})}, {\bf Ali\-HLTPHOSRaw\-Analyzer\-Component} {\rm (p.\,\pageref{classAliHLTPHOSRawAnalyzerComponent_a16})}, {\bf Ali\-HLTPHOSRaw\-Analyzer\-Crude\-Component} {\rm (p.\,\pageref{classAliHLTPHOSRawAnalyzerCrudeComponent_a5})}, {\bf Ali\-HLTPHOSRaw\-Analyzer\-Peak\-Finder\-Component} {\rm (p.\,\pageref{classAliHLTPHOSRawAnalyzerPeakFinderComponent_a3})}, and {\bf Ali\-HLTPHOSRcu\-Histogram\-Producer\-Component} {\rm (p.\,\pageref{classAliHLTPHOSRcuHistogramProducerComponent_a9})}.

Referenced by Ali\-HLTComponent\-Handler::Create\-Component().\index{AliHLTComponent@{Ali\-HLTComponent}!UnsetGlobalComponentHandler@{UnsetGlobalComponentHandler}}
\index{UnsetGlobalComponentHandler@{UnsetGlobalComponentHandler}!AliHLTComponent@{Ali\-HLTComponent}}
\subsubsection{\setlength{\rightskip}{0pt plus 5cm}int Ali\-HLTComponent::Unset\-Global\-Component\-Handler ()\hspace{0.3cm}{\tt  [static]}}\label{classAliHLTComponent_e1}


Clear the global component handler. The static method is needed for the automatic registration of components. 

Definition at line 129 of file Ali\-HLTComponent.cxx.

References Set\-Global\-Component\-Handler().

Referenced by Ali\-HLTComponent\-Handler::Add\-Standard\-Components(), and Ali\-HLTComponent\-Handler::Load\-Library().

\footnotesize\begin{verbatim}130 {
131   // see header file for function documentation
132   return SetGlobalComponentHandler(NULL,1);
133 }
\end{verbatim}\normalsize 




\subsection{Member Data Documentation}
\index{AliHLTComponent@{Ali\-HLTComponent}!fClassName@{fClassName}}
\index{fClassName@{fClassName}!AliHLTComponent@{Ali\-HLTComponent}}
\subsubsection{\setlength{\rightskip}{0pt plus 5cm}string {\bf Ali\-HLTComponent::f\-Class\-Name}\hspace{0.3cm}{\tt  [private]}}\label{classAliHLTComponent_r8}


name of the class for the object to search for 

Definition at line 846 of file Ali\-HLTComponent.h.

Referenced by Get\-First\-Input\-Block(), Get\-First\-Input\-Object(), and Get\-Next\-Input\-Object().\index{AliHLTComponent@{Ali\-HLTComponent}!fCurrentEvent@{fCurrentEvent}}
\index{fCurrentEvent@{fCurrentEvent}!AliHLTComponent@{Ali\-HLTComponent}}
\subsubsection{\setlength{\rightskip}{0pt plus 5cm}{\bf Ali\-HLTEvent\-ID\_\-t} {\bf Ali\-HLTComponent::f\-Current\-Event}\hspace{0.3cm}{\tt  [private]}}\label{classAliHLTComponent_r1}


Set by Process\-Event before the processing starts 

Definition at line 825 of file Ali\-HLTComponent.h.

Referenced by Process\-Event().\index{AliHLTComponent@{Ali\-HLTComponent}!fCurrentEventData@{fCurrentEventData}}
\index{fCurrentEventData@{fCurrentEventData}!AliHLTComponent@{Ali\-HLTComponent}}
\subsubsection{\setlength{\rightskip}{0pt plus 5cm}{\bf Ali\-HLTComponent\-Event\-Data} {\bf Ali\-HLTComponent::f\-Current\-Event\-Data}\hspace{0.3cm}{\tt  [private]}}\label{classAliHLTComponent_r4}


event data struct of the current event under processing 

Definition at line 834 of file Ali\-HLTComponent.h.

Referenced by Create\-Input\-Object(), Find\-Input\-Block(), Get\-Data\-Type(), Get\-Input\-Object(), Get\-Number\-Of\-Input\-Blocks(), Get\-Specification(), and Process\-Event().\index{AliHLTComponent@{Ali\-HLTComponent}!fCurrentInputBlock@{fCurrentInputBlock}}
\index{fCurrentInputBlock@{fCurrentInputBlock}!AliHLTComponent@{Ali\-HLTComponent}}
\subsubsection{\setlength{\rightskip}{0pt plus 5cm}int {\bf Ali\-HLTComponent::f\-Current\-Input\-Block}\hspace{0.3cm}{\tt  [private]}}\label{classAliHLTComponent_r6}


transient 

index of the current input block 

Definition at line 840 of file Ali\-HLTComponent.h.

Referenced by Get\-First\-Input\-Object(), Get\-Next\-Input\-Block(), Get\-Next\-Input\-Object(), and Process\-Event().\index{AliHLTComponent@{Ali\-HLTComponent}!fEnvironment@{fEnvironment}}
\index{fEnvironment@{fEnvironment}!AliHLTComponent@{Ali\-HLTComponent}}
\subsubsection{\setlength{\rightskip}{0pt plus 5cm}{\bf Ali\-HLTComponent\-Environment} {\bf Ali\-HLTComponent::f\-Environment}\hspace{0.3cm}{\tt  [private]}}\label{classAliHLTComponent_r0}


transient 

The environment where the component is running in 

Definition at line 822 of file Ali\-HLTComponent.h.

Referenced by Ali\-HLTComponent(), Alloc\-Memory(), Get\-Event\-Done\-Data(), and Init().\index{AliHLTComponent@{Ali\-HLTComponent}!fEventCount@{fEventCount}}
\index{fEventCount@{fEventCount}!AliHLTComponent@{Ali\-HLTComponent}}
\subsubsection{\setlength{\rightskip}{0pt plus 5cm}int {\bf Ali\-HLTComponent::f\-Event\-Count}\hspace{0.3cm}{\tt  [private]}}\label{classAliHLTComponent_r2}


internal event no 

Reimplemented in {\bf Ali\-HLTPHOSHistogram\-Producer\-Component} {\rm (p.\,\pageref{classAliHLTPHOSHistogramProducerComponent_r3})}, {\bf Ali\-HLTPHOSModule\-Merger\-Component} {\rm (p.\,\pageref{classAliHLTPHOSModuleMergerComponent_r0})}, and {\bf Ali\-HLTPHOSRcu\-Histogram\-Producer\-Component} {\rm (p.\,\pageref{classAliHLTPHOSRcuHistogramProducerComponent_r0})}.

Definition at line 828 of file Ali\-HLTComponent.h.

Referenced by Do\-Deinit(), Do\-Init(), Increment\-Event\-Counter(), and Init().\index{AliHLTComponent@{Ali\-HLTComponent}!fFailedEvents@{fFailedEvents}}
\index{fFailedEvents@{fFailedEvents}!AliHLTComponent@{Ali\-HLTComponent}}
\subsubsection{\setlength{\rightskip}{0pt plus 5cm}int {\bf Ali\-HLTComponent::f\-Failed\-Events}\hspace{0.3cm}{\tt  [private]}}\label{classAliHLTComponent_r3}


the number of failed events 

Definition at line 831 of file Ali\-HLTComponent.h.\index{AliHLTComponent@{Ali\-HLTComponent}!fgpComponentHandler@{fgpComponentHandler}}
\index{fgpComponentHandler@{fgpComponentHandler}!AliHLTComponent@{Ali\-HLTComponent}}
\subsubsection{\setlength{\rightskip}{0pt plus 5cm}{\bf Ali\-HLTComponent\-Handler} $\ast$ {\bf Ali\-HLTComponent::fgp\-Component\-Handler} = NULL\hspace{0.3cm}{\tt  [static, private]}}\label{classAliHLTComponent_v0}


The global component handler instance 

Definition at line 116 of file Ali\-HLTComponent.cxx.

Referenced by Ali\-HLTComponent(), and Set\-Global\-Component\-Handler().\index{AliHLTComponent@{Ali\-HLTComponent}!fOutputBlocks@{fOutputBlocks}}
\index{fOutputBlocks@{fOutputBlocks}!AliHLTComponent@{Ali\-HLTComponent}}
\subsubsection{\setlength{\rightskip}{0pt plus 5cm}vector$<${\bf Ali\-HLTComponent\-Block\-Data}$>$ {\bf Ali\-HLTComponent::f\-Output\-Blocks}\hspace{0.3cm}{\tt  [private]}}\label{classAliHLTComponent_r13}


list of ouput block data descriptors 

Definition at line 861 of file Ali\-HLTComponent.h.

Referenced by Insert\-Output\-Block(), and Process\-Event().\index{AliHLTComponent@{Ali\-HLTComponent}!fOutputBufferFilled@{fOutputBufferFilled}}
\index{fOutputBufferFilled@{fOutputBufferFilled}!AliHLTComponent@{Ali\-HLTComponent}}
\subsubsection{\setlength{\rightskip}{0pt plus 5cm}{\bf Ali\-HLTUInt32\_\-t} {\bf Ali\-HLTComponent::f\-Output\-Buffer\-Filled}\hspace{0.3cm}{\tt  [private]}}\label{classAliHLTComponent_r12}


size of data written to output buffer 

Definition at line 858 of file Ali\-HLTComponent.h.

Referenced by Insert\-Output\-Block(), and Process\-Event().\index{AliHLTComponent@{Ali\-HLTComponent}!fOutputBufferSize@{fOutputBufferSize}}
\index{fOutputBufferSize@{fOutputBufferSize}!AliHLTComponent@{Ali\-HLTComponent}}
\subsubsection{\setlength{\rightskip}{0pt plus 5cm}{\bf Ali\-HLTUInt32\_\-t} {\bf Ali\-HLTComponent::f\-Output\-Buffer\-Size}\hspace{0.3cm}{\tt  [private]}}\label{classAliHLTComponent_r11}


transient 

size of the output buffer 

Definition at line 855 of file Ali\-HLTComponent.h.

Referenced by Insert\-Output\-Block(), and Process\-Event().\index{AliHLTComponent@{Ali\-HLTComponent}!fpInputBlocks@{fpInputBlocks}}
\index{fpInputBlocks@{fpInputBlocks}!AliHLTComponent@{Ali\-HLTComponent}}
\subsubsection{\setlength{\rightskip}{0pt plus 5cm}const {\bf Ali\-HLTComponent\-Block\-Data}$\ast$ {\bf Ali\-HLTComponent::fp\-Input\-Blocks}\hspace{0.3cm}{\tt  [private]}}\label{classAliHLTComponent_r5}


array of input data blocks of the current event 

Definition at line 837 of file Ali\-HLTComponent.h.

Referenced by Create\-Input\-Object(), Find\-Input\-Block(), Get\-Data\-Type(), Get\-First\-Input\-Block(), Get\-Next\-Input\-Block(), Get\-Number\-Of\-Input\-Blocks(), Get\-Specification(), and Process\-Event().\index{AliHLTComponent@{Ali\-HLTComponent}!fpInputObjects@{fpInputObjects}}
\index{fpInputObjects@{fpInputObjects}!AliHLTComponent@{Ali\-HLTComponent}}
\subsubsection{\setlength{\rightskip}{0pt plus 5cm}TObj\-Array$\ast$ {\bf Ali\-HLTComponent::fp\-Input\-Objects}\hspace{0.3cm}{\tt  [private]}}\label{classAliHLTComponent_r9}


array of generated input objects 

Definition at line 849 of file Ali\-HLTComponent.h.

Referenced by Cleanup\-Input\-Objects(), Get\-Data\-Type(), Get\-Input\-Object(), and Get\-Specification().\index{AliHLTComponent@{Ali\-HLTComponent}!fpOutputBuffer@{fpOutputBuffer}}
\index{fpOutputBuffer@{fpOutputBuffer}!AliHLTComponent@{Ali\-HLTComponent}}
\subsubsection{\setlength{\rightskip}{0pt plus 5cm}{\bf Ali\-HLTUInt8\_\-t}$\ast$ {\bf Ali\-HLTComponent::fp\-Output\-Buffer}\hspace{0.3cm}{\tt  [private]}}\label{classAliHLTComponent_r10}


transient 

the output buffer 

Definition at line 852 of file Ali\-HLTComponent.h.

Referenced by Insert\-Output\-Block(), and Process\-Event().\index{AliHLTComponent@{Ali\-HLTComponent}!fpStopwatches@{fpStopwatches}}
\index{fpStopwatches@{fpStopwatches}!AliHLTComponent@{Ali\-HLTComponent}}
\subsubsection{\setlength{\rightskip}{0pt plus 5cm}TObj\-Array$\ast$ {\bf Ali\-HLTComponent::fp\-Stopwatches}\hspace{0.3cm}{\tt  [private]}}\label{classAliHLTComponent_r14}


stopwatch array 

Definition at line 864 of file Ali\-HLTComponent.h.

Referenced by Set\-Stopwatch(), and $\sim$Ali\-HLTComponent().\index{AliHLTComponent@{Ali\-HLTComponent}!fSearchDataType@{fSearchDataType}}
\index{fSearchDataType@{fSearchDataType}!AliHLTComponent@{Ali\-HLTComponent}}
\subsubsection{\setlength{\rightskip}{0pt plus 5cm}{\bf Ali\-HLTComponent\-Data\-Type} {\bf Ali\-HLTComponent::f\-Search\-Data\-Type}\hspace{0.3cm}{\tt  [private]}}\label{classAliHLTComponent_r7}


data type of the last block search 

Definition at line 843 of file Ali\-HLTComponent.h.

Referenced by Get\-First\-Input\-Block(), Get\-First\-Input\-Object(), Get\-Next\-Input\-Block(), Get\-Next\-Input\-Object(), and Process\-Event().

The documentation for this class was generated from the following files:\begin{CompactItemize}
\item 
/home/perthi/cern/aliroot/Ali\-Root\_\-head020507/HLT/BASE/{\bf Ali\-HLTComponent.h}\item 
/home/perthi/cern/aliroot/Ali\-Root\_\-head020507/HLT/BASE/{\bf Ali\-HLTComponent.cxx}\end{CompactItemize}
