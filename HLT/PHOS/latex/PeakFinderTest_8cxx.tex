\section{Peak\-Finder\-Test.cxx File Reference}
\label{PeakFinderTest_8cxx}\index{PeakFinderTest.cxx@{PeakFinderTest.cxx}}


{\tt \#include \char`\"{}Ali\-HLTPHOSPulse\-Generator.h\char`\"{}}\par
{\tt \#include \char`\"{}Ali\-HLTPHOSAnalyzer\-Peak\-Finder.h\char`\"{}}\par
{\tt \#include $<$stdio.h$>$}\par
{\tt \#include $<$cmath$>$}\par
\subsection*{Functions}
\begin{CompactItemize}
\item 
void {\bf set\-File\-Name} (char $\ast$f\-Name, int start, int length, double tau, double fs)
\item 
int {\bf main} ()
\end{CompactItemize}


\subsection{Function Documentation}
\index{PeakFinderTest.cxx@{Peak\-Finder\-Test.cxx}!main@{main}}
\index{main@{main}!PeakFinderTest.cxx@{Peak\-Finder\-Test.cxx}}
\subsubsection{\setlength{\rightskip}{0pt plus 5cm}int main ()}\label{PeakFinderTest_8cxx_a1}


Testing of the Class Ali\-PHOSFitter 

Definition at line 11 of file Peak\-Finder\-Test.cxx.

References Ali\-HLTPHOSPulse\-Generator::Get\-Pulse(), and set\-File\-Name().

\footnotesize\begin{verbatim}12 {
13   FILE *fp = 0;
14   int start  = 0;               // Start index of subarray of sample array
15   int N = 128;                  // Number of samples 
16   char fileName[100];           // Name of file containing Peakfinder vectors
17   double amplitude;             // Amplitude/energy in ADC levels
18   double t0;                    // timedelay in nanoseconds  
19 
20   double tau = 2;               // risetime in microseconds
21   double fs = 20;               // sample frequency in Megahertz
22   double timeVector[N];         // Peakfinder vector for reconstruction of time
23   double amplitudeVector[N];    // Peakfinder vector for reconstruction of energy
24   double aSystError;
25   double tSystError;
26 
27   int ts = (int)(1000/fs);  
28   printf("\nts=%d\n", ts);
29 
30   printf("type amplitude in ADC levels (0-1023):");
31   scanf("%lf", &amplitude);
32   printf("type timedelay in nanoseconds (0-%d):", ts); 
33   scanf("%lf", &t0);
34 
35   AliHLTPHOSPulseGenerator *pulseGenPtr = new AliHLTPHOSPulseGenerator(amplitude, t0, N, tau, fs);
36   double *data = pulseGenPtr->GetPulse(); 
37 
38   setFileName(fileName, start, N, tau, fs);
39   fp = fopen(fileName,"r");
40 
41   if(fp == 0)
42     {
43       printf("\nFile does not exist\n");
44     }
45   else
46     {
47       for(int i=0; i < N; i++)
48         {
49           fscanf(fp, "%lf", &amplitudeVector[i]);
50         }
51 
52       fscanf(fp, "\n");
53 
54      for(int i=0; i < N; i++)
55         {
56           fscanf(fp, "%lf", &timeVector[i]);
57         }
58 
59      fscanf(fp, "%lf", &aSystError);
60      fscanf(fp, "%lf", &tSystError);
61      printf("\nPeakfinder vectors loaded from %s\n", fileName);
62     }
63 
64 
65    tSystError = tSystError*pow(10, 9); //to give systematic error of timing in nanoseconds
66    aSystError = aSystError*100;        //to give systematic error of amplitude in percent
67 
68 
69    //  AliHLTPHOSAnalyzerPeakFinder *fitPtr= new AliHLTPHOSAnalyzerPeakFinder(data, fs); 
70    AliHLTPHOSAnalyzerPeakFinder *fitPtr= new AliHLTPHOSAnalyzerPeakFinder(); 
71   
72    fitPtr->SetData(data);
73    fitPtr->SetSampleFreq(fs);
74    fitPtr->SetTVector(timeVector);
75    fitPtr->SetAVector(amplitudeVector);
76    //   fitPtr->Set
77    fitPtr->Evaluate(start, N);
78   
79   double energy;
80   double time;
81 
82   time    = fitPtr->GetTiming();
83   energy  = fitPtr->GetEnergy();
84 
85   printf("\nReal amplitude \t\t= %lf ADC counts \nReconstructed amplitude\t= %lf ADC counts\n", amplitude, energy);
86   printf("\nReal time \t\t= %lf nanoseconds \nReconstructed time\t= %lf nanoseconds\n", t0, time);
87   printf("\n\nMaximum systematic error in amplitude \t= %lf %%", aSystError);
88   printf("\nMaximum systematic error for timing \t= %lf nanoseconds\n\n", tSystError);
89   return 0;
90 }
\end{verbatim}\normalsize 


\index{PeakFinderTest.cxx@{Peak\-Finder\-Test.cxx}!setFileName@{setFileName}}
\index{setFileName@{setFileName}!PeakFinderTest.cxx@{Peak\-Finder\-Test.cxx}}
\subsubsection{\setlength{\rightskip}{0pt plus 5cm}void set\-File\-Name (char $\ast$ {\em f\-Name}, int {\em start}, int {\em length}, double {\em tau}, double {\em fs})}\label{PeakFinderTest_8cxx_a0}




Definition at line 92 of file Peak\-Finder\-Test.cxx.

Referenced by main().

\footnotesize\begin{verbatim}93 {
94   sprintf(fName, "PFVectors/start%dN%dtau%.ffs%.f.txt", start, N, tau, fs);
95   //  printf("\nfilename: %s\n", fName);
96 }
\end{verbatim}\normalsize 


