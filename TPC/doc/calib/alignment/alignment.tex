\documentclass{elsart}  
\usepackage{epsfig,amssymb,amsmath}  
\usepackage{rotating}
\usepackage{listings}
\usepackage{booktabs}
\usepackage{fancyhdr}

\usepackage{float}

\begin{document}



\section{ Alice space point distortion - Nonlinearities}


The electric field in the TPC in the ideal case has just z component ($E=(0,0,E_z)$).
The deviation from the ideal behavior causes a distortion which is on the level  $\approx1$ mm
close to the Outer field cage.
The drift vector follows the E field vector. The effect of distortion is integrated over the 
electron drift length. The track position resolution $\sigma$ is on the level of 100 microns.
Therefore the nonlinearities due to the E field imperfection can not be neglected.

Distortions can be estimated by histograming the difference between the tracks and the corresponding
 space points. ($\Delta_z= Z_p-Z_f$,$\Delta_y= Y_p-Y_f$).The tracks are fitted in the region where the distortion is assumed to be minimal.
In our studies, the first and the last 15 pad-rows close to the Inner and Outer field cage were removed from the fit.

To estimate radial distortion we assume following model:
\begin{equation}
\Delta_y=\Delta_{y0}(R,Z)+\Delta_R(R,Z)\tan(\Phi) 
\end{equation}
\begin{equation}
\Delta_z=\Delta_{z0}(R,Z)+\Delta_R(R,Z)\tan(\Theta)
\end{equation}

\begin{figure}
  \centering\epsfig{figure=picDistortion/hisdz_XZ0.eps,width=0.8\linewidth}	
  \centering\epsfig{figure=picDistortion/hisdz_XAside.eps,width=0.8\linewidth}
  \centering\epsfig{figure=picDistortion/hisdz_XCside.eps,width=0.8\linewidth}
  \caption{Mean residuals between the track and the space point in z direction as a function of radial position. }
  \label{figLocalZDistortion}
\end{figure}

\begin{figure}
  \centering\epsfig{figure=picDistortion/hisdy_XZ.eps,width=0.8\linewidth}	
  \centering\epsfig{figure=picDistortion/hisdy_XAside.eps,width=0.8\linewidth}
  \centering\epsfig{figure=picDistortion/hisdy_XCside.eps,width=0.8\linewidth}
  \caption{Mean residuals between the track and the space point in y direction as a function of radial position. Only tracks with positive tan($\phi$) used. }
  \label{figLocalYDistortion}
\end{figure}


The influence of radial distortion effect to the cluster residuals in y direction is visualized on picture \ref{figLocalYDistortion}.
Close to the Outer field cage the points are systematically shifted. This shift can be
fitted with exponential fit function with decay length $\approx$ 7 cm.
 
The radial distortion is smaller on C side than on A side, and is proportional to the drift length (see \ref{figRadialDistortionMapDZ}).
To describe the effect in two dimensions 2 exponential decay model was fitted.
\begin{equation}
\Delta_R=(k5_0+k5_1z+k5_2z^2)e^{-d_{out}/5}+(k10_0+k10_1z+k10_2z^2)e^{-d_{out}/10}
\end{equation}
Chosen model is linear in parameters and therefore has a analytical solution.
(I do not have better idea for the moment). 

The fitted radial distortion is also sector  dependent (see pic. \ref{figRadialDistortionMapXY})
The statistical error of the fit using 400000 tracks is about 0.2 mm.




The effect of missing correction for the radial distortion on track angular matching is visualized in picture \ref{figAngularAlignRdist}. The radial distortion has to be calibrated before, or together with the alignment. 




\begin{figure}
  \centering\epsfig{figure=picDistortion/hisdrfitAC_DZ.eps,width=0.6\linewidth}
  \caption{Fitted radial distortion map.}
  \label{figRadialDistortionMapDZ}
\end{figure}

\begin{figure}
  \centering\epsfig{figure=picDistortion/hisdrfitA_XY2.eps,width=0.6\linewidth}
  \centering\epsfig{figure=picDistortion/hisdrfitC_XY2.eps,width=0.6\linewidth}
  \caption{Fitted radial distortion map (XY).}
  \label{figRadialDistortionMapXY}
\end{figure}



\begin{figure}
  \centering\epsfig{figure=picDistortion/dphi_zphi,width=0.5\linewidth}
  \centering\epsfig{figure=picDistortion/dtheta_ztheta,width=0.5\linewidth}
  \caption{Angular matching between OROC and IROC as function of z. Indication of the radial 
           distortions }
  \label{figAngularAlignRdist}
\end{figure}

\section{R-$\Phi$ distortion}

A R-$\Phi$ distortion is obsered at the edge of the chambers. This distortion depends on the distance to the edge pad - $d_{pad0}$ and on the distance to the wire mounting ($d_{w}$).

There are following components contributing:
\begin{itemize}
\item Main component. Cluster edge effect depends on the distance to edge pad (see pic. \ref{figAngularAlignRPHIdist})
\item Decrease of amplification close to the edge ($d_{w}$ dependence).
\item Field distortion ($d_{w}$ dependence)
\end{itemize}  

The effect can be described using weighted Center of gravity function (see eq.\ref{eq:WeightCOG})
At the edge of the chamber the signal is atenuated by factor $w_i$, and cut at the pad
less then 0. The atenuation factor and Pad Responsense Function (PRF) width were measured independently and they were used on correction formula. The PRF is approximated by gaussian 
distribution. 
\begin{eqnarray}
    \Delta_{R\Phi}(y)=y_{\rm{COG}}-y=\frac{\sum_{i=0}^{N}{iw_if_{ri}}}{\sum_{i=1}^N{w_if_{ri}}}-y \\
    w_i = 1-k_ae^{-d_il_a} \\
    f_{ri}=e^{-(y_p-y)^2/(2\sigma^2)} 	
\label{eq:WeightCOG}
\end{eqnarray}

The correction formula describe the data down to 2 cm distance with precission bellow 0.5 mm
(see pic. \ref{figAngularAlignRPHIdistCorr} upper plot). For practical usage in tracking we reject the clusters with in the TPC region with $R-\Phi$ correction bigger than cluster position resolution  ($\approx 1 mm$).

\begin{figure}
  \centering\epsfig{figure=picDistortionRPHI/ycl_ytcm.eps,width=0.5\linewidth}
  \epsfig{figure=picDistortionRPHI/ycl_ytpad.eps,width=0.5\linewidth}
  \centering\epsfig{figure=picDistortionRPHI/dycl_ytcm.eps,width=0.5\linewidth}
  \epsfig{figure=picDistortionRPHI/dycl_ytpad.eps,width=0.5\linewidth}
  \caption{Clusters residual  at the edge of TPC sectors. $Y_0$ is the position of the edge pad.
	   For y bellow 0.5 pad width, the COG of cluster is almost independent of the track position. Non linear effect is observed up to distance 2 pad width. Small difference between the different pads can be explained by different Pad response function width.
	  }
  \label{figAngularAlignRPHIdist}
\end{figure}

\begin{figure}
  \centering\
  \epsfig{figure=picDistortionRPHI/rphi_dist_max100.000000.eps,width=0.5\linewidth}
  \epsfig{figure=picDistortionRPHI/rphi_dist_max0.100000.eps,width=0.5\linewidth}
\caption{ Cluster residual at the edge of the TPC sectors. In upper part all clusters used, in lower part only the clusters with the estimated distortion bellow 1 mm used.}
\label{figAngularAlignRPHIdistCorr}
\end{figure}


\section{ Alice TPC alignment}

Problems with y and phi alignment in the OROC -left right alignment. 
Systematic shift in y and phi.
Shift observed mainly for the short track. 
The effect was reduced making stronger cut on pt matching and pt resolution.




\begin{figure}
  \centering\epsfig{figure=picAlignMag5/SigmaY_z.eps,width=0.5\linewidth}
  \centering\epsfig{figure=picAlignMag5/DeltaY_z.eps,width=0.5\linewidth}
  \centering\epsfig{figure=picAlignMag5/PullY_z.eps,width=0.5\linewidth}
  \caption{Field 0.5 T data. Extracted sigma, mean and pull in y matching between IROC and OROC}
  \label{figDeltaP0}
\end{figure}

\begin{figure}
  \centering\epsfig{figure=picAlignNoMag/SigmaY_z.eps,width=0.5\linewidth}
  \centering\epsfig{figure=picAlignNoMag/DeltaY_z.eps,width=0.5\linewidth}
  \centering\epsfig{figure=picAlignNoMag/PullY_z.eps,width=0.5\linewidth}
  \caption{No field data. Extracted sigma, mean and pull in y matching between IROC and OROC}
  \label{figDeltaP0}
\end{figure}

\begin{figure}
  \centering\epsfig{figure=picAlignMag5/SigmaZ_z.eps,width=0.5\linewidth}
  \centering\epsfig{figure=picAlignMag5/DeltaZ_z.eps,width=0.5\linewidth}
  \centering\epsfig{figure=picAlignMag5/PullZ_z.eps,width=0.5\linewidth}
  \caption{Field 0.5 T data. Extracted sigma, mean and pull in z matching between IROC and OROC}
  \label{figDeltaP1}
\end{figure}


\begin{figure}
  \centering\epsfig{figure=picAlignNoMag/SigmaZ_z.eps,width=0.5\linewidth}
  \centering\epsfig{figure=picAlignNoMag/DeltaZ_z.eps,width=0.5\linewidth}
  \centering\epsfig{figure=picAlignNoMag/PullZ_z.eps,width=0.5\linewidth}
  \caption{No field data.Extracted sigma, mean and pull in z matching between IROC and OROC}
  \label{figDeltaP1}
\end{figure}


\begin{figure}
  \centering\epsfig{figure=picAlignMag5/SigmaP4_z.eps,width=0.5\linewidth}
  \centering\epsfig{figure=picAlignMag5/DeltaP4_z.eps,width=0.5\linewidth}
  \centering\epsfig{figure=picAlignMag5/PullP4_z.eps,width=0.5\linewidth}
  \caption{Field 0.5 T data. Extracted sigma, mean and pull in curvature matching between IROC and OROC}
  \label{figDeltaP1}
\end{figure}

\begin{figure}
  \centering\epsfig{figure=picAlignNoMag/SigmaP4_z.eps,width=0.5\linewidth}
  \centering\epsfig{figure=picAlignNoMag/DeltaP4_z.eps,width=0.5\linewidth}
  \centering\epsfig{figure=picAlignNoMag/PullP4_z.eps,width=0.5\linewidth}
  \caption{No field data. Extracted sigma, mean and pull in curvature matching between IROC and OROC}
  \label{figDeltaP1}
\end{figure}



\begin{figure}
  \centering\epsfig{figure=picAlignNoMag/SigmaP4_z.eps,width=0.5\linewidth}
  \centering\epsfig{figure=picAlignNoMag/DeltaP4_z.eps,width=0.5\linewidth}
  \centering\epsfig{figure=picAlignNoMag/PullP4_z.eps,width=0.5\linewidth}
  \caption{No field data. Extracted sigma, mean and pull in curvature matching between IROC and OROC}
  \label{figDeltaP1}
\end{figure}



\section{ Alice TPC alignment comparison of data with and without field.}


\begin{figure}
  \centering\epsfig{figure=picAlignComp/mag5dPhi.eps,width=0.5\linewidth}
  \centering\epsfig{figure=picAlignComp/nomagdPhi.eps,width=0.5\linewidth}
  \centering\epsfig{figure=picAlignComp/diffnomagmag5dPhi.eps,width=0.5\linewidth}
  \caption{$\phi$ Angular alignment. Data with and without magnetic field}
  \label{figDeltaPhi}
\end{figure}

\begin{figure}
  \centering\epsfig{figure=picAlignComp/mag5dTheta.eps,width=0.5\linewidth}
  \centering\epsfig{figure=picAlignComp/nomagdTheta.eps,width=0.5\linewidth}
  \centering\epsfig{figure=picAlignComp/diffnomagmag5dTheta.eps,width=0.5\linewidth}
  \caption{$\theta$ Angular alignment. Data with and without magnetic field}
  \label{figDeltaTheta}
\end{figure}

\begin{figure}
  \centering\epsfig{figure=picAlignComp/mag5dZ.eps,width=0.5\linewidth}
  \centering\epsfig{figure=picAlignComp/nomagdZ.eps,width=0.5\linewidth}
  \centering\epsfig{figure=picAlignComp/diffnomagmag5dZ.eps,width=0.5\linewidth}
  \caption{Z alignment. Data with and without magnetic field}
  \label{figDeltaZ}
\end{figure}

\begin{figure}
  \centering\epsfig{figure=picAlignComp/mag5dY.eps,width=0.5\linewidth}
  \centering\epsfig{figure=picAlignComp/nomagdY.eps,width=0.5\linewidth}
  \centering\epsfig{figure=picAlignComp/diffnomagmag5dY.eps,width=0.5\linewidth}
  \caption{Y alignment. Data with and without magnetic field}
  \label{figDeltaY}
\end{figure}
















\end{document}




