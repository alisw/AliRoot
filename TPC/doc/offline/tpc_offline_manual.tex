\documentclass[a4paper,12pt]{article}
\usepackage{graphicx}
\usepackage[dvips]{epsfig}
\usepackage{a4wide}
\usepackage{amsmath}
\parindent=0pt

\begin{document}
  \title{TPC Offline Manual}
  \author{P.~Christiansen, H.~Helstrup, M.~Ivanov, M.~Kowalski, 
    and J.~Wiechula} 
  \date{v0.1, January 31, 2008} 
\maketitle
\begin{abstract}
  This document is meant to give an overview of TPC offline code
  relevant for expert shifters and for people interested in analyzing
  TPC data.
\end{abstract}

\tableofcontents

\newpage 



\section{Environment setup on lxplus cern.ch}

This part is valid only for login on lxplus.cern.ch. In case of problems contact marian.ivanov@cern.ch.


{\bf{Alien setup:}}
\begin{itemize}
\item Use bash shell: {\it bash}
\item {\it{source /afs/cern.ch/user/m/miranov/public/.balien}}
\item {\it alien-token-init \$USERNAME}
\item Default message example -  Your token is valid until: Sat Feb  9 15:21:17 2008
\item In case of problem be sure that your certificate is fine.
      e.g. Check the presence of the .globus director: 	ls ~/.globus 
\item {\it source /tmp/gclient\_env\_\$UID}
\end{itemize}

{\bf{AliRoot setup:}}
\begin{itemize}
\item The central - default aliroot setup - type {\it new}
\item The TPC setup -  regularly updated HEAD version of HEAD. Use this setup if you want to use TPC specific features. The will be  \\
      {\it source //afs/cern.ch/user/m/miranov/public/.balice}
\end{itemize} 




\section{Data Location and Access}


\subsection{Raw data}

The naming scheme for the raw files is:\\
\texttt{<year2digits><runno9digits><gdc3digits>.<fileno>0.root}\\
%FIXME: is this true?
Example: 07000015067004.10.root

A single tag file is created for each GDC.\\

The directories depends on whether the run was recorded in standalone
or in global mode. Unfortunately it is not always possible to see from
the logbook overview if a run was started as a stand alone or a global
run (DAQ is always standalone).\\

% FIXME: is this true?

\textbf{Standalone runs:}\\

CASTOR: \texttt{/castor/cern.ch/alice/tpc/<year>/<month>/<day>/<hour>} \\
AliEn: \texttt{/alice/data/<year>/LHC<year2digits><quarter>\_TPC/<runno>/raw/} \\

Where \texttt{<quarter>} can be w (winter).\\

% FIXME: update with all 4 quarters

\textbf{Global runs:}\\

CASTOR: \texttt{/castor/cern.ch/alice/data/<year>/<month>/<day>/<hour>}\\
AliEn:  \texttt{/alice/data/<year>/LHC<year2digits><quarter>/<runno>/raw/}

\subsubsection{Access to raw data}

The files located on CASTOR can be accessed directly from
\texttt{lxplus} machines where one can use \texttt{nsls} to list files
and \texttt{rfcp} to copy files to local disk. The files can also be
accessed directly inside a ROOT session if ROOT is compiled with rfio
option (\texttt{--enable-rfio}).

Example of ROOT filename:\\
\texttt{rfio:/castor/cern.ch/alice/tpc/2007/12/13/05/07000011524011.6380.root}\\
Nb! You have to specify \texttt{rfio:} to let ROOT know that it is a
remote/CASTOR file you wish to open.\\

The data registered in AliEn can be directly streamed in ROOT (if ROOT
is compiled with AliEn support).  One has to get an AliEn token
(\texttt{alien-token-init}) and then do\\
\texttt{Tgrid::Connect("alien://");}\\
inside root, aliroot or alieve.

Example of ROOT path-name:\\
\texttt{alien:///alice/data/2007/LHC07w\_TPC/000010897/raw/07000010897011.20.root}\\
NB! \texttt{alien://} tells ROOT that it is an AliEn file you wish to
open.

\subsection{Reconstructed data}

The automatically reconstructed data can be found on AliEn. Instead of the \texttt{raw} directory, the files are stored under \texttt{ESDs}.

Example of directory with reconstructed files:\\
\texttt{/alice/data/2007/LHC07w/000014493/ESDs/00120/}\\ 
The name of the directory \texttt{00120} corresponds to the last 5
digits in the raw file name.

\subsubsection{Analysis of reconstructed data}

An example AnalysisTask for looping over the clusters in reconstructed
tracks is under development.

An early working example can be found at:\\
\texttt{http://www.hep.lu.se/staff/christiansen/tpc\_ana\_track\_example.tgz}

\newpage 

\section{Data Monitors}

There two types of monitors. The TPC pad monitor and the standard
AliEve monitors.

\subsection{The TPC Pad Monitor}

The TPC pad monitor is maintained by Stefan Kniege as both an AliRoot
and standalone application. The standalone application is used for
installation on the TPC shift machine in the counting room.

The AliRoot version can be run from inside an aliroot session like this:

\begin{verbatim}
.L $ALICE_ROOT/lib/tgt_linux/libTPCmon.so
.L $ALICE_ROOT/TPC/macros/TPCMonitor.C
TPCMonitor()
\end{verbatim}

Now the GUI starts and you should click:\\
\texttt{Sel.Format} - Select ROOT and remember to press 
\texttt{Select Entry}\\
\texttt{Sel.File} - Open file - can soon be alien file\\
Press \texttt{Next Event} until the data is loaded\\

\subsection{AliEve}

AliEve is part of AliRoot and is a general purpose 3d graphical
monitoring framework. It is called with the command
\texttt{alieve}. The monitoring display for different detectors is
controlled by macros. For the TPC, AliEve can be used to display both
raw data and reconstructed data.\\

To run the TPC raw display one has to call the following macro inside alieve:

\begin{verbatim}
.L ../test-macros/tpc_gui.C 
tpc_gui(<filename>, <event_no>) 
\end{verbatim}

In the GUI you can double click on TPCLoader to control the many
different draw options. 

Click \texttt{Help} to see how to navigate in the opengl window. You
can also get the menu of graphical elements by clicking on them
(\texttt{SHIFT+right click}) and see 1d projections of the raw data
(\texttt{CTRL+left click}).

% FIXME Need description of ESD monitor. Especially the nice one used by Marian.

\newpage

\section{Calibration, Shuttle and QA}

\subsection{Calibration Classes}

% FIXME: Needs input

\subsubsection{AliTPCdataQA}

This special calibration class is used for QA of raw data, but can
also be run standalone. It is currently under development. The problem
is how to extract relevant parameters from the raw data which are not
dominated by the large noise.

\subsection{The Shuttle}

% FIXME: Needs input

\subsection{TPC QA}

A first version of the TPC QA have recently been committed. It
contains the basic structure and a few histograms.

The TPC QA is made in the standard ALICE QA framework and contains the
3 classes:\\
\texttt{AliTPCQAChecker, AliTPCQADataMakerSim,
AliTPCQADataMakerRec}.

The DataMaker classes makes the histograms from the relevant data ---
hits and digits for \texttt{Sim}, and raw, rec points and ESDs for
\texttt{Rec} --- and the Checker contains the rules for comparing the
histograms to the reference histograms (which currently does not
exist). The Checker is currently empty which means that the default
comparison is employed.

The implementation is slightly different than for the other QA classes
since we have chosen to have histogram pointers in the class to make
the class easier to debug. The only place where this could cause
a problem is in the copy constructor where the histogram pointers have
to be assigned after the ``lists'' have been copied. This has been
implemented, but not tested.

In the QA Checker for raw data we use the more powerful TPC
calibration framework (the class AliTPCdataQA, see above) so that the
method for handling the raw data QA is non-standard. Instead of
filling the histograms in the \texttt{MakeRaws} method, the
calibration object is filled, and it is first in the method
\texttt{EndOfDetectorCycle} when we have called
\texttt{AliTPCdataQA::Analyse} that the histogram is projected from
the calibration object and added to the list.

The QA can be run like this inside aliroot:

\begin{verbatim}
  AliQADataMakerSteer* dataMaker = new AliQADataMakerSteer();
  dataMaker->Run("TPC", <type>, filename);  
\end{verbatim}

Where \texttt{filename} does not have to be supplied for data in the
same directory (except for raw data), and \texttt{<type>} can be:

\begin{verbatim}
AliQA::kRAWS
AliQA::kHITS
AliQA::kSDIGITS
AliQA::kDIGITS
AliQA::kRECPOINTS
AliQA::kESDS
\end{verbatim}

\newpage

\section{Simulation and Reconstruction}

This section gives a short overview of the TPC simulation and
reconstruction software. Other sources of information can be found in
the ALICE TPC TDR~\cite{tpctdr}, the ALICE Physics Performance
Reports~\cite{Carminati:2004fp,Alessandro:2006yt} (1: p.1620-1627,
updated description of TPC, 2: p.~1326-1336, description of
clustering, tracking algorithms and performance).

The TPC simulation and reconstruction consists of the following steps:
\begin{itemize}
\item Energy loss simulation. The physics routine which is called by
e.g. GEANT3. Output: TPC.Hits.root.
\item Signal generation. The electrons drift to the ROCs and are
amplified and digitized. Output: TPC.Digits.root.
\item Clustering. The first step of reconstruction is the grouping of
signals into clusters. The same algorithm is applied to real data and
simulated data. Output: TPC.RecPoints.root.
\item Tracking. The last step of reconstruction is the grouping of
clusters into tracks. The same algorithm is applied to real data and
simulated data. Output: AliESDs.root.
\end{itemize}
This section gives a short overview of the TPC simulation and
reconstruction software. Other sources of information can be found in
the ALICE TPC TDR~\cite{tpctdr}, the ALICE Physics Performance
Reports~\cite{Carminati:2004fp,Alessandro:2006yt} (1: p.1620-1627,
updated description of TPC, 2: p.~1326-1336, description of
clustering, tracking algorithms and performance).

The TPC simulation and reconstruction consists of the following steps:
\begin{itemize}
\item Energy loss simulation. The physics routine which is called by
e.g. GEANT3. Output: TPC.Hits.root.
\item Signal generation. The electrons drift to the ROCs and are
amplified and digitized. Output: TPC.Digits.root.
\item Clustering. The first step of reconstruction is the grouping of
signals into clusters. The same algorithm is applied to real data and
simulated data. Output: TPC.RecPoints.root.
\item Tracking. The last step of reconstruction is the grouping of
clusters into tracks. The same algorithm is applied to real data and
simulated data. Output: AliESDs.root.
\end{itemize}

\subsection{Simulation}

The simulation is currently working for GEANT3 and FLUKA simulations. 

How to switch between GEANT3 and FLUKA? In Config*.C macro one has to:
\begin{itemize}
\item replace    \\
// libraries required by geant321\\
$\sharp$ if defined($_{--}$CINT$_{--}$)\\
      gSystem->Load("libgeant321");\\
$\sharp$ endif\\

      new     TGeant3TGeo("C++ Interface to Geant3");\\

 with \\
     gSystem->Load(''libTFluka'');\\
     new TFluka("C++ Interface to Fluka",0/*verbositylevel*/);\\
     
\item add a line TPC->SetPrimaryIonisation();  
\end{itemize}

\subsubsection{TPC versions}
We have 4 TPC versions:
\begin{itemize}
\item AliTPCv0 - no sensitive volumes, geometry as in other versions
\item AliTPCv1 - fast simulation - the sensitive elements are thin gas strips
placed at the pad center, where space-points are created, then the coordinates 
are smeared according the space-point resolution
\item AliTPCv2 - the defaul version - detailed simulation \\
In the case of GEANT3 the distance to the next primary ionization is calculated
in \texttt{AliTPCv2::StepManager} method. In case of FLUKA it is handled by 
the FLUKA code itself.
The energy loss simulation is well
described in the TPC TDR~\cite{tpctdr}.
\item AliTPCv3 - used only for the space-charge studies, the sensitive volume 
is the entire drift gas.
\item AliTPCv4 - used for Krypton simulations.
\end{itemize}
The comparison of the energy loss simulation with test beam data is
covered in the note~\cite{tpctune}.\\
The result of the simulation are hits - the space point coordinates and the 
track label.
Hits from the simulation are converted to Digits in: \\
\texttt{AliTPC::Hits2Digits()} which runs in a loop over sectors.
the relevant methods are:
\begin{itemize}
\item \texttt{MakeSector} (drift and amplification) - each electron is 
transported 
to the readout chamber and amplified.
\item \texttt{DigitizeRow} - for each electron at the given row the signal is 
calculated in the method \texttt{GetSignal}. This method returns the track 
label. For every digit labels of 3 tracks contributing mostly to this
digit are stored as well. This is controlled by the \texttt{GetList}
metod.
\end{itemize}
In time direction  a Gaussian response is assumed in pad direction we use 
the Gatti-Mathiesson function~\cite{math}.\\
It turned out recently that the
response function after filtering and baseline restoration is
asymmetric and one can eventualy change this.
\subsection{Reconstruction}

Parameters used by the reconstruction are set in the AliTPCRecoParam,
e.g., the CtgRange, which removes clusters with angles less than 45
degrees (default) with respect to the beam axis, where the track
density is too high in Pb+Pb collisions, but which are important in
cosmic ray data.

If one wants to have access to the TPC cluster information in the
reconstructed tracks it is important to call\\
\texttt{AliTPCReconstructor::SetStreamLevel(1)} and\\
\texttt{reco.SetWriteESDfriend(kTRUE)}\\
before calling the reconstruction.

\subsubsection{Clustering}

For each row in a sector the signals are stored in an array of size
Ntimebins $\times$ Npads. In the code there is a loop over this matrix
and for each peak (local maximum) one makes a cluster.
% FIXME: starting from the highest maximum and going down?

The clusters are matrices of 5 $\times$ 5 (pads $\times$ timebins) and
the position in space and time is calculated from a charge weighted
average over this matrix. The total charge $Q$ is the sum of all
charge and $Q_{\text{MAX}}$ is the charge at the center of cluster
(maximum).

For small clusters a symmetrization algorithm is applied to the cluster
which has been discussed if it should be removed since it introduces
systematic binning effects.

\subsubsection{Tracking}

% FIXME: To be done


\subsection{Reconstruction of raw data}

% FIXME: Should we have a grid example here?

\begin{thebibliography}{9}
  
\bibitem{tpctdr} ALICE Collaboration, ``Time Projection Chamber'',
  ALICE TDR 7, CERN/LHCC 2000-001.
\bibitem{math} J.S. Gordon and E. Mathieson, Nucl. Instrum. Methods 227 
(1984) 267;\\
E. Mathieson and J.S. Gordon, Nucl. Instrum. Methods 227 (1984) 277;\\
J.R. Thompson et al., Nucl. Instrum. Methods A234 (1985) 505; \\
E. Mathieson, Nucl. Instrum. Methods A270 (1988) 602.
%\cite{Carminati:2004fp}
\bibitem{Carminati:2004fp}
  F.~Carminati {\it et al.}  [ALICE Collaboration],
  %``ALICE: Physics performance report, volume I,''
  J.\ Phys.\ G {\bf 30}, 1517 (2004).
  %%CITATION = JPHGB,G30,1517;%%

%\cite{Alessandro:2006yt}
\bibitem{Alessandro:2006yt}
  B.~Alessandro {\it et al.}  [ALICE Collaboration],
  %``ALICE: Physics performance report, volume II,''
  J.\ Phys.\ G {\bf 32}, 1295 (2006).
  %%CITATION = JPHGB,G32,1295;%%

\bibitem{tpctune} P.~Christiansen and P.~Gros, for the Alice TPC
  Collaboration, ``Benchmarking the AliRoot TPC simulation against the
  2004 TPC test beam data'', draft for ALICE note.

\end{thebibliography}
\end{document}

